\documentclass{beamer}

\usepackage{listings}

\lstset{% general command to set parameter(s)
        frame=single, % frame code fragments
        breaklines=true, % break long lines
        basicstyle=\tt\small % print whole listing small
        %keywordstyle=\color{black}\bfseries\underbar, % underlined bold black keywords
        %identifierstyle=, % nothing happens
        %commentstyle=\color{white}, % white comments
        %stringstyle=\ttfamily % typewriter type for strings
        %showstringspaces=false % no special string spaces
       }

\mode<presentation>
{
  \usetheme{Warsaw}
  %\setbeamercovered{transparent}
}

\usepackage[english]{babel}

\usepackage[latin1]{inputenc}

\usepackage{times}
\usepackage[T1]{fontenc}
% Or whatever. Note that the encoding and the font should match. If T1
% does not look nice, try deleting the line with the fontenc.

\title{Dinner with Ambiants}
\subtitle{ICFP 2004 Programming Contest - Adapted version at University Utrecht, AFP 2005}

\author{Bogdan \textsc{Dumitriu} \and Jos\'e Pedro \textsc{Magalh\~aes} \and Huanwen \textsc{Qu}}

\institute
{
  Department of Computer Science\\
  Utrecht University
}

\date{30th June 2005}

% If you wish to uncover everything in a step-wise fashion, uncomment
% the following command:
%\beamerdefaultoverlayspecification{<+->}

\begin{document}

\begin{frame}
  \titlepage
\end{frame}

\begin{frame}
  \frametitle{Outline}
  \tableofcontents%[pausesections]
\end{frame}

\section{Defining a high level ant language}

\subsection{Features}

\begin{frame}
  \frametitle{High level features}
  \framesubtitle{The basics}
  \texttt{data HLI \ = \ldots}
  \pause

  \texttt{data Cond = \ldots}
  \pause
  \begin{block}{The features we all know}
    \begin{itemize}
    {\tt
        \item drop, turnLeft, turnRight :: HLI
        \pause \item markRaw, unmarkRaw :: Marker -> HLI
        \pause \item sense :: SenseDir -> Condition -> Cond
        \pause \item move, pickUp :: Cond
        \pause \item flip :: Int -> Cond
    }
    \end{itemize}
  \end{block}
\end{frame}

\begin{frame}
  \frametitle{High level features}
  \framesubtitle{Extensions}

  \begin{block}{\ldots and a few others:}
    \begin{itemize}
    {\tt \small
        \item label, goto :: String -> HLI
        \pause \item nop, break :: HLI
        \pause \item true, false :: Cond
        \pause \item not :: Cond -> Cond
        \pause \item (.\&), (.|) :: Cond -> Cond -> Cond
        \pause \item (.<) :: Cond -> HLI -> (HLI -> HLI)
        \pause \item (.+) :: (HLI -> HLI) -> HLI -> HLI
        \pause \item (.:) :: HLI -> HLI -> HLI
        \pause \item (.@) :: HLI -> Cond -> HLI
        \pause \item (.*) :: Cond -> HLI -> HLI
    }
    \end{itemize}
  \end{block}
\end{frame}

\subsection{Utilities}

\frame[all:1] {
  \frametitle{High level features}
  \framesubtitle{Defining utilities for easier programming}

  \begin{lstlisting}
unite :: [HLI] -> HLI
unite = foldr (.:) nop

repeatN :: Int -> HLI -> HLI
repeatN n = unite
            . take n
            . repeat

turnBack :: HLI
turnBack = repeatN 3 turnLeft

wanderAroundP :: Int -> HLI
wanderAroundP n = turnLeft .@ (move .& (flip n))
  \end{lstlisting}
}

\section{A compiler for the high level ant language}
\subsection{The compiler as a State Monad}

\frame[all:1] {
  \frametitle{The high level ant language compiler}
  \framesubtitle{implemented as a State Monad}
  Since the compiler has to keep track of state information,
we implement it using a State Monad:
  \begin{lstlisting}
data CompilerState =
  CompilerState
    { ic           :: Int
    , inDo         :: Maybe Int
    , jumpTo       :: Maybe (Int, Int)
    , env          :: FiniteMap String Int
    , compileGotoF :: Bool
    }

type Compiler a = State CompilerState a
  \end{lstlisting}
}

\begin{frame}
  \frametitle{The high level ant language compiler}
\begin{block}{Some remarks}
\begin{itemize}
    \item Nothing more than a (simple) compiler
    \pause \item Tracks code size and keeps an environment for the labels
    \pause \item Makes (naive) optimizations automatically
    \pause \item Could be extended with {\tt call}, but\ldots
        \begin{itemize}
            \pause \item Maybe that's not really necessary
            \pause \item That would generate too much code
        \end{itemize}
\end{itemize}
\end{block}
\end{frame}

\section{Example - our own ant}
\subsection{Strategy description}

\begin{frame}
  \frametitle{Strategy description of our ant}
  \framesubtitle{Initialization}
\begin{center}
      \pgfdeclareimage[interpolate=true,height=3cm]{anthillMarked}{../images/anthillMarked}
      \pgfuseimage{anthillMarked}
  \end{center}
    \begin{block}{Initialization procedures}
    \begin{itemize}
        \pause \item Form a protective storage area where all food is
        dropped
        \pause \item Mark paths to the storage area on all borders
        of the anthill
        \pause \item Carefully mark the exit point
    \end{itemize}
    \end{block}
\end{frame}

\begin{frame}
  \frametitle{Strategy description of our ant}
  \framesubtitle{Marking paths to food}
\begin{center}
      \pgfdeclareimage[interpolate=true,height=3cm]{pathToFood}{../images/antMarkPathToFood}
      \pgfuseimage{pathToFood}
  \end{center}
    \begin{block}{\small Food gathering}
    \begin{itemize}
    {\small
        \pause \item Ant finds food, picks it up and then returns home
        \pause \item Follows home markers and marks a new path to food
        \pause \item When exiting, walks around the anthill searching for a path to food
        \pause \item Paths that lead to no food are erased
    }
    \end{itemize}
    \end{block}
\end{frame}

\begin{frame}
  \frametitle{Strategy description of our ant}
  \framesubtitle{Home markers}
\begin{center}
      \pgfdeclareimage[interpolate=true,height=6cm]{worldMarked}{../images/worldMarked}
      \pgfuseimage{worldMarked}
  \end{center}
\end{frame}

\subsection{Excerpt of the code}
\frame[all:1] {
  \frametitle{The high level language - example}
  \framesubtitle{A collision handling routine}
\begin{lstlisting}
getOutOfTheWay :: Dir -> HLI
getOutOfTheWay d = unite [
 label $ tag "Get out of the way" d,
 wait 12,
 move
  .< goto (tag "Go home and mark" d)
  .+ (flip 2
      .< (turnRight .: tryMove .:
         (goto (tag "Go home and don't mark" (turn Right d))))
      .+ (turnLeft .: tryMove .:
         (goto (tag "Go home and don't mark" (turn Left d)))))]
\end{lstlisting}
}

\section{Conclusion}

\begin{frame}
  \frametitle{Conclusion}
  \framesubtitle{Analysis of the solution}
  The pros and cons of our EDSL:\pause
  \begin{block}{Pros}
	\begin{itemize}
	\item Rely on Haskell for functional style and correctness
	\pause \item Extending the low level ant language with many useful features
	%\pause \item 
	%\pause \item 
	%\pause \item 
	%\pause \item 
	\end{itemize}
  \end{block}
  \pause
  \begin{block}{Cons}
	\begin{itemize}
	\item The high level language might still be hard to use
	\pause \item Nested blocks require parenthesis
	\pause \item Compiler could (probably) optimize more
	%\pause \item 
	%\pause \item 
	%\pause \item 
	%\pause \item 
	\end{itemize}
  \end{block}\end{frame}


\end{document}
