\chapter{Future Development and Conclusions}

This chapter presents some ideas which we think could be taken into account in order to direct
future efforts in the group. We end the chapter with some brief conclusions which try to sum up our
achievements.

\section{Future development}

There are several things which we thought about during our work worth mentioning here, hoping
that they might be useful to current or future members of the group. These are all just ideas,
not actual solutions to clear problems and should be regarded as such. Some of them might turn
out to be feasible, while others might turn out to be rather difficult to achieve.

\subsubsection{Mixed text/graphical editor}

As currently we have two distinct asynchronous editors available, a text editor and a graphical
editor, it seems like a good idea to work towards an integration of the two, such that we can
create a system which would allow the user to insert graphics within the text (s)he is editing.
The way we see this, the system should treat the graphic unit as a special character (which could
seen as be a separate semantic unit, either a word or a sentence or a paragraph). In this way,
the text algorithm can be applied without any modifications (with the exception of the introduction
of the special character represented by the graphic) in order to edit the document. None of the
operations, however, would have to be modified because if the graphic is a special character then
we can simply use the same insertion/deletion operations as the ones for normal characters. The
benefit of this would be that the merging algorithm would work exactly in the same way and no new
inclusion/exclusion procedures would have to be created.


As far as the graphic itself is concerned, the graphical editing subsystem could be used there
to facilitate its collaborative editing. In this way, the system would automatically switch
between text and graphical style of asynchronous collaborative editing and store separate logs
for the text editor as well as for each separate graphic that appears in the document.

This would be a feature with a lot of potential for most users, since there are many cases in
which text documents need to include figures representing or describing the concepts from the
text. The idea seems to have a rather straightforward implementation (given that the two subsystems
already exist and seem to be working well), but there might be unexpected issues which might arise
during the integration.

Aside from allowing the simple graphics which the existing graphical editor currently supports,
the system could also permit the integration of various image formats into the editor. Of course,
there would be no
possibility to edit such images (especially not in a collaborative manner), but still they
could be inserted and deleted. They too would have to be regarded as a special character from
the point of view of the text editor, but they would have to be differentiated from the normal
graphics since there would be no possibility of editing them.

\subsubsection{Extension of the editor features}

A more difficult extension of our text editor would consist in adding attributes (such as
font, size, color, bold, italic and underlined text) to the characters. We say this would
be significantly more difficult than
the previous idea because both the data structure representing the document and the set of
operations would have to be changed. The data structure would have to provide the possibility
of knowing exactly what attributes each semantic unit has (and overwriting of the attributes
by child nodes). The set of operations, on the other hand, would have to be extended in order
to offer the possibility of changing each attribute separately.

These changes would have further implication in most of the algorithms used throughout the
system. The merging algorithm would have to be transformed in order to cope with new types
of conflicts and with new types of operations. The compression algorithm would have to be
extended in order to be able to compress attribute changes both by means of combining several
attribute changing operations at a lower level into a single attribute changing operation at
a higher level and by means of canceling operations which have been overwritten by newer
ones (this is case with all attribute changing operation which change the same attribute of
the same semantic unit).

The most difficult issue, however, would be to change the inclusion/exclusion functions in
order to deal with the fact that some of the new operations might have an effect on more than
just one semantic unit (for example, the user could change the font of two and a half paragraphs).
This would probably imply changing the whole concept of inclusion and exclusion, because it might
be the case that the effect of including an operation might change the current operation to
two different new operations (think of two users, one of which changes the color of an entire
paragraph, while the other changes the color of a sentence - neither the first, nor the last -
in that paragraph to a different color than the first user; when the change paragraph color
would include the change sentence color it would have to be transformed in at least two new
operations - one that changes the color of all the sentences before the one changed by the
second user and one that changes the color of all the sentences after that one).\\
\\
By combining this development of the text editor and the one mentioned previously we would
obtain a collaborative editor suited for high-level text editing with support for fonts, color,
text size and graphics. This is, of course, still far from what current (non-collaborative)
text editors offer (support for creating tables, text alignment, etc.) but it would a significant
step forward. The final goal would obviously be the creation of an collaborative editor which
would offer all the features of current non-collaborative editors, but there is still a lot
of work which has to be done before reaching that stage.

\subsubsection{Extending the current data structure}

As we have mentioned throughout this paper, the current merging algorithm is scalable to
working on trees with more than just the four current levels (paragraph, sentence, word
and character). We therefore propose as a project topic the extension of the current editor
with what we call editor modes. These could be selected by the user based on the type of
document (s)he is working on. There could be several predefined modes (such as book, report,
article, etc.) which would imply the use of various tree structures and sets of operations
by the editor.

In a book mode, for example, certain semantic units should be added to the model (such as
chapters, sections and subsections). Operations should also support
the insertion/deletion of such units. The issue that would still have to be dealt with is
that of finding the right separators for such units. Currently, it is pretty obvious what
the paragraph, sentence and word separators are. In the case of units like chapters, however,
the situation is bit more complicated. Possible solutions would be to introduce some special
markers as chapter separators or to decide that a certain number of $\backslash$n's separate
a chapter from the other.

Either way, such editing modes would contribute to an even more increased efficiency of the
entire system, especially if the chosen editing mode corresponds to the actual type of document
that is being edited. To further support the idea, we believe that such an addition would be
rather easily implementable.

\subsubsection{The split/merge issue}

Future work in this area could also focus on finding a better solution to the split/merge
problem than the one we used. In a nutshell, the split/merge problem is this: when the user
adds a character that is a word/sentence/paragraph separator in the middle of a
word/sentence/paragraph (s)he is effectively \emph{splitting} an existing semantic
unit into two new semantic unit; when the users deletes such a separator (s)he is
\emph{merging} two existing semantic units into one. The actual problem arises when
another user makes some modifications in the original semantic unit (in the case of
a split) or in one of the two original semantic units (in the case of a merge). An
operation encoding such a modification is very difficult to transform in order to be
executed on the site of the user which performed the split or the merge. More precisely,
it is very difficult to preserve user intention when doing such an operation. For example,
in the case of a split, when one user inserts a word in the same place where a sentence
separator (i.e. a '.') was inserted by the other user it is difficult to tell which of
the two new sentences the inserted word should belong to.

Currently, our solution to the problem is to treat a split as a deletion of the old semantic
unit and the insertion of the two new semantic units and a merge as the deletion of the two
old semantic units and the insertion of the new semantic unit. This ensures that the consistency
of the local working copies is kept, but strange effects may appear in certain cases. In
order to solve this, it is clear that two new special operations, split and merge, need to be
introduced and the inclusion/exclusion procedures need to be modified in order to take these operations
into account. However, this is not as simple as it sounds, as many research projects have dealt
with the issue and a fully working solution has not yet been found. The split/merge problem is,
consequently, still an open issue in the field of collaborative editing, and further efforts
can and should be directed towards solving it.

\subsubsection{XML editor}

Another idea that reaches a bit further than a mere extension of the current work would be
the creation of a collaborative XML editor. This is actually something we have attempted up
to certain extent. What we had in mind was the creation of a DTD-driven collaborative editor
which would take a DTD as input and allow several users to work together in order to graphically
edit an XML document which would have to conform to the specified DTD. The tricky part
here would be the ensuring of the conformity of the final document to the DTD (while allowing
intermediate, local, version to be incorrect). This is something we have have given some thought
to and appears to be no child's play.

The main difficulty arises from the fact that we wanted the entire editing to be done by means
of a graphical interface. The problem we got stuck on was how to create such an interface which,
on one hand, should not allow the user to stray away from the grammar but also be aware that
there might be cases when the grammar should be relaxed (since there might be other users
filling out certain parts of the document which, when merged with the work of the current user,
would yield a final correct document). Had we had more time to think about this, some solutions
to this problem might have appeared. Nevertheless, we think it is a good project topic for
future consideration.

\section{Conclusions}

This project had the purpose of studying and improving existing techniques in the field of 
asynchronous collaborative editing. The result of this work was the design and implementation
of an entire asynchronous editing system which allows a group of users to work together
in order to create and modify a text document. The special achievements of this project were
the creation of a system based on the hierarchical representation of the text documents and,
more importantly, a system whose algorithms work with hierarchical logs of operations (as
opposed to linear logs used in the other existing systems). We have successfully designed
some new algorithms especially suited for this purpose and have informally proven that the
performance improvements over their linear counterparts are significant.

We also think that another improvement worth mentioning is the possibility of allowing the
user to choose the granularity level (s)he wants to be working at. This, to the best of our
knowledge, is something no other system is currently offering. In addition to that, another
unique feature our system provides is the choice of the conflict resolution policy to be
employed, allowing for automatic as well as manual resolution with several choices in both
cases. Last, but not least, we are aware of no other system which offers the possibility of
direct user synchronization as ours does.

There are still some unsolved issues which need to be addressed in the future (some of which
we have mentioned in the previous section of this chapter), but the system as it is now is
a fully functional one which has proven to work correctly and also to be very helpful
with the editing of some chapters of this paper itself. We believe our work is valuable,
holds a lot of potential and is a prototype whose commercial implementation (with a few additions,
perhaps) would be most welcome by real users.

Our project fits into a greater project developed in the Global Information Systems group at
Eidgen\" ossische Technische Hochschule Z\" urich and current intentions suggest that it will be
part of a future integrated system which would allow various types of collaborative editing,
with the final aim of providing groups of people working together with a complete tool for editing
as many types of documents as possible.
