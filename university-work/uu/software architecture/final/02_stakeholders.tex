\chapter{Stakeholders}

In this part of the architecture document the different
stakeholders will be described and their concerns listed.
There will be a distinction between minor concerns and major
concerns, denoted with - and + respectively. A minor concern
is one that is important to some extent, but can be (partially
or totally, depending on the concern) dropped. A major concern,
by contrast, is one that is of great importance to its stakeholder
and which (s)he would like to see in the final system. However,
compromises can be (and have been) made even in terms of major
concerns. A compromise does not imply a total disregard to a certain
concern, but rather providing a solution which might only partially
address it. The following stakeholders have been considered:
\begin{itemize}
\item Users: staff of the SureThing Company
\item Acquirers: SureThing Company
\item Developers and Project Manager
\item Maintainers
\item System Administrator
\item Management (of the company doing the development)
\end{itemize}

\section{Users}

\subsubsection{Description}
The users are the people that will be working at the SureThing
Company. This group will be interested mostly in things like how
friendly the new system will be, if they have to learn new things
and how big the changes will be that are being made to the user
interface. The people visiting the web site of the SureThing Company
are not viewed as part of this stakeholder because it would be difficult
to sit with them around the table to discuss their concerns. We assume
that their major concern would be whether the web site gives them the
information they are looking for in an usable manner or not. To cope with
this, we have transferred this concern to the acquirers, since
they are stakeholder for which the customers' concerns are most
important.

\subsubsection{Concerns}
The concerns of the users are:
\begin{itemize}
\item[-] Availability during work hours: the system needs to be
fully operational especially in the period of time when they need it,
namely during work hours. Little or no downtime of the system is thus
desired in this period.
\item[+] User friendliness: the users want to work with a easy to use,
intuitive and clear system.
\item[+] Response time: the users are interested that the system
takes as little as possible for performing its actions and responds
quickly.
\item[-] Understandability, learnability and operability: this concern
expresses the wish of the users for the new system to be as easy to
understand, use and switch to as possible. The users hope not to have
to learn too many new things in order to be able to work with the system.
\end{itemize}

\section{Acquirers}

\subsubsection{Description}
The acquirers are the managers of the SureThing Company. This
group is mostly interested in making sure that all the goals put forward
in the assignment description are achieved. The system which the present
document describes needs to be beneficial to their business and in many
ways better than what they have now. Although they have a wide range
of concerns, they are primarily interested in making sure that the
resulting system will address their (company's) needs in a proper manner.

\subsubsection{Concerns}
The concerns of the acquirers are the following:

\begin{itemize}
\item[-] Realization of the system is in conformance with their
requirements.
\item[+] Scalability: as the number of customers increases,
the number of requests for proposal calculation will increase as well
and so will the use of most of the other components of the system
(such as the amount of output of the GOC, for example). Therefore it
should be possible to scale the system as the need arises in order to
cope with the increased load. Scalability is viewed mostly in two aspects:
web site availability and responsiveness and how the company's hardware
deals with the load.
\item[-] Suitability: the system should address the company's needs
and not the other way around.
\item[+] Security: the new system needs to provide at least as much
security as the current one does, but preferably even more. The stakeholder
underlines that the insurance business deals with highly sensitive data
which should stay in the company (as opposed to out in the open).
\item[-] Cost: the system needs to be improved at a reasonable cost in
terms of money and human resources, it needs to be finished by the
end of 2005 (but preferably sooner) and it should be as adaptable as
possible in order to prevent high costs in the future.
\item[+] Progressive evolution of the system: the old system should be
turned into the new one in several phases (also see our assumption on this).
\item[+] Correctness and usability during transition: during the transition
of the old system to the new one, the system has to be usable and behave
as expected the whole time.
\item[+] Interoperability: the system has to be based on and work as much
as possible with the already existing components, in order to keep costs low
and also in order to avoid the need for extensive retraining of people.
\item[-] Compliance to standards: the software of the system needs to be able
to connect to other software if necessary and, most importantly, the web site
needs to be browser independent (in other words, it needs to run on any of the
major browsers in use today).
\item[+] Fault tolerance: the system needs to be available 24 hours a day, 7 days
a week.
\item[-] Clarity of web interface to customers: this is in fact a concern of the
users of the web site, but since we have not identified them as stakeholder,
we make this a concern of the acquirers. Thus, the acquirers want their product
to be as appealing to and usable by their customers as possible.
\end{itemize}

\section{Developers and Project Manager}

\subsubsection{Description}
The developers and the project manager are the people that will
coordinate and implement the whole transition of the old system into the new system.
They are especially interested in understanding how the current system
works, what kind of changes they have to do to it, how these changes are
scheduled in time, how the new system is to be structured, what kind of
constraints they have to take into account and many more.

\subsubsection{Concerns}
The concerns of the developers and the project manager are as follows:
\begin{itemize}
\item[+] Modularity: the system needs to be as well structured as possible, so
that all components of the system can be developed, tested and replaced
independently of each other.
\item[-] Security: the developers are interested to know what kind of security
the system has to provide and what this security implies on their part. They
want to know which components of the system are most sensitive and what
should be done in order to protect them.
\item[-] Scalability: the stakeholder is concerned with what parts of the
system have to be scalable and what has to be taken care of in order to
ensure this scalability.
\item[+] Technical constraints: what are the technologies that have to be used
in order to develop/update the system? What knowledge does the use of this
technologies imply on their part? Also, it is of interest to know what underlying
systems are currently running the in the IT environment of the company, especially
if these are to be maintained in the new system.
\item[+] Understandability of the design: they would prefer the design to be
as intuitive as possible so as to be able to focus more on development and less
on deciphering the architectural description.
\item[-] Testability: the system has to be tested, so the stakeholder is interested
to know which parts have the highest correctness requirements so that they
can test them extensively.
\item[-] Development time: developers are worried about how strict the final
and intermediate deadlines are. They would like to be told the consequences of
exceeding development time.
\end{itemize}

\section{Maintainers}

\subsubsection{Description}
The maintainers are people that will have to solve any bugs or changing
requests of the system once it is released. They, as persons, might overlap
to some extent with the developers, depending on company policy, but
their concerns are nevertheless distinct. This group is interested if the
system is adjustable to the future needs of the company, if it is easy to
change, if it is easy to track down errors that might appear in it and if it
is appropriately documented.

\subsubsection{Concerns}
The following concerns are typical for the maintainers:
\begin{itemize}
\item[+] Changeability: if changes need to be made, this must
be done to a limited number of components and there must be a
documentation to know how. This goes partially hand in hand with the
modularity concern of the developers as a modular system is also easier
to change.
\item[+] Testability: the system needs to be easily testable in order to
find bugs and ensure its correctness now and in the future.
\end{itemize}

\section{System Administrator}

\subsubsection{Description}
The system administrator is the person (or group of people) that
is initially responsible for deploying the software on the existing hardware
configuration according to specifications and then for making sure that
the system is operational at all times. Also, the system administrator
is concerned with making backup copies of important information
so that it can be restored in case of hardware malfunction. Finally,
the system administrator also has to understand what hardware
and software changes have to be done once the current configuration is
no longer sufficient to deal with the load.

\subsubsection{Concerns}
The concerns of the system administrator will be mostly related to:
\begin{itemize}
\item[+] Security: what has to be done in terms of configuration
so that the system is secure (firewalls, protocols, configuration files,
etc.).
\item[+] Recoverability: it should be easy to spot what the problem
with the system is and it should be easy to solve the problem.
\item[+] Fault tolerance: is the system supposed to provided 24/7
access? If so, what has to be done from his/her part in order to ensure
this?
\item[-] Installability: the system and the new components should
be easy to install.
\item[-] Manageability: analysis of the running system should be
possible and modifications to it should be straightforward.
\end{itemize}

\section{Management}

\subsubsection{Description}
Management represents the interests of the developing company and
has to make sure that the proposed system can be done within
budget, has to make sure that the customer (i.e., the acquirer stakeholder)
is pleased with the result and also has to make sure that costs are
kept as low as possible and, more importantly, not exceeded.

\subsubsection{Concerns}
The following are the concerns belonging to the management:
\begin{itemize}
\item[+] Feasibility of constructing the system: management needs to
be convinced that the system, as described in the architecture, can be
put into practice within the money and time budget.
\item[-] Changeability: it is important to management that the
system be easily changeable as they believe the SureThing company might
have changing requirements in the future and that their company will
again have to work on the software, so it would be best if that implied
easy to do changes.
\item[+] Profit: they would like the architecture to be created in such
a way that implementation thereof can be done with as low costs as
possible in terms of time, personnel, resources, training and money
so that their profit can be higher.
\item[-] Costs might be exceeded: management wants to diminish the
risk that the costs allocated to the project are exceeded, as this would conflict
with their concern for making profit.
\end{itemize}
