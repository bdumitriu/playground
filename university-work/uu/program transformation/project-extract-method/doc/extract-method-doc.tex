\documentclass[a4paper,10pt]{article}
\usepackage{verbatim}
\usepackage{listings}

\begin{comment}

\end{comment}

\title{Program Transformation Project\\The \emph{Extract Method} refactoring in Java}
\author{Bogdan \textsc{Dumitriu}\\Jos\'e Pedro Rodrigues \textsc{Magalh\~aes}}

\begin{document}

\maketitle

\section{Description}
\emph{Extract Method} is the transformation that deals with
selecting a piece of code and extracting it to a new
method\footnote{or function, procedure or similar construct,
depending on the language.}, accordingly removing the old code and
invoking the new method. The behavior of the program must be
preserved in the transformation, with the objectives being:
\begin{itemize}
\item increased readability, since the parent method becomes
shorter, and the extracted code's purpose can be easily understood
if given a meaningful name;
\item reusability, since the extracted method might be useful somewhere
else.
\end{itemize}

Our project deals with the implementation of \emph{Extract Method}
in Java, using Stratego \cite{stratego} as our language of choice.

\section{General architecture}

\subsection{Overview}
The refactoring is achieved through a three-state process, namely:

\begin{enumerate}

\item Parsing an input Java source code file, which includes marking
for what fragment to extract as well as the name for the new method;

\item Performing the actual transformation;

\item Pretty printing the result as Java source code.

\end{enumerate}

The first part is performed with the aid of the SGLR parser fed with the
parse table for Java and an extension for delimiting the region to extract.

The second one uses our own code for the transformation as well as components
of \textbf{Dryad} \cite{dryad}, for ambiguous names reclassification
and type checking.

The last part is done with the existing hand-crafted pretty printer for Java.

\subsection{Information to collect}
To perform the extraction, the following information must be considered:

\begin{itemize}

\item Is the extraction really possible?

\item What are the various kinds of variables affected by or involved in
the process?

\item What exceptions are thrown in the fragment to extract and
which of them, if any, are not caught by potential \textsf{catch} clauses
that appear in the fragment?

\end{itemize}

\subsubsection{Is extraction possible?}
Some situations may not allow for extraction to be performed:

\begin{enumerate}

\item Does the code contain the markup for the fragment to extract?
If not, then the input file did not contain the necessary
information for extraction to be done.

\item Does the fragment contain any control-flow instructions? With
our implementation, a fragment cannot contain any of the
instructions \textsf{return}, \textsf{break}, \textsf{continue} or
\textsf{label}. Although some configurations could be allowed (like
a \textsf{return} statement at the end of the fragment or a
loop that is fully enclosed in the fragment to extract and contains
any number of \textsf{break} or \textsf{continue}
statements), most situations will imply that the normal flow of the
program is disrupted from the current block, thus not allowing the
extraction (since the fragment can only direct the flow back to the
parent method, by means of a \textsf{return}).

\item How many variables does the fragment modify? If the fragment
modifies --- by assignment --- more than one external variable (that
is, a variable declared before the fragment) which is also used
after the fragment, then the extraction cannot be done, since a
method can only return one variable. Some solutions would be
possible, but they would imply too big a change to the original
source code, thus contributing against the main purpose of the
refactoring, namely to increase readability. One such solution could
be to construct and use a holder class with setters and getters for
all the changed variables. Another solution could be to extend the
original class by adding some more fields to it, one for each
variable and use those for variable passing between methods.

\end{enumerate}

The extraction is possible only if all the conditions above are met.

\subsubsection{Variable information}
\label{vars}
In order to generate the new method, three sets of variables need to
be collected:

\begin{itemize}

\item Variables used in the fragment, but not assigned to in the fragment.
These have to be sent as parameters to the new method;

\item Variables whose value is changed by assignment in the fragment
and are then used after the fragment as well. If one variable is
changed inside the fragment, its value can be returned as the return
value of the extracted method and assigned to the original variable
in the calling fragment. If two or more variables of this type are
changed the refactoring will be disallowed (as we have already
discussed above);

\item Variables declared in the fragment and used after it. Extracting the
fragment to a different method would make the scope of the declaration
restricted to that method. The effect is that the usage of such a variable
after the fragment would result into a compilation error, due to the fact
that the variable would no longer be declared in that scope. The simple
solution to this problem is moving the declaration of the variable before
the fragment, and passing the variable itself as a parameter to the extracted
method.

\end{itemize}

\subsubsection{Exceptions}
For exception handling, we need to know all the exceptions that
every instruction in the fragment can throw. Furthermore, we need to
analyze which of those are already caught in any (possibly nested) \textsf{try -
catch} clause, not forgetting to consider that catching an exception
implies catching all the exceptions from the hierarchy of exceptions
rooted in the explicitly caught one. After effectively knowing which
exceptions are not caught, we have to add them to the \textsf{throws}
clause of the new method.

\section{Implementation details}

\subsection{One problem, two implementations}
Our implementation actually consists of two separate solutions,
although they share some commonalities. There is a naive
implementation, the first one we attempted, and afterwards, with the
gathered knowledge, we tried to come up with a more efficient and
elegant solution, which would achieve the same effect with a single
traversal on the program tree. However, since the second, improved
version was mostly worked as a ``proof of concept'' based on the
experience of the first one, we have not aimed to fully complete it.
Notably, it misses exception handling. Therefore, we supply the two
solutions: the first one\footnote{file \textsf{extract-method.str.}},
less efficient and perhaps less readable, but more powerful, and the
second one\footnote{file \textsf{better-extract-method.str.}}, which
is meant as a starting point of a reimplementation, but actually goes
quite a bit further than that.

\subsection{Parsing extended Java syntax}
To know what fragment to extract and what name to give to the new
method, we extended the syntax for Java with a new symbol
(\textsf{@}) to mark both the beginning and the end of the piece of
code to extract to a new method. The name to be be given to the extracted
method is expected as a string after the first \textsf{@} symbol.

In terms of implementation, this is essentially a one-line work:
\begin{lstlisting}[basicstyle=\small,breaklines=true,frame=single]
"@" Id BlockStm+ "@" -> BlockStm{cons("Extract")}
\end{lstlisting}
Of course this is followed by the generation of the new parse table
(by combining the original Java syntax definition and our extension),
and using the new parse table to parse files (instead of a plain
invocation to \textsf{parse-java}).

\subsection{\textbf{Dryad} issues}
Our implementation introduces a new constructor in the Java
representation: the \textsf{Extract(name, code)}, which is used to
name and mark the piece of code to extract. Unfortunately,
\textbf{Dryad} is not expecting its input files to contain such
markup, and thus fails when applied. We were then confronted with
the following problem: Dryad cannot easily be extended to support
extra constructors, but it had to be applied in order to reclassify
ambiguous names and obtain type information. We solved this issue by
converting the \textsf{Extract} markup to annotations: every
statement inside the block to extract is annotated with a
\textsf{{"extract-ann", name}}, where \textsf{name} is the name of
the method to extract, and the first string is just a markup for
later recognizing the annotation. After this we can apply
\textsf{dryad-reclassify-ambnames} and \textsf{dryad-type-checker},
and then reconvert the annotations back into the \textsf{Extract}
constructor.

Another problem with \textbf{Dryad} is that it currently does not
support field variables distinction. So far, our project works correctly
as long as there are no name clashes between the field names and
the variable names whose declaration we might have to move up
(the third category of variables described above). As soon as \textbf{Dryad}
will classify field names in a distinguishable way from local variables,
the project can be quickly updated with some minor modifications
to support this.

\subsection{\textsf{extract-method.str} overview}

Our first implementation can be found in the file
\textsf{extract-method.str}. Basically, it traverses the entire Java
code to collect all the necessary information about variables and
exceptions, binds all the information in the \textsf{config} hash
table and retrieves it any time it needs.

Most strategies are commented in the source code itself, making it
easy to understand how the whole information gathering and actual
rewriting happens throughout the process.

Perhaps an interesting thing to discuss about this version, especially
since it does not also appear in the other one, is the exception handling
mechanism. The strategy for exception collection (\textsf{define-exception-rules})
is based on three dynamic rules:

\begin{itemize}

\item \textsf{InTry} - this rule will be defined as long as the current
traversal point is within a (possibly nested) \textsf{try} block.
It's simply a marker to help us know whether we are inside a
\textsf{try} block or not;

\item \textsf{TempException} - this rule is appended with each exception
thrown by a method invocation. Also, every time a \textsf{try-catch}
block is exited, but the current point of traversal is still in a
higher-level \textsf{try} block, a \textsf{TempException} rule is
defined for each uncaught exception;

\item \textsf{PermException} - this rule will collect all the uncaught
exceptions which will eventually have to appear in the \textsf{throws}
clause of the extracted method. A \textsf{PermException} is defined
when a \textsf{try-catch} block is exited and the current point of traversal
is no longer inside a higher-level \textsf{try} block. All uncaught exceptions
thrown in the block are appended to \textsf{PermException}.

\end{itemize}

You should note that \textsf{TempException} is scoped at the level of
each \textsf{try} block, which means that all \textsf{TempException}'s
are automatically undefined when the \textsf{try} block is exited. The
only \textsf{TempException}'s which might still be defined after the
traversal are the ones which are outside any \textsf{try} block. Therefore,
a union of the \textsf{bagof-TempException} and the \textsf{bagof-PermException}
is taken after the traversal.

\subsection{\textsf{better-extract-method.str} overview}

The second implementation, available in the \textsf{better-extract-method.str}
file, provides an easier to understand solution, as well as more elegant
and efficient. However, is not as complete as the first implementation,
as explained before.

After similar ``preparation'' using Dryad, the bulk of this second
approach is contained in a main traversal which deals with all the
aspects of the problem. Collection of information is simulated by
making heavy use of dynamic rules and rewriting is done either
directly where possible, or by applying a dynamic rule defined
elsewhere than the place of application, where the information
needed for the rewriting is present. Careful scoping takes care
that the result is indeed the desired one.

The main traversal has specialized behavior for:

\begin{enumerate}

\item Class declaration: if the class contains the fragment, then
a new method will have to be added. This will be done by applying
the \textsf{RewriteClass} dynamic rule, defined in the handler
for the block which contains the fragment to extract (since it is
there where we have the information we need for creating this
rule).

\item Method declaration: this is only used to scope the \textsf{VarType}
rule, since this is the scope for the types of the method parameters and
the local variables defined inside the method.

\item Block level: many rules have to be scoped at the block level,
and if this block contains the code to extract then this is the
point where we have the necessary information to generate the rule
to rewrite the class. The block itself also has to be rewritten so that
the fragment of code to extract is replaced with a call to the new method.
This is done by using the dynamic rule \textsf{RewriteExtract},
mentioned below, when discussing the specialized behavior for
the fragment to extract.

\item Fragment to extract: here we define a rule to store the
fragment in. This will allow us to retrieve it later, when we need
to create the new method. A dynamic rule to rewrite the fragment
to a call to the extracted method (possibly preceded by the list of
declarations of those variables in the third category described in
section \ref{vars}) is then defined. This rule will be applied at
the level of the block containing the fragment.

\item Variable declaration, use or assignment: here a variable is
``added'' to the appropriate dynamic rule, depending on its category,
as defined in section \ref{vars}.

\item Control flow: if control flow instructions are found in the
fragment to extract, extraction will stop.

\end{enumerate}

\section{General remarks}

\subsubsection*{Scope for variables}

Why does moving the declarations of the third kind of variables
(that is, variables declared in the fragment to extract and used after it)
before the fragment work, you might wonder. The answer lies in the
way scoping in defined in Java. We are going to consider the cases here:

\begin{itemize}

\item the variable is at the top level (i.e., not within an additional
block) in the fragment to extract, which makes it visible in the
rest of the fragment, as well as after it. In this case, moving the
declaration before the fragment will ensure a slightly larger
visibility scope. We have these cases:

\begin{itemize}

\item the variable name is not used between the beginning of the
fragment and the declaration. Then, moving its declaration up
is harmless.

\item the variable name is used between the beginning of the
fragment and the declaration (obviously, referring to another
variable). In this case, simply moving the declaration up would
render a different semantics, since the use would now refer to
the moved up declaration. However, this can easily be solved.
We know that the variable's use cannot refer to another local
declaration or even parameter name, since Java doesn't allow
redeclaration of variable names. Then, the variable must either
refer to a field name, may it be static or non-static or to a static
variable of another class (since Java 1.5, a class can be
statically imported, thus making it possible to refer to a
public/protected static variable of that class without prefixing
it with the class name). In order to solve this case, we simply
have to prefix the field name with \textsf{this.} (if it is a
non-static field) or with the class name followed by a `.' (if
it is a static field). We are currently not doing this, since Dryad
doesn't yet offer support for distinguishing field names from
local parameters or variables. But it can easily be done, once
support is available.

\end{itemize}

\item the variable declaration is nested in an extra block within
the fragment. In this case, we can be sure that it won't be
used after the fragment since the fragment can only be
extracted as a new method if all blocks therein are fully contained
inside it and since the scope of a variable declared in a block
is limited to that block.

\end{itemize}

\subsubsection*{How clear is the code?}

We have gone at great length to make our code as clear as
possible, by commenting it quite thoroughly, by laying it out
in a readable fashion, by separating things into strategies
as much as possible, by using meaningful names, by refactoring
things which we realized could be done easier and by various
other techniques. Last, but not least, the fact that we have
created a brand new extra implementation should stand as
proof for our drive to make the code clearer.

The only thing which might make it less clear is the heavy
use of dynamic rules (in our second implementation), but
given the complexity of the refactoring, there is hardly anything
we could do about this.

And, agreed, the first solution might be made a bit more difficult
to understand by the combination of a more imperative style of
programming with the classic Stratego style, hence the reason
for creating our second implementation.

\subsubsection*{Is Stratego appropriate for the transformation?}

This issue can be regarded in two ways. On one hand, Stratego is
the perfect tool for implementing such a refactoring, considering
the amount of tools which it offers. Not only is parsing and
pretty-printing fully covered by java-front, but it also provides a
very useful toolset through the Dryad package, that came in extremely
handy. We are sure that without such a supporting platform, the
work involved by this type of refactoring is a very tedious task, to
say the least.

On the other hand, the lack of support for more elaborate debugging
combined with lengthy times for compilation of Stratego code to C
makes working with Stratego on a project of this scale a living hell.
We honestly believe we have spent towards half of the working
hours adding \textsf{say} or \textsf{debug} statements and
recompiling the code.

But, to end on a positive note, we are fully convinced that
the advantages of Stratego clearly surpass its disadvantages, especially
since the latter are mostly due to its current state of development, more
than to structural faults.

\begin{thebibliography}{99}

\bibitem{dryad} M. Bravenboer. \textbf{Dryad --- The Tree Nymph}
Available from
http://catamaran.labs.cs.uu.nl/twiki/pt/bin/view/Stratego/TheDryad

\bibitem{refactoring} M. Fowler, et al. \textbf{Refactoring: Improving the Design of Existing Code
(Hardcover).} Addison-Wesley Professional, 1999.

\bibitem{stratego} E. Visser. \textbf{Program Transformation with Stratego/XT:
Rules, Strategies, Tools, and Systems in StrategoXT-0.9.} In C.
Lengauer et al., editors, Domain-Specific Program Generation, volume
3016 of Lecture Notes in Computer Science, pages 216--238.
Spinger-Verlag, June 2004.

\end{thebibliography}

\begin{appendix}

\section{File \textsf{extract-method.str}}
\lstinputlisting[basicstyle=\tiny,breaklines=true,frame=single]{../extract-method.str}

\section{File \textsf{better-extract-method.str}}
\lstinputlisting[basicstyle=\tiny,breaklines=true,frame=single]{../better-extract-method.str}

\end{appendix}

\end{document}
