\chapter{Guide to the Architecture}

\section{Incremental approach}
We have created our architecture in such a way that the transformation
of the old system into the new one takes place in several phases. This is
necessary because of the desire to have an incremental upgrade and is
also profitable for the acquirer as it ensures early benefit of the changes
that are already made. In addition, it is better for testing and user
acceptance of the new system in small steps. As mentioned, during this
transition the system will have to be working correctly and be usable
the whole time. The transition, as explained, will consist of several
phases, each of them including a number of steps. There will also be
steps that span across 2 phases, but these represent isolated
cases, as most steps will be completed in one phase only. Below, we
describe the steps we have divided the upgrade into.

\begin{enumerate}

\item \textbf{UCIS refactoring to benefit from on-demand proposal request
processing by STIFF}: We do this step as soon as possible because it
implies quite small changes and because it is necessary in order
to allow the customer (which, at first will be just the Call-Center
employee, but once step 5 is also completed, will also be the regular
customer) to benefit from real time proposal calculation

\item \textbf{Create STIFF wrapper to emulate future on-demand protocol
on old STIFF system}: STIFF currently calculates the proposal request
over night. After the addition STIFF will be able to do calculations
right away. We achieve this by adding a wrapper which will start
STIFF on request. This might not be real time yet, but still it will be
considerably faster than before

\item \textbf{Migration of ForSure data from Office documents to Oracle database}:
Currently, the data of ForSure is spread in a non-standardized
way among a number of Microsoft Office documents. Clearly, it is
impossible for an automated system to work with such data, which
implies that all of it has to be transferred into the Oracle database
of Zeker \& Vast, where it will be stored in a structured manner

\item \textbf{Rest of UCIS refactoring with GOC integration}: We refactor the
business logic of the UCIS system so that all the business logic currently
existing on the client-side is moved to the server-side, leaving nothing
but the presentation to be done on the client. This step is necessary
before we can advance to the next one, in which we transform
the presentation layer into HTML. During the refactoring, we also need
to have in mind that all output has to be now forwarded to the GOC
component, which is also introduced as part of this step

\item \textbf{Porting of UCIS client GUI to website}: This will be the step
at the end of which it will be possible for customers to access SureThing's
system directly, through the Internet

\item \textbf{Update CHIPS and BuRP to interface with the new UCIS}: As UCIS
is modified, CHIPS \& BuRP will have to be updated in order to use the
new interfaces of UCIS (if necessary)

\item \textbf{Rewriting of STIFF in Java for scalability and caching of data}: While
STIFF is still in its current form, running multiple instances of STIFF at the
same time is not possible (reasons are explained in the process view),
which means that it cannot be scaled to run on multiple machines at the
same time. In order to allow such scalability, we need to rewrite STIFF
from scratch and, since we are doing it anyway, we propose to rewrite it
in Java since this way it would integrate easier with UCIS (which is also
written in Java).

We will also want to make the response time even smaller by caching
product information from the database in STIFF's memory. There is a point
to this since we expect product information to change very rarely (in terms of
days). There is thus no point in rereading such information from the
database every time a proposal has to be computed. However, in order
for this system to work, STIFF has to be notified when the product
information in the database changes. Creating such a notification
mechanism is also the purpose of this step.

\item \textbf{Migration of product data from ProD to Oracle database}: Alongside
with rewriting STIFF, we naturally want to drop the text database (ProD)
and move all the data found therein into the Oracle database, together
with the customer information. Also connected to this step is the necessity
of creating a database schema for the products that is flexible enough
to allow for future product customization

\item \textbf{Use Oracle stored procedures to centralize the SQL code}: This step is
concerned with creating stored procedures in the Oracle database which
provide all the data access that is now randomly spread through the
sub component of the system. In this way, we would ensure that the
application will not be sensible to future changes in the data model,
as long as the stored procedures are updated accordingly

\end{enumerate}

Steps 1 through 3 are planned to be finished by the end of phase 1, steps 4
through 6 by the end of phase 2 and the other steps by the end of phase 3. We
also include an additional phase (phase 4) which is meant for testing the system
and deciding if the load is properly dealt with or hardware additions are
necessary. This division of steps into phases will also be discussed in more
detail in the development view.

\section{\textcolor{red}{Guide for stakeholders}}

The interesting part of this document for the users is the
scenarios view (ref).

The part of this document that is of interest of the acquirers is
the Physical view (ref).

The parts of interest for the developers and the project manager
of this document are the Logical view (ref), the Process view
(ref) and the Development view (ref).

The parts of interest for the maintainers of this document are the
Logical view (ref), the Process view (ref) and the Development
view (ref).

The parts of this document that are of interest of the system
administrator are the Process view (ref) and the Physical view
(ref).

The parts of interest for the management are ...
