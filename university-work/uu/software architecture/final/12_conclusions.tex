\chapter{Conclusion}

Using the directives stated by this architecture description, SureThing will
benefit from an IT system that is completely scalable (ready to cope with the
company's growth expected after the merger) and unbounded (very unlikely to be
threatened by the eventual discontinuation of proprietary technologies).
In four words: SureThing will be \textbf{ready for the future}.


\section{Why the .NET requirement has been discarded}

The .NET framework is a new technology released by Microsoft
in order to compete with Java, as-a-matter-of-factly coming
with its share of advantages and disadvantages.

On the one hand, programs in .NET are supposedly slightly faster
than the ones written in Java, but not significantly: both use
run-time emulation via a virtual machine, which makes them about
two to three times slower than native programs (programs that
have been translated to machine code).

On the other hand, this performance overhead is forgivable for
Java, because this separation between virtual machine code and
machine code makes it possible to run the very same Java program
on a vast variety of platforms whose manufacturers have already
agreed to comply with Java standards, whereas .NET's expansion to
other platforms than Microsoft Windows does not seem promising at
all.

Therefore, had we chosen to develop the SureThing architecture
using the .NET framework, no savings could have been made by
reusing part of the existing J2EE implementation of UCIS. Instead,
the project would have suffered incremental costs implied by the
licensing of the .NET technology. As for the future, it is impossible
to assert which of the two rival techologies will prevail, but one
thing is for sure: today, the Java technology is widely spread
whilst the .NET technology has yet to prove itself.

Finally, considering the fact that the future of the Microsoft Windows
operating system itself has never been less sure since it overcame Apple's
MacOS on the market in the late eighties / early nineties, with the
Apple's recent come-back stories on every business channel nowadays,
and worldwide organizations such as the UN praising the Linux operating
system to the people, and computer market leaders such as IBM, HP and SUN
promoting Linux-based IT solutions, and the increasing security threat to
users of the Windows operating system, we believe that chosing the Java
technology will bring extra insurance to the project in regard to the
uncertain future of computer operating systems and architectures, since
it is possible to easily move Java applications from a Windows- and
PC-based platform to any other platform supporting Java, which is approximately
\textit{any existing platform}.

\section{Why the buy-before-build strategy has been ignored}

The project is mainly about merging two existing architectures and enhancing
the result. No software on the market would be exactly suitable for the task,
and buying a whole new software system would prove to be costlier than
building on top of the existing system as described by this proposal.