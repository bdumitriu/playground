\chapter{Viewpoints}

\section{Logical Viewpoint}
Because of the fact that all the stakeholders have different requirements and 
different concerns, not all those concerns can be covered in one view. The next view
is the Logical view, which is part of the "4+1 View Model of Software Architecture".
This chapter will first describe which stakeholders will be addressed by this view.
At second, the concerns that are covered with this view are mentioned, and after that
we will show diagrams with explanation on how the system should be implemented to 
cover all these concerns.

\subsubsection{Stakeholders}
The stakeholders that are addressed by the logical view are:
\begin{itemize}
\item Developers and the Project Manager
\item Maintainers
\item Management and the Acquirers (partially)
\end{itemize}

\subsubsection{Concerns}
The concerns that are covered by the logical view are:
\begin{itemize}
\item Will the system have a progressive evolution? %Progressive evolution of the system
\item How is the correctness and usability guaranteed during the transitions? %Correctness / usability during transitions
\item Will the system be interoperable? %Interoperability
\item Will the system be modular? %Modularity
\item Will the system be testable? %Testability
\item Will the system be changeable? %Changeability
\end{itemize}

\subsubsection{Modelling techniques}
The logical viewpoint can be modelled by using any of the following UML diagrams:
\begin{itemize}
\item Class- / Object diagrams
\item Sequence/collaboration diagrams
\item State-chart diagrams
\item Activity diagrams
\end{itemize}

\subsubsection{Library reference}
For more details about the logical viewpoint, you are referred to \cite{kru95}.

\section{Process Viewpoint}

The process viewpoint is meant to give an image of the running system, as described by
the architecture. It comes to complete the logical viewpoint by modelling the processes
which, through their interaction, define the behaviour of the system. The viewpoint explains
issues such as synchronization, concurrency, communication protocols, threads of control
and so on.

\subsubsection{Stakeholders}

The stakeholders the process viewpoint is relevant to are:

\begin{itemize}
\item Developers and project manager
\item Maintainers
\item System administrator (to some extent)
\item Management (to a very small extent)
\end{itemize}

\subsubsection{Concerns}

The concerns addressed by this viewpoint are:

\begin{itemize}
\item How does the system behave dynamically?
\item What are the processes that co-exist in the system and how do they interact?
\item What are the protocols used for communication?
\item How is the system scalable (in terms of process multiplication)?
\item What is influenced by changes in the behaviour of a process?
\item Which processes have to be able to communicate directly with others?
\end{itemize}

\subsubsection{Modelling techniques}

The process viewpoint can be modelled by using any of the following UML diagrams:

\begin{itemize}
\item Sequence/collaboration diagrams (usu. including active objects)
\item Component diagrams
\item State diagrams
\item Activity diagrams
\end{itemize}

Naturally, any other kind of diagram the architect might find appropriate is welcome
if accompanied by appropriate explanations.

\subsubsection{Library reference}

For a slightly more extended description of the what the process viewpoint should contain,
as well as for an example of a process view, you are referred to \cite{kru95}.

The description of UML diagrams can be found in a variety of sources (both printed and
on-line), one of which is \cite{fow00}.

\section{Development viewpoint}

\subsubsection{Stakeholders}
The stakeholders that are addressed by the development view are:
\begin{itemize}
\item Developers and the Project Manager
\item Maintainers
\end{itemize}

\subsubsection{Concerns}
The concerns that are covered by the development view are:
\begin{itemize}
\item Is the final system modular? %Modularity
\item Is the final system testable? %Testability
\item Is the final system easily changeable? %Changeability
\end{itemize}

\subsubsection{Modelling techniques}
The development viewpoint can be modelled by using any of the following diagrams:
\begin{itemize}
\item Package/component diagrams
\item Gantt diagrams
\end{itemize}

\section{Scenarios Viewpoint}

The scenarios viewpoint is redundant with most of the other viewpoints, its purpose being
that of providing a guide to reading the other architectural views. By describing a number
of relevant, representative use cases of the system and explaining how their goal can be
achieved by using the proposed architecture, the scenarios viewpoint can also be regarded
as a proof that the architecture is a good solution for the creation of the required system.

\subsubsection{Stakeholders}

The stakeholders the scenarios viewpoint is relevant to are:

\begin{itemize}
\item Acquirers
\item Management
\item Developers and project manager
\item Maintainers (to some extent)
\item System administrator (to some extent)
\end{itemize}

\subsubsection{Concerns}

The concerns addressed by this viewpoint are:

\begin{itemize}
\item To what extent does this architecture comply with the requirements?
\item How can the proposed architecture be used for achieving user goals?
\item How (in what order) should the other views be read through?
\item Where (in what view) can relevant information about a certain aspect of the system be found?
\item What actions of the system are implied by (some relevant) user goals?
\end{itemize}

\subsubsection{Modelling techniques}

The scenarios viewpoint is usually described using UML use case diagrams, accompanied by
a narrative description of each use case. The sentences in the description will be in the active
voice, present tense, describing an actor successfully achieving a goal.

\subsubsection{Library reference}

For a understanding the scenarios viewpoint in a larger context, you are referred to \cite{kru95}.

\section{Security Viewpoint}

The purpose of the security viewpoint is to provide developers and system deployers with vital
information about how the system has to be built and later on deployed, such that all the
security concerns expressed by the customer are handled. Security information can vary from
instructions about what protocols to be used to network configuration and yet to suggestions
about authentication and authorization mechanisms. The viewpoint should express enough details
about what has to be done, but should also be clear enough so that the acquirer stakeholder can
understand it and be convinced that the system security will indeed correspond to his/her level of
expectation.

\subsubsection{Stakeholders}

The stakeholders the security viewpoint is relevant to are:

\begin{itemize}
\item System administrator
\item Developers and project manager
\item Acquirer
\item Maintainers (to some extent)
\end{itemize}

\subsubsection{Concerns}

The concerns addressed by this viewpoint are:

\begin{itemize}
\item What protocols have to be used in order to secure communication?
\item What parts of the system have the highest security requests?
\item How should the network be configured in order to protect the system?
\item How is authentication/authorization to be done?
\item How secure is the system?
\item What parts of the system are less secure and why?
\end{itemize}

\subsubsection{Modelling techniques}

The security viewpoint will mostly be narrative, but appropriate diagrams can be
used where they could make understanding easier. Examples of such diagrams:

\begin{itemize}
\item (annotated) Deployment diagrams
\item (any kind of) Network diagrams
\item UML State diagrams (for explaining security protocols)
\item Database entity-relation diagrams (for, e.g., explaining access control mechanisms)
\item \ldots\ and many more
\end{itemize}