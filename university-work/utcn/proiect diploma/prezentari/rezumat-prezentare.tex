\documentclass[a4paper]{article}

\title{Editare colaborativ\u a asincron\u a de text}
\author{Bogdan Dumitriu}
\date{29 iunie 2004}

\begin{document}

\maketitle

Scopul proiectului de fa\c t\u a a fost crearea unui sistem complet
func\c tional care s\u a permit\u a utilizatorilor editarea
concurent\u a, \^ in izolare unul fa\c t\u a de altul, a aceluia\c si
document (de tip text) precum \c si sincronizarea la cerere a
versiunilor \^ in vederea ob\c tinerii unei versiuni comune unice a
documentului editat. Procesul de sincronizare trebuie s\u a se fac\u a
\^ intr-un mod sigur \c si consistent, astfel \^ inc\^ at nici o
modificare s\u a nu se piard\u a, s\u a nu apar\u a modific\u ari
inexistente (de exemplu prin dublarea unor opera\c tii) \c si astfel
\^ incat to\c ti utilizatorii s\u a ajung\u a s\u a aib\u a exact
aceea\c si versiune a documentului dup\u a ce sincronizarea s-a \^ incheiat.

Cea mai solicitant\u a parte a proiectului a constat \^ in dezvoltarea
unui algoritm robust care s\u a fie folosit \^ in faza de sincronizare.
Chiar dac\u a astfel de algoritmi exist\u a deja integra\c ti \^ in produse
comerciale (cum ar fi CVS, Subversion, BitKeeper sau Microsoft SourceSafe),
ei sunt baza\c ti pe compararea st\u arilor diverselor versiuni ale
documentelor gestionate, \^ in timp ce cercet\u arile din domeniu  \^ in
ultimul deceniu sunt \^ indreptate spre algoritmi baza\c ti pe compararea
opera\c tiilor efectuate asupra documentelor \c si nu a documentelor propriu-zise.
Pornind de la ni\c ste studii serioase a altor proiecte similare dezvoltate
\^ in ultimii ani (\c si prelu\^ and \c si dezvolt\^ and idei din acestea)
am reu\c sit s\u a cre\u am unui algoritm care, consider\u am noi, este
mai bun \^ in mai multe privin\c te dec\^ at to\c ti cei care exist\u a la ora
actual\u a. Dintre \^ imbun\u at\u a\c tirile aduse am dori s\u a men\c tion\u am:
reprezentarea arborescent\u a a textului (majoritatea cov\^ ar\c sitoare a
proiectelor similare sunt bazate pe o reprezentare liniar\u a), distribuirea
log-ului de opera\c tii efectuate asupra textului \^ in nodurile arborelui
(pu\c tinele proiecte care lucreaz\u a pe reprezentare arborescent\u a folosesc
log-uri asociate documentului \c si nu distribuite \^ in arbore), crearea
unui algoritm care s\u a poat\u a fi aplicat recursiv pe un astfel de arbore
\c si care s\u a aduc\u a astfel un spor semnificativ de performan\c t\u a
(prin reducerea masiv\u a a num\u arului de opera\c tii care trebuie comparate),
precum \c si dezvoltarea unei proceduri de reducere a log-ului existent \^ in momentul
sincroniz\u arii, sc\u az\^ and astfel la minim num\u arul de opera\c tii care
trebuie gestionate.

Un alt aspect inedit al proiectului se refer\u a la posibilitatea (implementat\u a
de noi) de a se realiza \c si o sincronizare direct\u a \^ intre utilizatori, f\u ar\u a
a mai fi nevoie de folosirea repository-ului ca intermediar. Astfel, orice utilizator
poate solicita permisiunea oric\u arui alt utilizator pentru a prelua direct de la acesta
modific\u arile curente aduse textului. Opera\c tiile preluate prin sincronizare direct\u a
trebuie tratate special, astfel \^ inc\^ at \^ in momentul \^ in care se trimit
(\^ impreun\u a cu cele locale) c\u atre repository s\u a se evite duplicarea lor
(av\^ and \^ in vedere c\u a vor fi trimise at\^ at de c\u atre utilizatorul care le-a
generat c\^ at \c si de cel care le-a preluat prin sincronizare direct\u a). Alte dou\u a
aspecte ale proiectului care \^ il diferen\c tiaz\u a de unele similare sunt posibilit\u a\c tile
multiple de rezolvare a conflictelor \c si configurabilitatea granularit\u a\c tii la
care sunt definite conflictele.

Desigur, a fost necesar\u a construirea unei \^ intregi infrastructuri \^ in jurul
algoritmilor dezvolta\c ti pentru a furniza utilizatorului final un produs complet. Aceasta
a inclus un sistem de comunicare \^ in re\c tea compus din partea de repository \c si
cea de client (care s\u a asigure comunicarea cu repository-ul). Dintre func\c tiile
repository-ului amintim stocarea tuturor versiunilor documentului, furnizarea
opera\c tiilor solicitate de clien\c ti pentru efectuarea sincroniz\u arii
(opera\c tii care reprezint\u a liniariz\u ari ale log-urilor trimise de al\c ti
clien\c ti) sau reg\u asirea numerelor de versiuni pe baza datei la care au fost
create. Clien\c tii, la r\^ andul lor, pe l\^ ang\u a asigurarea comunic\u arii
\^ in re\c tea, con\c tin at\^ at interfa\c ta grafic\u a prin care se re\-a\-li\-zea\-z\u a
de c\u atre utilizatori toate opera\c tiile permise de sistem (editare, trimitere
de opera\c tii c\u atre repository, sincronizare cu modific\u arile f\u acute de
al\c tii, gestionare listei de utilizatori cu care se face sincronizarea direct\u a,
etc.) c\^ at \c si o serie de module auxiliare precum parser de XML pentru citirea
fi\c sierelor de con\-fi\-gu\-ra\-re, parser de text (bazat pe o gramatic\u a definit\u a de
noi) pentru crearea \c si \^ intre\c tinerea structurii de arbore asociate documentului,
server pentru permiterea sincroniz\u arii directe \c si multe altele.

\^ Intregul proiect a fost implementat folosind exclusiv tehnologii Java. Ca detalii,
comunicarea \^ in re\c tea s-a realizat folosind RMI, interfa\c ta grafic\u a s-a
creat \^ in Swing, iar pentru parsing XML s-a folosit pachetul destinat acestui
scop furnizat de API-ul Java standard.

\end{document}
