\documentclass{beamer}

\mode<presentation>
{
  \usetheme{Warsaw}

  \setbeamercovered{transparent}
}

\usepackage[english]{babel}
\usepackage[latin1]{inputenc}
\usepackage{hyperref}
\usepackage{listings}
%\usepackage{pgf,pgfarrows}

\lstset{% general command to set parameter(s)
        %frame=single, % frame code fragments
        breaklines=true, % break long lines
        basicstyle=\tt\small % print whole listing small
        %keywordstyle=\color{black}\bfseries\underbar, % underlined bold black keywords
        %identifierstyle=, % nothing happens
        %commentstyle=\color{white}, % white comments
        %stringstyle=\ttfamily % typewriter type for strings
        %showstringspaces=false % no special string spaces
       }

\title{Thread-Specific Storage}

\author{Bogdan Dumitriu}

\institute
{
  Department of Computer Science\\
  University of Utrecht
}

\date{\today}

% If you wish to uncover everything in a step-wise fashion, uncomment
% the following command: 
%\beamerdefaultoverlayspecification{<+->}

\begin{document}

\begin{frame}
  \titlepage
\end{frame}

\begin{frame}
  \frametitle{Outline}
  \tableofcontents
\end{frame}

\section{Thread-Safe Storage in Java}

\subsection{Sample Application}

\begin{frame}
  \frametitle{Imagine that...}

Context:
\begin{itemize}
\item multi-threaded application using version control
\item originally, anonymous access was allowed
\item then, the admin changed his/her mind
\item code has to be changed to support authentication
\end{itemize}

\end{frame}

\begin{frame}[fragile]
  \frametitle{Class used to access the repository}

\begin{block}{class RepositoryOps}
\begin{verbatim}
public void checkOut() {
  System.out.println("checking out...");
}

public void update() {
  System.out.println("updating...");
}

public void commit() {
  System.out.println("committing...");
}

// ...
\end{verbatim}
\end{block}

\end{frame}

\begin{frame}[fragile]
  \frametitle{Application uses multiple MyThread threads}

\begin{block}{class MyThread extends Thread}
\begin{verbatim}
private RepositoryOps repoOps;

public MyThread(RepositoryOps repoOps) {
  this.repoOps = repoOps;
}

public void run() {
  String threadId = "Thread " + getId();
  System.out.println(threadId + " checking out...");
  repoOps.checkOut();
  System.out.println(threadId + " working...");
  System.out.println(threadId + " updating...");
  repoOps.update();
  System.out.println(threadId + " committing...");
  repoOps.commit();
}
\end{verbatim}
\end{block}

\end{frame}

\begin{frame}[fragile]
  \frametitle{The application}

\begin{block}{Main class}
\begin{verbatim}
public static void main(String args[]) {
  RepositoryOps repoOps = new RepositoryOps();

  new MyThread(repoOps).start();
  new MyThread(repoOps).start();
  new MyThread(repoOps).start();
}
\end{verbatim}
\end{block}

\end{frame}

\begin{frame}
  \frametitle{Adding authentication (1)}

\begin{itemize}
\item Now, let's add authentication
\pause
\item First step: create the User object
\end{itemize}

\end{frame}

\begin{frame}[fragile]
  \frametitle{Data about a user}

\begin{block}{User class}
\begin{verbatim}
static public final User ANONYMOUS_USER =
  new User("anonymous");

private String name;

public User(String name) {
  this.name = name;
}

public void setName(String name) {
  this.name = name;
}

public String getName() {
  return name;
}
\end{verbatim}
\end{block}

\end{frame}

\begin{frame}[fragile]
  \frametitle{User rights}

\begin{block}{User class}
\begin{verbatim}
public boolean canCommit() {
  if (name.equals(ANONYMOUS_USER_NAME)) {
    return false;
  }
  else {
    return true;
  }
}

public boolean canUpdate() {
  return canCommit();
}

public boolean canCheckOut() {
  return canCommit();
}
\end{verbatim}
\end{block}

\end{frame}

\begin{frame}[fragile]
  \frametitle{Basic solution}

\begin{itemize}
\item Immediate solution: add user field to MyThread
\end{itemize}

\pause

\begin{block}{class MyThread extends Thread}
\begin{verbatim}
private User user;

public void setUser(User user) {
  this.user = user;
}
\end{verbatim}
\end{block}

\pause

\begin{itemize}
\item But what if MyThread cannot be changed for whatever reason?
\end{itemize}

\end{frame}

\subsection{Discussing ThreadLocal}

\begin{frame}
  \frametitle{Adding authentication (2)}

\begin{itemize}
\item Second step: intorducing the \tt{ThreadLocal<T>}\normalfont{ class}
\end{itemize}

\begin{center}
\pgfdeclareimage[interpolate=true,height=3.5cm]{tlc}{img/threadLocalClass}
\pgfuseimage{tlc}
\end{center}

\end{frame}

\begin{frame}[fragile]
  \frametitle{Getting thread local values}

\begin{block}{ThreadLocal class}
\begin{verbatim}
public T get() {
  Thread t = Thread.currentThread();
  ThreadLocalMap map = t.threadLocals;
  if (map != null)
    return (T) map.get(this);

  T value = initialValue();
  t.threadLocals =
    new ThreadLocalMap(this, firstValue);
  return value;
}
\end{verbatim}
\end{block}

\end{frame}

\begin{frame}[fragile]
  \frametitle{Setting thread local values}

\begin{block}{ThreadLocal class}
\begin{verbatim}
public void set(T value) {
  Thread t = Thread.currentThread();
  ThreadLocalMap map = t.threadLocals;
  if (map != null)
    map.set(this, value);
  else
    t.threadLocals =
      new ThreadLocalMap(this, firstValue);
}
\end{verbatim}
\end{block}

\end{frame}

\subsection{Using ThreadLocal}

\begin{frame}
  \frametitle{Adding authentication (3)}

\begin{itemize}
\item Third step: change RepositoryOps to use authentication
\end{itemize}

\end{frame}

\begin{frame}[fragile]
  \frametitle{Adding thread local variable to RepositoryOps}

\begin{block}{class RepositoryOps}
\begin{verbatim}
private static ThreadLocal<User> user =
  new ThreadLocal<User>();
\end{verbatim}
\end{block}

\pause

\begin{block}{class RepositoryOps}
\begin{semiverbatim}
private static ThreadLocal<User> user =
  new ThreadLocal<User>() \{
    \alert{@Override}
    \alert{protected User initialValue() \{}
      \alert{return User.ANONYMOUS_USER;}
    \alert{\}}
  \};
\end{semiverbatim}
\end{block}

\end{frame}

\begin{frame}[fragile]
  \frametitle{Adding thread local variable to RepositoryOps}

\begin{block}{class RepositoryOps}
\begin{semiverbatim}
private static ThreadLocal<User> user =
  new ThreadLocal<User>() \{
    @Override
    protected User initialValue() \{
      \alert{int threadId = Thread.currentThread();}
      \alert{// lookup User based on threadId -> user}
      \alert{return user;}
    \}
  \};
\end{semiverbatim}
\end{block}

\end{frame}

\begin{frame}[fragile]
  \frametitle{Adding security to RepositoryOps}

\begin{block}{class RepositoryOps}
\begin{semiverbatim}
public void checkOut() \{
  \alert{if (user.get().canCheckOut()) \{}
    System.out.println("checking out...");
  \alert{\}}
\}
public void update() \{
  \alert{if (user.get().canUpdate()) \{}
    System.out.println("updating...");
  \alert{\}}
\}
public void commit() \{
  \alert{if (user.get().canCommit()) \{}
    System.out.println("committing...");
  \alert{\}}
\}
\end{semiverbatim}
\end{block}

\end{frame}

\begin{frame}
  \frametitle{Adding authentication (4)}

\begin{itemize}
\item However, it is unpleasant to use \tt{}user.get()\normalfont{} all the time
\pause
\item Fourth step: introduce a proxy...
\item ... and refactor RepositoryOps to use it
\end{itemize}

\end{frame}

\begin{frame}[fragile]
  \frametitle{Use a proxy to access the thread local object}

\begin{block}{class UserProxy}
\begin{verbatim}
private static ThreadLocal<User> user =
  new ThreadLocal<User>() {
    // ...
  };

public boolean canCommit() {
  return user.get().canCommit();
}

public boolean canUpdate() {
  return user.get().canUpdate();
}

public boolean canCheckOut() {
  return user.get().canCheckOut();
}
\end{verbatim}
\end{block}

\end{frame}

\begin{frame}[fragile]
  \frametitle{Refactoring RepositoryOps}

\begin{block}{class RepositoryOps}
\begin{semiverbatim}
\alert{private UserProxy userProxy = new UserProxy();}

public void checkOut() \{
  if (\alert{userProxy.canCheckOut()}) \{
    System.out.println("checking out...");
  \}
\}

public void update() \{
  if (\alert{userProxy.canUpdate()}) \{
    System.out.println("updating...");
  \}
\}

// ...
\end{semiverbatim}
\end{block}

\end{frame}

\begin{frame}
  \frametitle{Summing up}

So what have we learned about thread specific storage?
\begin{itemize}
\item similar to defining a field in each thread class, but actually you don't have to define it
\item single logical variable that has independent values in each separate thread
\item in Java, all you have to use is the ThreadLocal class
\end{itemize}

\end{frame}

\section{Discussion}

\subsection{Participants}

\begin{frame}
  \frametitle{Participants}

What about all the participants in the book?
\begin{itemize}
\item Thread Specific Object: \uncover<2->{\alert{User}}
\item Key: \uncover<3->{\alert{threadLocalHashCode in ThreadLocal}}
\item Key Factory: \uncover<4->{\alert{ThreadLocal.nextHashCode()}}
\item Thread Specific Object Set: \uncover<5->{\alert{threadLocals in Thread}}
\item Thread Specific Object Proxy: \uncover<6->{\alert{UserProxy}}
\item Application Thread: \uncover<7->{\alert{MyThread}}
\end{itemize}

\end{frame}

\subsection{When to use?}

\begin{frame}
  \frametitle{Issues with ThreadLocal}

Why use ThreadLocal when...
\begin{itemize}
\item we can just add a field to our thread class
\item we don't need to understand how ThreadLocal works
\item we don't need to write proxies/more code
\item we don't need to worry that the objects are garbage collected too late
\end{itemize}

\pause

\begin{block}{Tom Hawtin}
In general, if you have control of Thread construction, adding a field will be faster and possibly less buggy than using ThreadLocal.
\end{block}

\end{frame}

\begin{frame}
  \frametitle{Still...}

However:
\begin{itemize}
\item sometimes we can't add fields to our thread class
\pause
\item using ThreadLocal makes it easier to associate a thread with its per-thread data
\begin{itemize}
\item e.g. for passing per-thread context information
\end{itemize}

\end{itemize}

\end{frame}

\subsection{Some examples}

\begin{frame}
  \frametitle{Nice uses}

In \textit{Exploiting ThreadLocal to enhance scalability}, Brian Goetz explains
possible uses.\\
\vspace*{0.5cm}
Per-thread Singleton:
\begin{itemize}
\item process-wide Singletons vs. thread-wide ones
\item e.g. a JDBC Connection is not thread safe
\item a Connection pool is the customary solution
\item alternative: use ThreadLocal in the Singleton
\end{itemize}

\end{frame}

\begin{frame}[fragile]
  \frametitle{Per-thread Singleton}

\begin{block}{class ConnectionDispenser}
\begin{verbatim}
private ThreadLocal<Connection> conn =
  new ThreadLocal<Connection>() {
    @Override
    protected Connection initialValue() {
      return DriverManager.getConnection(<url>);
    }
  };

public static Connection getConnection() {
  return (Connection) conn.get();
}
\end{verbatim}
\end{block}

\end{frame}

\begin{frame}
  \frametitle{Nice uses}

Debugging multi-threaded applications:
\begin{itemize}
\item more generally, applications which collect per-thread info
\item debugging a multi-threaded app is cumbersome
\item instead of println's, use a per-thread logger
\item use DebugLogger.put() during run
\item retrieve saved info with DebugLogger.get() at the end
\end{itemize}

\end{frame}

\begin{frame}[fragile]
  \frametitle{Per-thread Singleton}

\begin{block}{class DebugLogger}
\begin{verbatim}
private ThreadLocal<List> list =
  new ThreadLocal<List>() {
    @Override
    public List initialValue() {
      return new ArrayList();
    }
  };

public static void put(String text) {
  list.get().add(text);
}

public String[] get() {
  return list.get().toArray(new String[0]);
}
\end{verbatim}
\end{block}

\end{frame}

\begin{frame}
  \frametitle{Nice uses}

Servlet-based applications:
\begin{itemize}
\item use ThreadLocal variables to store per-request info
\item only use when unit of work is one request!
\item otherwise, context will get mixed up
\end{itemize}

\end{frame}

\begin{frame}[fragile]
  \frametitle{Nice uses}

Based on another article by Brian Goetz, \textit{Can ThreadLocal solve the double-checked locking problem?}\\
\vspace*{0.5cm}
Remember the DCL (double-checked locking) problem?

\begin{block}{class DoubleCheck}
\begin{verbatim}
private static Resource resource = null;

public static Resource getResource() {
  if (resource == null) {
    synchronized {
      if (resource == null) 
        resource = new Resource();
    }
  }
  return resource;
}
\end{verbatim}
\end{block}

\end{frame}

\begin{frame}
  \frametitle{Double-Checked Locking}

\begin{itemize}
\item idea is to replace the check that the resource is null
\item instead, check if thread has executed the synchronized block
\item use ThreadLocal to do this
\item can you suggest how?
\end{itemize}

\end{frame}

\begin{frame}[fragile]
  \frametitle{Nice uses}

\begin{block}{class ThreadLocalDCL}
\begin{verbatim}
private static ThreadLocal<Boolean> initHolder
  = new ThreadLocal<Boolean>();
private static Resource resource = null;

public Resource getResource() {
  if (initHolder.get() == null) {
    synchronized {
      if (resource == null)
        resource = new Resource();
      initHolder.set(true);
    }
  }
  return resource;
}
\end{verbatim}
\end{block}

\end{frame}

\begin{frame}
  \frametitle{ThreadLocal performance is an issue!}

\begin{itemize}
\item DCL was designed to help with performance
\item does ThreadLocal beat synchronized lazy initialization?
\pause
\item \alert{in JDK 1.2 ThreadLocal uses synchronized WeakHashMap with Thread as key}
\pause
\item \alert{in JDK 1.3 Thread has threadLocals field, but Thread.currentThread() is bottleneck}
\pause
\item \textcolor{green}{ThreadLocal and Thread.currentThread() rewritten in JDK 1.4 $\rightarrow$ finally faster}
\end{itemize}

\end{frame}

\section{Summary}

\subsection{Benefits/Drawbacks}

\begin{frame}
  \frametitle{Advantages}

Thread specific storage is good when you want to:
\begin{itemize}
\item avoid synchronization $\rightarrow$ scalability
\item make thread unsafe objects easily usable by threads
\item add context to your threads but:
\begin{itemize}
\item you can't/won't change your thread classes
\item you don't want to pass objects around (per-thread Singleton)
\end{itemize}
\item document a variable as being thread safe
\end{itemize}

\end{frame}

\begin{frame}
  \frametitle{Liabilities}

You should think twice before using thread specific storage when:
\begin{itemize}
\item your application will run in a web application server
\item more generally, when using thread pools
\item memory is more important than synch time
\end{itemize}

\end{frame}

\subsection{Questions}

\begin{frame}
  \frametitle{Questions}

  \begin{center}
  ???
  \end{center}

\end{frame}

\begin{frame}[fragile]
  \frametitle{Question 1}

Assume SUN wants to add syntax support for thread specific storage.

\begin{block}{Proposal}
\begin{semiverbatim}
\alert{threadlocal} <type> <variable-name>
\end{semiverbatim}
\end{block}

Explain how this would be handled ``behind the scenes''. Discuss (some of)
the following issues:
\begin{itemize}
\item what would such a declaration be internally translated to?
\item what is the semantics of assigning to such a variable?
\item what is the semantics of reading such a variable?
\item garbage collection of such a variable?
\item whatever else you think is relevant...
\end{itemize}

\end{frame}

\begin{frame}
  \frametitle{Question 2}

If you don't like question 1, you can choose this one instead. You are not required
(or expected) to answer both questions.\\
\vspace*{0.5cm}
Identify and explain one or two significant differences between the pthread
model (discussed in the book) and the ThreadLocal model
(discussed during the lecture).\\
\vspace*{0.5cm}
For each identified difference, discuss why one version or the other is preferable.
Or if you think both have advantages/disadvantages, discuss those.

\end{frame}

\end{document}
