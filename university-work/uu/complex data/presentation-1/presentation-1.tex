\documentclass{beamer}

\mode<presentation>
{
  \usetheme{Warsaw}

  \setbeamercovered{transparent}
}

\usepackage[english]{babel}

\usepackage[latin1]{inputenc}

%\usepackage{times}
%\usepackage[T1]{fontenc}
% Or whatever. Note that the encoding and the font should match. If T1
% does not look nice, try deleting the line with the fontenc.

\title{Sequence Alignment}

\author{Bogdan Dumitriu}

\institute
{
  Department of Computer Science\\
  University of Utrecht
}

\date{\today}

% If you wish to uncover everything in a step-wise fashion, uncomment
% the following command: 
%\beamerdefaultoverlayspecification{<+->}

\begin{document}

\begin{frame}
  \titlepage
\end{frame}

\begin{frame}
  \frametitle{Outline}
  \tableofcontents%[pausesections]
\end{frame}

\section{The Scoring Model}

\subsection{Alignment Significance}

\begin{frame}
  \frametitle{Alignment Significance}
  \framesubtitle{Problem formulation}

  \begin{block}{Bram:}
    What kind of statistical methods can be used to evaluate the
    significance of an alignment score?
  \end{block}
\end{frame}

\begin{frame}
  \frametitle{Alignment Significance}
  \framesubtitle{Explanation}

  Obviously, a non-significant alignment:
  \\
  \vspace{0.5cm}
  {\tt THESEALGRITHMARETR--YINGTFINDTHEBESTWAYTMATCHPTWSEQENCES}
  {\tt TH\ S++\ +\ ++\ \ +++T\ \ \ Y\ \ \ \ FIND++\ \ \ \ \ \ \ YT\ \ +\ \ P\ +++\ \ \ \ \ \ }
  {\tt THISDESNTMEANTHATTHEYWILLFINDAN-------YTHIN-GPRFND------}
\end{frame}

\begin{frame}
  \frametitle{Alignment Significance}
  \framesubtitle{A solution}
  
  A common and simple test - a permutation test:
  \begin{enumerate}
    \item Randomly rearrange the order of one or both sequences.
    \item Align the permuted sequences.
    \item Record the score for this alignment.
    \item Repeat steps 1-3 a large number of times.
  \end{enumerate}

  \vspace{0.5cm}
  Then, compare the score with the obtained distribution.
\end{frame}

\subsection{Mutation Independence}

\begin{frame}
  \frametitle{Mutation Independence}
  \framesubtitle{Problem formulation}

  \begin{block}{Marjolijn:}
    Since RNA is a copy of a part of the DNA, why does the independence
    assumption regarding mutations hold for DNA, but not for RNA?
  \end{block}
\end{frame}

\begin{frame}
  \frametitle{Mutation Independence}
  \framesubtitle{Possible explanation}

  Messenger RNA (mRNA) undergoes a few processing steps:
  \begin{itemize}
    \item a modified guanine is added at the ``front'' of the message
    \item splicing: certain parts of non-coding sequences (introns) are removed
    \item a sequence of adenine nucleotides are added at the ``end'' of the message
  \end{itemize}

  Also, some errors can occur when copying DNA to RNA.
\end{frame}

\subsection{Relative Entropy}

\begin{frame}
  \frametitle{Relative Entropy}
  \framesubtitle{Problem formulation}

  \begin{block}{Jacob:}
    Why is \alert{$\displaystyle \sum_{a,b}q_a q_b \log \frac{q_a q_b}{p_{ab}}$}
    equal to the relative entropy \alert{$H(q^2 || p)$} and furthermore,
    what \emph{is} this entropy and what has it to do with this local
    alignment algorithm?
  \end{block}
\end{frame}

\begin{frame}
  \frametitle{Relative Entropy}
  \framesubtitle{What is information?}

  \begin{definition}
    \alert{Information} is a decrease in uncertainty.
  \end{definition}
  \begin{block}{Alternative definition}
    \alert{Information} can be seen as a degree of surprise.
  \end{block}

  \pause

  \begin{block}{Quantitative approach}
    \alert{Information}: $H(p)=\log_{2} \frac{1}{p}$ or $H(p)=-\log_{2}{p}$
  \end{block}
\end{frame}

\begin{frame}
  \frametitle{Relative Entropy}
  \framesubtitle{What is entropy?}

  Flat frequency distribution of symbols:
  \begin{itemize}
    \item each symbol has probability $\frac{1}{n}$
    \item each symbol has holds $\log_{2}{n}$ bits of information
  \end{itemize}

  Non-flat frequency distribution of symbols:
  \begin{itemize}
    \item each symbol has probability $p_i$
    \item \alert{on average}, each symbol holds $-\displaystyle \sum_i^n p_i \log_2 p_i$ bits of information
  \end{itemize}

  \pause

  \begin{definition}
    \alert{Entropy} represents the average information per symbol.
  \end{definition}
\end{frame}

\begin{frame}
  \frametitle{Relative Entropy}
  \framesubtitle{What is relative entropy?}

  Applying the entropy definition to scoring matrix:
  
  \begin{displaymath}
    \begin{split}
      H & = \sum_{i=1}^{20} \sum_{j=1}^i q_i q_j \log \frac{q_i q_j}{p_{ij}}\\
        & = -\sum_{i=1}^{20} \sum_{j=1}^i q_i q_j \log \frac{p_{ij}}{q_i q_j}\\
        & = -\sum_{i=1}^{20} \sum_{j=1}^i q_i q_j s(i,j)\\
	& = -\sum_{a,b} q_a q_b s(a,b)
    \end{split}
  \end{displaymath}

\end{frame}

\section{Global Alignment}

\subsection{Needleman-Wunsch Algorithm}

\begin{frame}
  \frametitle{Needleman-Wunsch Algorithm}
  \framesubtitle{Problem formulation}

  \begin{block}{Ingmar:}
    How do you determine which steps backwards to take in the
    matrix composed using dynamic programming by the Needleman-Wunsch
    algorithm in order to create the best alignment and why?
  \end{block}
\end{frame}

\begin{frame}
  \frametitle{Needleman-Wunsch Algorithm}
  \framesubtitle{In a nutshell}

  \begin{itemize}
    \item $F(i,j)$ - score of the best alignment between $x_1$$x_2$\ldots$x_i$
          and $y_1$$y_2$\ldots$y_j$
    \item $F(i,j)$ - built recursively based on $F(i-1,j-1)$, $F(i-1,j)$
          and $F(i,j-1)$
    \item \begin{displaymath}
            F(i,j)=\max \begin{cases}
                          F(i-1,j-1)+s(x_i,y_i),\\
			  F(i-1,j)-d,\\
			  F(i,j-1)-d.
	                \end{cases}
          \end{displaymath}
    \item for each $F(i,j)$ \alert{we record the choice we make}
  \end{itemize}
\end{frame}

\begin{frame}
  \frametitle{Needleman-Wunsch Algorithm}
  \framesubtitle{How traceback works}

  \begin{itemize}
    \item we start at $F(n,m)$
    \item $F(i,j)$ \alert{traces back} to one of $(i-1,j-1)$, $(i-1,j)$ or $(i,j-1)$
    \item depending on the choice, we add $x_i$ / $y_i$ / `\_'
    \item we stop at $F(0,0)$
  \end{itemize}
\end{frame}

\section{Alignment Variations}

\subsection{Multiple Sequence Alignments}

\begin{frame}
  \frametitle{Multiple Sequence Alignments}
  \framesubtitle{Problem formulation}

  \begin{block}{Adriano:}
    What are the implications of extending the \alert{a pairwise} alignment
    of sequences to \alert{multiple} sequence alignment?
  \end{block}
\end{frame}

\begin{frame}
  \frametitle{Multiple Sequence Alignments}
  \framesubtitle{Use Needleman-Wunch}

  \begin{block}{Two sequences}<1->
    Needleman-Wunch complexity for pairs: \alert{$O(n^2)$}.
  \end{block}
  \begin{block}{Multiple sequences}<1->
    Needleman-Wunch complexity for multiple: \alert{$O(n^m)$}, where $m$ is the
    number of sequences.
  \end{block}
  \begin{example}<2->
    With the same 10,000,000,000 operations:
    \begin{itemize}
      \item align 2 sequences of 100,000 nucleotides each.
      \item align 5 sequences of 100 nucleotides each.
    \end{itemize}
  \end{example}
  \begin{block}{Problem}<3->
    Needleman-Wunch doesn't scale well.
  \end{block}
\end{frame}

\begin{frame}
  \frametitle{Multiple Sequence Alignments}
  \framesubtitle{Use pairwise algorithm repeatedly}

  Another approach:
  \begin{itemize}
    \item align sequence 1 with sequence 2 $\Rightarrow$ trial consensus
    \item align consensus with sequence 3 $\Rightarrow$ new consensus
    \item carry on until global consensus is achieved
  \end{itemize}

  \pause

  \begin{block}{Problem}
    A different ordering of the sequences yields a different alignment.
  \end{block}
\end{frame}

\begin{frame}
  \frametitle{Multiple Sequence Alignments}
  \framesubtitle{The CLUSTAL algorithm}

  Steps of the CLUSTAL algorithm:
  \begin{enumerate}
    \item all pairs are aligned separately, result in distance matrix
    \item a guide tree is calculated from the distance matrix
    \item the sequences are progressively aligned based on guide tree
  \end{enumerate}
\end{frame}

\begin{frame}
  \frametitle{Multiple Sequence Alignments}
  \framesubtitle{Example of running CLUSTAL (1)}

  Say we have the set of proteins:
  \begin{enumerate}
    \item Hba\_Human: human $\alpha$-globin
    \item Hba\_Horse: horse $\alpha$-globin
    \item Hbb\_Human: human $\beta$-globin
    \item Hbb\_Horse: horse $\beta$-globin
    \item Myg\_Phyca: sperm whale myoglobin
    \item Glb5\_Petma: lamprey cyanohaemoglobin
    \item Lgb2\_Luplu: lupin leghaemoglobin
  \end{enumerate}
\end{frame}

\begin{frame}
  \frametitle{Multiple Sequence Alignments}
  \framesubtitle{Example of running CLUSTAL (2)}

  \alert{Step 1. The distance matrix / pairwise alignments}

  \begin{center}
    \begin{tabular}{lr|cccccc|}
      \cline{3-8}
      Hbb\_Human & 1 & -- &&&&& \\
      Hbb\_Horse & 2 & .17 & -- &&&& \\
      Hba\_Human & 3 & .59 & .60 & -- &&& \\
      Hba\_Horse & 4 & .59 & .59 & .13 & -- && \\
      Myg\_Phyca & 5 & .77 & .77 & .75 & .75 & -- & \\
      Glb5\_Petma & 6 & .81 & .82 & .73 & .74 & .80 & -- \\
      Lgb2\_Luplu & 7 & .87 & .86 & .86 & .88 & .93 & .90 \\
      \cline{3-8}
      &\multicolumn{1}{c}{} & 1 & 2 & 3 & 4 & 5 & \multicolumn{1}{c}{6} \\
    \end{tabular}
  \end{center}
\end{frame}

\begin{frame}
  \frametitle{Multiple Sequence Alignments}
  \framesubtitle{Example of running CLUSTAL (3)}

  \alert{Step 2. The guide tree}

  \begin{center}
    \setlength{\unitlength}{6cm}
    \begin{picture}(2,0.57)
      \put(0,0.2){\line(1,0){0.1}}
      \put(0.1,0.1){\line(0,1){0.2}}
      \put(0.1,0.1){\line(1,0){0.6}}
      \put(0.45,0.11){\small .442}
      \put(0.1,0.3){\line(1,0){0.1}}
      \put(0.09,0.31){\small .062}
      \put(0.2,0.2){\line(0,1){0.2}}
      \put(0.2,0.2){\line(1,0){0.5}}
      \put(0.45,0.21){\small .389}
      \put(0.2,0.4){\line(1,0){0.1}}
      \put(0.19,0.41){\small .015}
      \put(0.3,0.3){\line(0,1){0.2}}
      \put(0.3,0.3){\line(1,0){0.4}}
      \put(0.45,0.31){\small .398}
      \put(0.3,0.5){\line(1,0){0.1}}
      \put(0.29,0.51){\small .061}
      \put(0.4,0.42){\line(0,1){0.16}}
      \put(0.4,0.42){\line(1,0){0.2}}
      \put(0.45,0.43){\small .219}
      \put(0.4,0.58){\line(1,0){0.2}}
      \put(0.45,0.59){\small .226}
      \put(0.6,0.38){\line(0,1){0.08}}
      \put(0.6,0.38){\line(1,0){0.1}}
      \put(0.6,0.39){\small .065}
      \put(0.6,0.46){\line(1,0){0.1}}
      \put(0.6,0.47){\small .055}
      \put(0.6,0.54){\line(0,1){0.08}}
      \put(0.6,0.54){\line(1,0){0.1}}
      \put(0.6,0.55){\small .084}
      \put(0.6,0.62){\line(1,0){0.1}}
      \put(0.6,0.63){\small .081}

      \put(0.8,0.08){Lgb2\_Luplu:}
      \put(1.2,0.08){0.442}
      \put(0.8,0.18){Glb5\_Petma:}
      \put(1.2,0.18){0.398}
      \put(0.8,0.28){Myg\_Phyca:}
      \put(1.2,0.28){0.411}
      \put(0.8,0.36){Hba\_Horse:}
      \put(1.2,0.36){0.203}
      \put(0.8,0.44){Hba\_Human:}
      \put(1.2,0.44){0.194}
      \put(0.8,0.52){Hbb\_Horse:}
      \put(1.2,0.52){0.225}
      \put(0.8,0.60){Hbb\_Human:}
      \put(1.2,0.60){0.221}
    \end{picture}
  \end{center}
\end{frame}

\begin{frame}
  \frametitle{Multiple Sequence Alignments}
  \framesubtitle{Example of running CLUSTAL (4)}

  \alert{Step 3. Progressive alignment}

  \begin{itemize}
    \item order successive alignments based on guide tree
    \item each step: align a pair of sequences or alignments
    \item if aligning alignments, use weighed average
  \end{itemize}
\end{frame}

\begin{frame}
  \frametitle{Multiple Sequence Alignments}
  \framesubtitle{Improved version}

  CLUSTAL-W: CLUSTAL with some improvements

  \begin{itemize}
    \item gap penalties varies with position/residue
    \item scoring matrices varied with sequence pairs
    \item the weight assigned to sequences
  \end{itemize}

  \pause

  \begin{block}{More details:}
    {\tiny http://www.pubmedcentral.nih.gov/picrender.fcgi?artid=308517\&blobtype=pdf}
  \end{block}
\end{frame}

\begin{frame}
  \frametitle{Two more questions}

  \begin{block}{Marjolijn}
    Is it possible that you get the same alignment when using the
    Needleman-Wunsch algorithm (global) and the Smith-Waterman algorithm
    (local) to align two different sequences? If it is not, what are the
    differences in alignment and score?
  \end{block}
  \begin{block}{Lee}
    Exercise 2.9 gives you a little assignment to calculate the dynamic
    programming matrix and an optimal alignment for two given DNA
    sequences. It gives a linear gap penalty of d = 2 however. Shouldn't
    penalties be negative? So I assume this is just a type mistake from the
    authors?
  \end{block}
\end{frame}

\subsection{Overlap Matches}

\begin{frame}
  \frametitle{Overlap Matches}
  \framesubtitle{... in images (1)}

  \begin{center}
    \setlength{\unitlength}{3cm}
    \begin{picture}(2,1.4)
      \put(0.1,0.5){\line(1,0){1.8}}
      \put(1.9,0.5){\line(0,1){1}}
      \put(0.1,1.5){\line(1,0){1.8}}
      \put(0.1,0.5){\line(0,1){1}}
      \put(0.8,1.55){$x$}
      \put(0,1){$y$}

      \put(0.1,0.3){\line(1,0){1.8}}
      \put(0.7,0.2){\line(1,0){1.4}}
      \put(0.8,0.35){$x$}
      \put(1.4,0.10){$y$}

      \put(1.9,0.7){\line(-3,2){1.2}}
    \end{picture}
  \end{center}
\end{frame}

\begin{frame}
  \frametitle{Overlap Matches}
  \framesubtitle{... in images (2)}

  \begin{center}
    \setlength{\unitlength}{3cm}
    \begin{picture}(2,1.4)
      \put(0.1,0.5){\line(1,0){1.8}}
      \put(1.9,0.5){\line(0,1){1}}
      \put(0.1,1.5){\line(1,0){1.8}}
      \put(0.1,0.5){\line(0,1){1}}
      \put(0.8,1.55){$x$}
      \put(0,1){$y$}

      \put(0.1,0.3){\line(1,0){1.8}}
      \put(0.2,0.2){\line(1,0){1.5}}
      \put(0.8,0.35){$x$}
      \put(1.1,0.10){$y$}

      \put(1.7,0.5){\line(-3,2){1.5}}
    \end{picture}
  \end{center}
\end{frame}

\begin{frame}
  \frametitle{Overlap Matches}
  \framesubtitle{Idea}

  \begin{itemize}
    \item $F(0,0)$ turns into top or left border
    \item $F(n,m)$ turns into bottom or right border
    \item $F(i,0)$ becomes 0, $\forall i$
    \item $F(0,j)$ becomes 0, $\forall j$
    \item traceback starts at \emph{max} of bottom/right border
  \end{itemize}
\end{frame}

\end{document}
