\documentclass[a4paper,10pt]{article}

\usepackage{verbatim}
\usepackage[pdftex]{graphicx}

\title{Diner with Ambiants: Simulator}
\author{Bogdan \textsc{Dumitriu}, Jos\'e Pedro \textsc{Magalh\~aes}, Huanwen \textsc{Qu}, Jinfeng \textsc{Zhu}}

\begin{document}

\maketitle

\section{Code Structure}

Our simulator is designed to run inside a state monad\footnote[1]{Actually, a
{\tt StateT} monad with an {\tt IO} monad inside. This is necessary due to the
use of {\tt IOArray}s in as part of the state.}, whose state we use
as the current state of the world. Well, this is actually a rather simplistic
description of the state, since the components thereof are slightly more elaborate.
Perhaps the best way of describing them is by first supplying the actual definition
of the world state in our implementation:

\begin{verbatim}
        data WorldState = WorldState { board :: Board
                                     , ants  :: Ants
                                     , codes :: Codes
                                     , rng   :: [Int]
                                     , round :: Int
                                     }
\end{verbatim}

The first component, the {\tt board}, represents the array of cells that form the
world in which the ants perform their actions. The {\tt Board} type is defined
as follows:

\begin{verbatim}
        type Board = IOArray Pos Cell
\end{verbatim}

We have chosen to use an {\tt IOArray} to represent our board because we:

\begin{itemize}

\item needed a {\bf mutable} data structure. This is due to the fact that cells
are updated almost at each step, and having to copy an entire (potentially large)
data structure as this one for every update is certainly unacceptable.

\item needed a data structure which would allow us {\bf constant random access}
to any of its elements based on a position. The need for this arises from the
possibility of having ants randomly on the board at all times. Running a step for
an ant thus implies finding the cell that ant is located in. Naturally, we want to be
able to do this in $O(1)$ time.

\end{itemize}

The second component of the world state is the {\tt ants} component. It encodes
the positions of all the ants on the board in the following way:

\begin{verbatim}
        type Ants = IOArray AntId (Maybe Pos)
\end{verbatim}

Just as before, we needed a mutable structure, since the positions of ants frequently
change. In the same time, we also needed a structure which allows constant
random access since we had to ensure support for finding the position of an ant
based on the ant's id. The reason for having a {\tt Maybe Pos} instead of simply
{\tt Pos} as the values of the array is that ants can also be killed. If such an event
occurs, the position of the ant on the board is replaced with {\tt Nothing}, thus
encoding the death of the ant.

Using these to structures, the {\tt board} and the {\tt ants}, we achieved the goal
of providing $O(1)$ efficiency for all the functions required by the simulator. The
remainder elements in the world state serve different purposes, as follows:

\begin{itemize}

\item {\tt codes :: (Code, Maybe Code)} stores the sets of instructions for the red ants
and the black ants, respectively. Since black ant code is optional, we use {\tt Maybe Code}
as the second element of the tuple.

\item {\tt rng} represents a list from which random numbers are drawn when
necessary.

\item{\tt round} simply denotes the number of the current round of the simulation.

\end{itemize}

The {\tt Cell}s and the {\tt Ant}s are stored as records with the following types:

\begin{verbatim}
        data Cell = Cell { isRocky   :: Bool
                         , ant       :: Maybe Ant
                         , anthill   :: Maybe Color
                         , food      :: Int
                         , raMarkers :: Int8
                         , baMarkers :: Int8
                         }
\end{verbatim}

and

\begin{verbatim}
        data Ant = Ant { antid     :: AntId
                       , color     :: Color
                       , state     :: AntState
                       , resting   :: Int
                       , direction :: Dir
                       , hasFood   :: Bool
                       }
\end{verbatim}

We have documented our code quite extensively, so further explanations can be found as
comments. These can be also viewed in a nice(er) fashion by generating HTML output using the
Haddock\footnote[2]{http://www.haskell.org/haddock/.} documentation tool. You can run
it with the following command in the top directory of our project:

\begin{verbatim}
        haddock -o /path/to/output/dir -h ./*.hs
\end{verbatim}

\section{Profiling}

We have also run some profiling checks on our implementation. We
tested our implementation with a 100,000 rounds simulation, world
file {\tt sample0.world}, and red and black ant code {\tt sample.ant} (400
states), displaying no output (except for final food counts and the
winner) and having compiled with GHC's {\tt O1} optimization level and
{\tt auto-all} profiling information. The following results were obtained
on a laptop with a Pentium 4 processor running at 3.06GHz, with
512MB RAM memory, a 60GB disk and a Linux operating system:

\verbatiminput{time-profile.prof}

The heap profile (of the same run) is presented in figure \ref{heap}.

\begin{figure}
\label{heap}
%\begin{flushleft}
\includegraphics*[viewport=0 0 684 504,width=\textwidth]{heap-profile.pdf}
%\end{flushleft}
\caption{The heap profile}
\end{figure}

But since including profiling information adds a overhead to the
running time, we have also profiled the same kind of run, but
without having inserted any profiling information during
compilation. The unix tool {\tt time} then gave the following
information:
\begin{verbatim}
real    1m0.211s
user    0m58.074s
sys     0m0.422s
\end{verbatim}

\end{document}
