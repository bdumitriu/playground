\chapter{Requirements}

As a result of our discussions with the stakeholders we have decided
upon the requirements described in this chapter as the defining ones
for our architecture.

\section{Functional requirements}
The functional requirements display the needs and expectations of
the stakeholders in terms of software and hardware demands. We
would like to point out that most of the functional requirements of
the system are already met by the existing software and therefore
we only list here those requirements which are new. It should be
self-evident that the new system will include all the functionality
of the old system (possibly providing a changed implementation for
it). Here are then the extra requirements:
\begin{itemize}
\item Web access to customers: One of the major functional
requirements is the requirement of web access for customers. The
situation now only allows customers to call the Call-Center and get
their information there. The new system needs to supply a web
access for customers also. At this website there will be information
about the company but also the possibility to make a request for a
proposal calculation.
\item Discount for customers requesting
proposals via the web site: The company wants to have a kind of
discount for the customers that make a request for a proposal
calculation via the web site, to stimulate this kind of proposal
requests instead of proposal requests through the Call-Center.
\item Stability for web request for proposal: If a customer visits
the web site and makes a request for a proposal calculation through
the web site, it should be possible to freeze the prices and
discounts of that moment until the time that the customer actually
signs a contract. It is not appreciated when the customer makes a
request for a proposal calculation at one time, and thinks about
it for a couple of days, then returns to the web site and finds
out that in the mean time the same request will give a much higher
outcome. Therefore a period of price and discount freezing is
suggested (e.g. a week).
\item Use the GOC component: The GOC
component is not yet in use, because some miscalculations by one
of the preferred suppliers of the IT development organization. The
GOC is predicted to be available in March 2005. By then, all the
paper communication must start using the GOC component.
\item Access control to system functionality: There need to be different
access levels for the different groups of people accessing the
database and the other components of the system. The system needs
to distinguish especially the web site users from the companies
users, because the information in the database is confidential.
\end{itemize}

\section{Quality attributes}
The quality attributes of the system are a reflection of the
stakeholders concerns. Because there are so many concerns involved
in the system and some of them are contradictions, a selection
needed to be made. For each group only the major concerns (denoted
with +) where selected and of this pool of concerns we have chosen
the six most important quality requirements:
\begin{enumerate}
\item Interoperability: we view this as the most important quality
attribute, because the system needs above all to be able to
interact with the components the company already has and build
upon these.
\item Fault tolerance (24/7 availability): One of the vital concerns
of the acquirer is that the system must be available for the company's
customers 24 hours a day, 7 days a week. They see this as strategic
for their expected growth. The system to be developed, then, needs
to be robust enough in order to ensure that, even in the case of
isolated hardware failure, the functionality will still be there for
the customers. Only in the case of extended failure (e.g. a fire in
the building, with most servers burning down) is it acceptable to
have an nonoperational system.
\item Scalability: we expect the company to grow fast, as a result
of the merging and of the company's future Internet availability,
so the new system needs to be scalable to be able to deal with a
bigger amount of customers in the future. We expect a steady increase
in the number of customers and failure to cope with this number
of customers would be highly counter-productive for the company.
Therefore we have set this as an important quality aspect of our
architecture.
\item Security: because of the confidential kind of information the
company deals with every day, in combination with an access to
this information database via the web site, the system needs to be
designed with security in mind. We see this as an important attribute
which needs to be taken into account because its lack might have
very negative effects on the acquirer and its customers.
\item Changeability: this attribute (together with its related one,
modularity) is of interest to a great part of our stakeholders, therefore
we have decided to also take it into account as prioritary. Another
reason for selecting it is the desire to take an incremental approach
in the transformation of the system, which would obviously require
\textbf{changes at each step}. It is only natural, then, to see this as a quality
attribute to have in mind in order to ensure a smooth process.
\item Response time: The last significant attribute on our list is response
time due to a firm demand on the part of the acquirer of having a
system capable of computing proposals in real time. During discussions
it has become clear that this is a very requested attribute especially in
what web interaction is concerned so we were convinced to include it.
\end{enumerate}

Of course quality attributes can be associated with the other concerns
as well, but we believe that if we are able to make a system that satisfies
these six quality requirements, then every stakeholder will be satisfied and
will be able to use the system in a right and fulfilling way.

\section{Constraints}
The constraints that guide our design process are as follows:
\begin{itemize}
\item Cost: It appears from the discussions that both acquirers and management
wish to keep costs quite low for this project, so we will define it as a constraint
for us and, as a consequence, strive to come up with solutions which are
as less costly as possible and, most importantly, make sure that the development
of the system does not exceed these costs. This embraces the cost in terms of
money, time and people needed to accomplish the system.
\item Correctness and usability during transition: The whole transition of the
old system to the new system takes place in several phases. During each of
this phases the components that are already finished should be usable and
should be working correctly. We list this as a constraint for us because each
phase in particular has to be able to be materialized into a runnable entity.
In fact, this constraint is largely related to the fact that SureThing wishes to have
an incremental upgrading of their system.
\item Oracle \& J2EE: In terms of technologies that have to be used, we are constrained
to using the Oracle database platform as it is already purchased by the company
and it is neither likely nor necessary that it will be changed. On the other hand,
we will impose the use of the J2EE platform as  a constraint ourselves, largely
due to the fact that it is already currently in use by the UCIS system, but also due
to reasons which we explain later in this document.
\end{itemize}
