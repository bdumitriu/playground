\documentclass[a4paper,10pt]{article}

%\usepackage{graphicx}

\title{Implementation Description}
\author{Laurence \textsc{Cabenda}\\Bogdan \textsc{Dumitriu}\\Richard \textsc{Nieuwenhuis}\\Huanwen \textsc{Qu}}

\begin{document}

\maketitle

\section{Changes to the design}

In terms of design, we have not made significant changes, but we have slightly adjusted
things here and there. The most important change (made at your suggestion) has been the
separation of the GroupwareManagement interface in three distinct interfaces:

\begin{itemize}

\item \textbf{the UserManagement interface} - defines the methods to use for user creation
and deletion, for logging in and logging out and for obtaining lists of the users of the system.

\item \textbf{the CalendarManagement interface} - defines the methods to use for calendar
management activities which cannot be part of the Calendar interface itself, since they
don't pertain to a specific user's calendar. These methods are related to group appointments.

\item \textbf{the MailManagement interface} - defines the methods to use for mail related
activities. Again, it only contains those methods which do not pertain to a specific mail
account or to a specific mail box.

\end{itemize}

We have also kept the GroupwareManagement interface itself, but only as a aggregation
interface (i.e., it does not define any methods itself, but it aggregates the methods from
the three mentioned interfaces by inheritance).

Most of the methods defined in these three interfaces (and, by aggregation, in the
GroupwareManagement interface as well) should be seen as points of access to the
system. In other words, most of these methods return some remote object which can
then be used for particular functionalities. Because of that, their execution is only allowed
if the user can provide a valid authentication token (see the Security section of our
previous document for a discussion about this authentication token).

Apart from this structural change, the rest of the changes to the design were restricted
to changing some of the method names, modifying their lists of parameters and adding
a few setter and getter methods where we had omitted them. Another remark is that
the implementation of the GroupwareManagement interface is indeed defined as a
Singleton.

\section{Implementation Details}

In this section we give a quite detailed overview of our implementation, which should
be helpful as a guide to browsing our code. Most of this project is written in Java, the
only exception being the interoperability client, which was written in C. The organization
of the project is as follows:\\
\\
\indent/src - contains the source code for the entire project\\
\indent/config - contains a configuration file for the project\\
\indent/lib - contains a JDBC driver for version 8 of the PostgreSQL database server\\
\indent/sql - contains the script to be used for creating the needed database and tables\\
\indent/classes - contains the compiled Java classes\\
\\
\indent As we mentioned in our previous document, our design is based on a 3-tier architecture,
consisting of a database tier, a business logic tier and a client tier. The database tier consists
of the PostgreSQL database server, which is a 3\textsuperscript{rd} party product and
shall not be described here. That leaves us with the business logic tier (which we, from now
on, will call the server) and the client tier. We shall proceed to describing each of these two.

\subsubsection*{The server code}

The server code can be found in the core Java package. It contains three main groups of classes:
the so-called finder hierarchy, the remote interfaces and their implementations and the
(data) holder objects, which are passed back and forth between client and server.

The first group of objects (the finder hierarchy) is used for mapping database rows to
domain objects and making sure that there will never be more than one object in memory
representing the same database row (in order to prevent inconsistencies in the data).
The base class of this hierarchy is called \textit{AbstractFinder}, and it provides some generic
methods which leave only a few wholes to be filled in by the subclasses in order to
specialize behaviour. It is also this class that ensures the uniqueness of mappings from
database rows to domain objects. For this, the \textit{Identity Map} pattern is used (see \cite{fow03},
pp. 195-199). In a nutshell, this pattern implies the use of a Map (e.g., a HashMap) for
storing all objects loaded in memory, indexed by some identifier (usually, the primary key
of the database table). In this way, when an object is requested (based on its id) it can be
checked if the object is already loaded and, if so, it will not be loaded again and instead the
existing instance will be returned.

Subclasses of AbastracFinder are \textit{UserFinder} and \textit{AppointmentFinder}. These
classes define the specific SQL code for retrieving objects from the Users and Appointments
tables, respectively, and handle the loading of data into UserImpl and AppointmentImpl
objects. In addition, they define a few other useful methods for finding and loading of certain
database rows into objects. You will also notice a \textit{CalendarFinder} class in this hierarchy.
This is just pseudo-finder, since it doesn't get anything from the database, but it does ensure
the Identity Map functionality for CalendarImpl objects (i.e., we don't want more than one
CalendarImpl object per user to be active in memory at any time because this can also lead
to problems).

In order for this whole concept to work, we had to make sure that in all the server code objects
were always created using methods of the Finder classes. Also, all of the Finder classes are
defined as singletons, since otherwise their whole use is lost.

The second group of classes (and interfaces) you will find in the core package are those which
define the remote objects that can are used in the system. We have the \textit{UserManagement}
and \textit{CalendarManagement} interfaces aggregated by inheritance in the \textit{GroupwareManagement}
interface which define the methods for accessing the system and we have the \textit{User},
\textit{Calendar} and \textit{Group} interfaces which define methods for finding and setting
user details, adding/modifying/deleting single appointments and adding/modifying/deleting
group appointments, respectively. The last four mentioned interfaces are implemented in
classes with the same name followed by the \textit{Impl} suffix.

The \textit{GroupwareManagementImpl} class, apart from implementing the GroupwareManagement
interface, also contains the main method which is used to start the server. Here, the singleton
instance of GroupwareManagementImpl is retrieved and bound to an RMI registry which is
started (or found) also in the main method. If so specified by a command line argument, the
main method also loads the interoperability module, but this will be discussed separately later
on. Relevant about this class is that:

\begin{itemize}

\item it keeps a authenticatedUsers hash table, indexed by the authentication token, where all
the active users of the system are found. This table is also used for authentication when methods
of this class are called.

\item it unexports all exported remote objects associated with a user (namely, its User object and
its Calendar object) when the user logs off.

\item it uses functionality of the implementation (i.e., Impl-suffixed) classes which is generally
not available in the interfaces for remote method invocation for proving implementations of 
the methods defined in the GroupwareManagement interface.

\end{itemize}

The \textit{UserImpl} class is quite simple as it only offers a few setters and getters, most of
which can also be called remotely. Apart from these, the class contains methods which handle
the typical database operations (insertion, deletion and modification) for users as well as two
methods for changing the password, which also work directly on the database.

The \textit{CalendarImpl} class has methods for adding, changing and deleting appointments,
all of which are implemented by writing directly to the database (i.e., there is no permanent
in-memory list of appointments). The CalendarImpl class does the database work for administering
user - appointment id mappings, while for the actual management of appointment data (such
as location, duration and period), the class' methods delegate to their counterparts in the
\textit{Appointment} class. It should be noted that the Appointment class is actually part of
the 3\textsuperscript{rd} group of classes (i.e., the data holder classes), but it does have some
extra code added to it for handling database operations.

Besides the handling of appointments, the CalendarImpl class also takes care of notifying the
clients whenever appointments are changed. This is done in a multithreaded manner for
efficiency reasons. Last, but not least, the CalendarImpl class allows for retrieval of lists
of appointments.

\textit{GroupImpl} perhaps is the class that contains the most interesting business logic,
namely the one for retrieving time slots which can be used for setting a group appointment.
For this, it has some quite nice algorithms which are able to compute the periods of time
which are free for all the users within a group of users, given some constraints. Besides that,
it also contains methods for database operations which concern group appointments. In these
methods, transactions are used in order to ensure correct registration of group appointments.
This means that whenever the creation of a group appointment is requested, the selected
time slot is checked again, to make sure that it is indeed available for all the users in the group,
and then the appointment is actually registered for all the users. These operations are the
ones done inside a transaction in order to ensure correctness.

The final group of classes is the one which contains data holder classes. These are the
\textit{Appointment} class, the \textit{Period} class and the \textit{UserData} class. They
are usually just filled with data and send through the network.

For completeness, it should be mentioned that in the core package there are also the
\textit{ConnectionManager} class, which is a singleton class that provides connections
to the database server to whoever needs them and the \textit{ServerConfig} class,
which is a wrapper over the configuration file.

\subsubsection*{The client code}

We will naturally go into a lot less detail about the client code, since this mainly deals
with displaying a GUI for the user to be able to use the facilities of the server. All the
client-side classes are part of the client Java package.

The remote methods calls on the client side are done in the \textit{GroupwareClient} class,
which works as an interface between the GUI classes and the server. The GroupwareClient
handles all the details of the remote calls, and only gives the GUI the raw data it needs
(or some error messages if the case arises).

Another relevant class on the client side is the \textit{ClientObserverImpl} class, an
instance of which will be sent to the server so that the latter can notify the client by
means of RMI whenever changes to the logged in user's calendar take place. Such
an instance is registered by the GroupwareClient, right after the log in procedure
succeeds and unregistered right before the log out procedure.

Of course, most of the remaining client code consists of the GUI code, which is organized
in a classical Model-View-Controller style, with the view and the controller in the
(quite large) \textit{ClientController} class (and a few other helper classes) and the
model distributed between \textit{ClientModel} and \textit{ClientAppointmentModel}.

\subsubsection*{The interoperability code}

The code specific to the interoperability module is separated into the core.interop (the
server side) and client.interop (the client side) packages. The module provides just some
limited functionality, namely the retrieval of the appointments of a user. The whole
interoperability module is based on CORBA, and the IDL definition for it can be found
in the Calendar.idl file, which you can find in the src/ directory.

The class created by us (in contrast to the automatically generated classes) on the
server side is \textit{core.interop.CalendarImpl} which gets the list of requested
appointments from the main server module and transforms them in such a way that
they fit into the automatically generated framework.

The counterpart on the client side is the \textit{client.interop.CalendarClient} class
which provides a text based interface for the user to specify his login name and
password, then gets the appointments from the server by using CORBA for the
remote call and then displays the retrieved appointments.

A client with a similar functionality has been written in C as well, using the ORBit2 2.12.1
CORBA ORB implementation. This client can be found in the src/c-client directory. It
simply illustrates the interoperability concept even more by connecting to the Java
server from a C client by using CORBA, getting and displaying the list of appointments.

\section{Security}

In what security of our system is concerned, we have ensured a certain degree thereof
by the authentication mechanism together with the fact that clients cannot get access
to objects they are not allowed to, both of which are described in our previous document.
Naturally, without any kind of encryption employed by our system, the authentication
step can quite easily be breached by simply monitoring the network traffic and thus
retrieving the (clear text) passwords. Furthermore, the privacy of the users is also
in jeopardy without encryption, since their entire calendar can be viewed by the use
of network sniffers. Last, but not least, an attacker could spy the network and get
references to the objects to which normally access is not allowed and then use these
references to actually change a user's details or appointments. The natural solution
to all these problems is, of course, encryption of the communication between the client
and the server, either at the network level (e.g. by using IPSec) or at the application
level (e.g. by using RMI over SSL, if such an possibility exists).

Since this is a groupware system, we would normally expect it to run within an
organization, which means that additional security could be provided by disallowing access
to the machine running the server from any outside hosts. This would still leave us
with the problem of internal security, but it would nevertheless be a welcome addition.

Of course, a virtual attacker could also connect to the RMI server directly and, with
sufficient knowledge about the protocol, he might be able to get access to the objects
running in the server even without previously being in the possession of a reference
to them. Encryption wouldn't necessarily be able so solve this problem (unless we decide
to give each user a key and only allow access to the keys we have distributed to the users),
This leaves us with the solution of protecting access to \emph{all} the methods which
can be called remotely, either with a password or with the authentication token which
is already in use.

\section{Instructions for running the software}

Please see the README file that comes with our source code.

\begin{thebibliography}{99}

\bibitem[Fowler2003]{fow03} Fowler, M.: Patterns of Enterprise Application Architecture. Addison-Wesley, 2003.

\end{thebibliography}

\end{document}
