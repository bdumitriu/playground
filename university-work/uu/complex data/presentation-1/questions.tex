\documentclass{article}

\title{Session 1: Sequence Alignment}
\author{Bogdan Dumitriu}
\date{\today}

\begin{document}

\maketitle

Here are your questions:

\begin{enumerate}

\item {\bf (Bram)} What kind of statistical methods can be used
to evaluate the significance of an alignment score? How are these
methods used to distinguish true alignments form spurious alignments?

\item {\bf (Marjolijn)} The article says that ``The assumption of
independence appears to be a reasonable approximation for DNA and
protein sequences [...] However, it is seriously inaccurate for
structural RNAs, where base pairing introduces very important
long-range dependencies.'' (section 2.2 page 14)

How is it possible that independence holds for DNA and not for RNA
when you think of the fact that RNA is a copy of a small piece of DNA?

\item {\bf (Jacob)} Why is (equation 2.10) $\displaystyle \sum_{a,b}q_a q_b \log \frac{q_a q_b}{p_{ab}}$
equal to the relative entropy $H(q^2 || p)$ and furthermore,
what \emph{is} this entropy and what has it to do with this local
alignment algorithm?

\item {\bf (Ingmar)} I have a question about the dynamic programming
part. I understand how they build up a matrix as shown in picture 2.5,
however I do not completely understand how they derive an optimal
alignment from this matrix. It seems that when a letter matches a
letter they will go up diagonal in the matrix. When there is no
match they make a gap in the string for which the penalty is lowest.
However if you look at the bottom right of this matrix, then why do
they go from 1 to -5 instead of from 1 to 2 (it seems to me that 2
is better then -5 even though E matches with E).

So my question is: How do you determine which step backwards to take
in the matrix and why?

\item {\bf (Adriano)} In this paper we have seen different kinds of
alignment algorithms to find an optimal alignment for a pair of
sequences (Global Alignment, Local Alignment, Repeat Matches, Overlap
Matches and Hybrid Match Conditions), each one with complexity $O(n^2)$.
I think that it could be possible to find an optimal alignment
considering three (or more) sequences using the same intuition of
these algorithms, but this would lead to have a complexity $O(n^3)$
and in same cases (using dynamic programming with more complex models)
$O(n^9)$.

Is my hypothesis possible?

If it were possible, it seems to me that it has high complexity to be
used. $O(n^3)$ algorithms are only feasible for very short sequences.
Would you agree?

\item {\bf (Marjolijn)} Is it possible that you get the same
alignment when using the Needleman-Wunsch algorithm (global) and
the Smith-Waterman algorithm (local) to align two different
sequences? If it is not, what are the differences in alignment
and score?

\item {\bf (Lee)} Exercise 2.9 gives you a little assignment to
calculate the dynamic programming matrix and an optimal alignment
for two given DNA sequences. It gives a linear gap penalty of d = 2
however. Shouldn't penalties be negative? So I assume this is just a
type mistake from the authors?

\end{enumerate}

\end{document}