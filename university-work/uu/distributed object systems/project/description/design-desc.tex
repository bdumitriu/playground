\documentclass[a4paper,10pt]{article}

%\usepackage{graphicx}

\title{Design Description}
\author{Laurence \textsc{Cabenda}\\Bogdan \textsc{Dumitriu}\\Richard \textsc{Nieuwenhuis}\\Huanwen \textsc{Qu}}

\begin{document}

\maketitle

\section{Requirements}

The requirements for the system at hand can be found on the course's web site \cite{desc}.
Additional and detailed information about our understanding of these requirements can be found
in the description of the use cases (included separately with this documentation). The following
sections describe the design of the application which will be built to tackle these requirements,
so if you are not familiar with them, it might be a good idea to have a look at the mentioned
sources at this point.

We have chosen to design the calendar and the email modules of the groupware system. So far, we
have only covered the minimal requirements listed on the web site, but we shall, if time allows,
take extra functionality into consideration once we have the basics covered.

In this description, we first give a short overview of the general design of our system.
Then we go on to describing the remote objects in the system and the interfaces that define
their behaviour as remote objects. Further on, we discuss the security requirements relevant
to this system and explain how our design tackles these issues. We end by running through a
few scenarios which cover an extended part of our system, by this providing, in our opinion,
a guide to the design.

\section{General design}

Our system is organized according to a classical three tier architecture, with the following
components:

\begin{itemize}
\item \textbf{the client(s) tier}, which includes the presentation layer and some limited business
logic, mostly targeted towards reducing the amount of network traffic by doing consistency checks.
\item \textbf{the business logic tier}, which consists of a number of remote and local business
objects, running on a server and handling client requests. Access control is also done in this tier.
\item \textbf{the database tier}, which contains the persistent data needed in the application.
\end{itemize}
 
We have structured our design is such a way that most work will be done on the server so that,
on one hand, only small amounts of data (strictly what is necessary) will be transferred through
the network and, on the other hand, the consistency of updates can be ensured without the need
of complicated protocols/algorithms.

We have also included a notification process, by which clients will be made aware of changes
(legally) made by other users to their data as soon as these changes are operated (in particular,
this is the case with group appointments and receiving messages).

The technology that we shall use in order to implement our system will be Java's RMI. Therefore,
all the interfaces that we describe in the following section will eventually be coded as Java
interfaces extending java.rmi.Remote and all the remote objects in the system will be Java
objects. Both the client and the server will consequently be developed using Java technologies,
while the database management system which we shall use is PostgreSQL. As RMI is fully integrated
into the Java technology, no extra middleware will be necessary for running our application.
The only additional software that has to be used is a RMI Registry for binding remote objects
to symbolic names so that clients can have a point to start from.

In what the interoperability part of the project is concerned, we have not made a final decision
yet, but most likely we shall be using CORBA for this purpose.

\section{Remote objects}

In order to understand the system, it is best to start by listing the remote objects which form
its core and explain the purpose thereof. As you can also see in our class diagrams, there are
quite a number of remote objects with which the clients interact in order to perform all the
functions which the system offers.

Here is, then, the list of interfaces that define the remote objects used in the system. First of all,
there are two interfaces which are common to both the calendar and the email module. They
can be easily identified since they appear in both class diagrams\footnote{However, it should
be clear that they are defined just once. The fact that they appear in both class diagrams does
not mean that there will be two pairs of identical interfaces.}. Actually, these two interfaces
define objects which would be used more or less in all modules of a groupware system.

\subsubsection*{The \emph{GroupwareManagement} interface}

This interface represents the entry point into the system (from the point of view of the clients).
There will be only one remote object defined by this interface in the system at any given time.
Moreover, this object will be the only object bound to a symbolic name in the naming server. Any
other remote object running on the server can be created on demand and obtained by clients
only by calling methods on the GroupwareManagement instance.

Besides being an entry point into the system, this interface also provides a few basic functions.
These are as follows:

\begin{itemize}

\item \textbf{user management}: using this interface users can register a new account with the
system, get authenticated, log in, be accepted as valid users by the system during their session,
and finally log out. User accounts are deleted also by using this interface.

\item \textbf{group retrieval}: this functionality is particular to the Calendar module. When users
need to create (or later change/delete) a group appointment, there has to be a way in which the
group of users involved in the appointment is created (or retrieved). The GroupwareManagement
interface provides this way.

\item \textbf{sending of messages}: this functionality is particular to the Email module. Since a
user cannot (and should not) access another user's mail box directly, there has to be a way in
which user can send messages to other users (including the case when the recipient is actually
a mailing list). This is also provided by the GroupwareManagement interface.

\item \textbf{access to information}: a user (for various purposes) has to be able to get a list of the
other users in the system. The same holds for mailing lists. It is again the GroupwareManagement
interface that provides this kind of access.

\end{itemize}

The actual way in which the interface methods are used will become clear later on, when we shall
explain some use case scenarios. This holds for the following interfaces as well.

\subsubsection*{The \emph{User} interface}

The second ``general'' interface is the User interface. This interface, as its name suggests, is
designed to handle a single user. Thus, there will be a remote object, instance of this interface,
for every \emph{active} user in the system at a given time. Such an object is created by the
GroupwareManagement instance when authentication is successful\footnote{Since the security
aspect, including authentication and user management, will be discussed in the security aspect,
will be silently ignored here.} and is returned to the client (as a remote reference). The object
gets deleted when the user logs off or after a reasonable amount of inactivity of the user.

Basic functions provided by the user interface are what you would normally expect from such
an interface, i.e., changing of password, changing of various preferences (for example, the
granularity of the meetings from the calendar), accessing of own calendar, address book and
mail account as well as accessing of own authentication token (see section \ref{sec:security}).

Any other remote object pertaining to a user (such as his calendar or address book) can only be
obtained by means of calls to this remote object.\\
\\
We carry on now by describing the interfaces that pertain to the Calendar module.

\subsubsection*{The \emph{Calendar} interface}

This interface actually provides most of the access to functionality concerning calendar
management. Just as with the User interface, there will be a remote object, instance of
the Calendar interface, for each \emph{active} user of the system. Users can obtain their
respective Calendar remote objects by means of the User interface. These objects will disappear
together with the User objects they are associated with.

The functions available through this interface are \emph{single} appointment management
(\emph{group} appointment management is accessible through the Group interface), including
addition, modification and deletion of appointments, as well as various types of appoitment
retrieval (per month, per week, etc.).

\subsubsection*{The \emph{Group} interface}

The Group interface is more or less the equivalent of the Calendar interface \emph{in terms of
functionality}, with the distinction that this interface deals with group appointments, not single
appointments. However, it is not similar to the Calendar interface \emph{in terms of manipulation
and life span}. In other words, a user will not have an associated remote object instance of the
Group interface, as it is the case with the calendar, since this would make no sense. Instead,
a Group object is created ``on the fly'' by making a call to the GroupwareManagement instance
and is only used for an amout of time limited to the creation, deletion or modification of a group
appointment. This means that as soon as the creation, deletion or modification is performed, the
Group object can and should be, at least theoretically speaking, destroyed. This is achieved by
coding the client in such a way that it releases any reference it has to the object once the group
appointment is created, deleted or modified and, by this, indirectly causes the distributed garbage
collector to delete it (since nobody references it anymore).

In addition to allowing for the management of group appointments, the Group interface also provides
a method which comes up with some suggestions for possible times when an group appointment
can be organized (i.e., checks the calendars of all the users involved in the appointment and finds
some common free spots. This can (and probably should) be used by the client in order to decide
a time slot for the group appointment.\\
\\
Next we describe the interfaces that define the Email module as a remotly accessible system.

\subsubsection*{The \emph{AddressBook} interface}

The AddressBook is a simple interface, which allows the user to manage his\ldots\ address book.
Remote objects, instances of this interface, have a similar life span as those which are instances
of the Calendar interface, i.e., there is one such remote object in the system for every \emph{active}
user at any time and they get destroyed together with the User object they are associated with.
The client gets his AddressBook object by making a call to the User object.

The functions provided by the interface are no surprise: adding, deleting and retriving contacts.
Such contacts can be either users or mailing lists. A distinction between the two is made by
providing a contact type parameter when creating a contact. While users can be added to the
address book only if they already exist in the system, mailing lists can be created and added
to the address book at will. In order to create a mailing list, a user will have to use the
GroupwareManagement interface and then he will be able to add it to his address book.

\subsubsection*{The \emph{MailAccount} interface}

Just like the AddressBook (and Calendar), remote objects of this interface will fit into the same
life span and access pattern, so it will not be repeated here.

A MailAccount object not only allows the user to manage his mail boxes (add, delete or rename
them) and retrieve them, but also allows him to move messages from one mail box to another.
Methods are offered to retrieve some ``special'' mail boxes, such as the \emph{default} mail
box (where all incoming messages go to) or the \emph{sent mail} mail box.

\subsubsection*{The \emph{MailBox} interface}

Finally, the MailBox interface comes to provide an abstraction over an user's mail box. Since
a user can have more than one mail box, the number of mail box remote objects per \emph{active}
user depends on the actual numer of mail boxes the user has. Naturally, there will be one
mail box object per mail box. These objects are obtained by means of the MailAccount interface.
Since such objects can contain quite an amount of data, it might very well be the case
that in the implementation they will only be created on demand (i.e., not loaded together
with the MailAccount object). They will die at the latest when the user logs off (i.e., together
with the User remote object).

The functionality provided by a mail box consists of retrieving of messages, deleting them
and marking them as read. Messages can only be added to a mail box by using a method
which is \emph{not} part of the interface, because a user should not be able to add messages
directly to his own mail box. Messages will be put into a mail box either by means of sending
them through the GroupwareManagement instance or by moving them from another mail
box by means of the MailAccount instance.\\
\\
The only interface defining remote objects that is still to be defined is the callback interface.

\subsubsection*{The \emph{ClientObserver} interface}

This interface defines the remote objects that will be running on the \emph{clients} (all
the other remote objects mentioned until now run on the server). This interface is very
reduced in size, and it only provides methods for notification of changes. The idea is that
when a group appointment is created, deleted or changed or when a new message arrives,
the client should be notified of this (if he is active). Such a notification is done through
this interface\footnote{Indeed, we have another slip up in our class diagrams, since we
have only described the notification mechanism for appointments, and not for emails.
The idea, however, is exactly the same in both cases.}.

Normally, each client will instantiate a single remote object, instance of this interface,
when it begins execution and delete it when it ends execution. In order to make the object
available to the server, it will be registered with the Calendar and the default MailBox
instances belonging to this user. Once it is registered, every time something changes
in the Calendar or the default MailBox, the client will be notified by a remote method
invocation.\\
\\
We have another cateogory of objects in our system which will be exchanged between clients
and server, but they will not be used remotely. Instead, copies of them will be marshalled,
sent through the network and unmarshalled at the destination. These objects server the
purpose of data containers. Most of their methods are getters and setters and, as we
mentioned, these methods will be called locally by both sides. Notable examples of such
classes of objects are the Appointment and the Message classes. Except for these, there
is also a number of other ``smaller'' classes, which are too irrelevant for the big picture
in order to be mentioned here.

\section{Security}
\label{sec:security}

We continue our description of the design by addressing now another important issue,
namely the security of the system. In terms of security, we are mainly interested in two
aspects:

\begin{itemize}

\item only users with proper credentials should be granted access to the system.

\item authenticated users should not be allowed to access (i.e., see or change) any data
which does not belong to them. There are, however, two exceptions to this: group
appointments and sending of messages.

\end{itemize}

The first issue is addressed by including a so called authentication token for all the
methods of the groupware management system. This token is checked against the list
of valid tokens which exist at a certain moment in the system. If the token is invalid,
then no method can be called (except for the login and the create account methods).
In this way, access to the system is not allowed for unauthorized users.

A user can get authorized by logging in. At this point, the GroupwareManagement instance
will check his credentials and, if valid, will generate an unique token which will be sent
to the user and which will also be stored in the list of valid tokens. Once the user has
this token, it can send it every time it needs to call a method of the GroupwareManagement
instance. A token will be destroyed when the user logs off or after a certain predefined
time of inactivity. The user can access the system again only after he relogs in.

The second issue is addressed by simply not providing any way for a user to get a hold
of another user's remote objects. There is no way for a user to get a reference to a remote
object he is not allowed to. The only way remote objects providing access to user data
can be retrieved is by using a User remote object. But a User remote object can only be
retrieved using proper authentication. Otherwise, access is not possible.

Group appointments and sending of messages, the two exceptions we were mentioning
above, can be done by means of the GroupwareManagement instance. However, the processes
involved does not, at any time, reveal the remote objects belonging to other users. In this
way, it is ensured that access control is strictly enforced.

\section{Sample scenarios}

In order to completely grasp the remaining details of the system, we think it is best to run
through a couple of use cases and show how they can be achieved by using the proposed
design. In all of the following scenarios, we assume the user is already properly logged in.

\subsubsection*{The Add Contact use case}

We're going to start with one of the ``easier'' use cases, namely the addition of a contact to
the address book. For following through, it is helpful to use the matching sequence diagram,
attached to this documentation.

Once a user expresses, through the GUI, his desire to add a contact to his address book, this
is what happens:

\begin{enumerate}

\item the GUI will make a RMI to the GroupwareManagement instance and ask it for the
list of all users in the system (naturally, providing the authentication token in order to be
allowed to call the method\footnote{This will be done in all calls to the GroupwareManagement
instance which we are going to describe further on, but we will not mention it every time.
It should be considered implicit.}).

\item the GroupwareManagement instance will retrieve this list from the database (not
shown in the diagram) and return it to the client's GUI.

\item the same process goes on in order to retrieve the list of mailing lists that exist in
the system.

\item once this information is available, it is displayed to the user so that he can choose
the contacts (either users or mailing lists) he wants to add to his address book.

\item after the selection is done, the GUI will make a RMI to the User instance belonging
to this user in order to get the AddressBook instance belonging to this user.

\item a RMI is then made to the previously obtained AddressBook instance for each contact
which has to be added to the address book, requesting the AddressBook instance to add
the contact to the address book.

\item the AddressBook instance will insert the data into the database as well as store it in
its in-memory structures.

\item the GUI notifies the user if the action has been performed successfully.

\end{enumerate}

\subsubsection*{The Create Group Appointment use case}

The following use case is a bit more involved, which is why we have specifically chosen it as
an example. Running this will get us through a lot of the classes of the Calendar module so
that you can see how they all work together. Again, please use the associated sequence
diagram as a visual guide.

As soon as the user initiates action, this is what happens:

\begin{enumerate}

\item the GUI will make a RMI to the GroupwareManagement instance and ask it for the
list of all users in the system.

\item the GroupwareManagement instance will retrieve this list from the database (not
shown in the diagram) and return it to the client's GUI.

\item the GUI displays the list of users of the system to the user.

\item the user will select the other users he wants to have a group appointment with from
the list.

\item the GUI will make another RMI to the GroupwareManagement instance at this point
and ask it to create a Group remote object with the members specified by the user (including
himself).

\item the GroupwareManagement instance creates this group and returns a reference to it
to the GUI.

\item the user is then prompted to specify some details about the appointment (location, duration
and the like) as well as some constraints (for example, that he wants the appointment to be
scheduled no sooner than the following day, but no later than the following week).

\item with this information available, the GUI will now make a call to the Group remote
object it has obtained earlier and ask it to compute the possible time slots when the appointment
could be scheduled.

\item in order to do this, the group object will look into the database at the calendars of
all the users in the group and find all the available time slots (not shown in the diagram).
Once these time slots are computed, a list of them is returned to the GUI.

\item the GUI will display this list to the user and prompt him to choose the time slot that is
most convenient for him.

\item once the user does that, the GUI creates a new Appointment object (locally, as the appointment
is not used as a remote object) and fills it with the data specified by the user (including the selected
time slot).

\item next, the GUI does another RMI to the group object, this time asking it to create a new appointment.
It sends the previously created Appointment object as a parameter.

\item the Group instance will, in a transactional manner, check if the time slot is still valid for
all the users involved in the appointment and, if so, add this appointment to all of the users'
calendars. In order to do this, the Group instance will use the calendar object associated with
each of the users in the group. These objects either exist already in memory (if the users they
belong to are currently active) or have to be instantiated. When the add appointment method
on each calendar is called, if the calendar has a registered observer (this will be the case with
calendar objects belonging to users that are currently active), that observer will be notified so
that that user's GUI can be updated.

\item once all this is done, the GUI will notify the user if the action completed successfully.

\end{enumerate}

\subsubsection*{The Send Email use case}

Another rather involved scenario which we are going to use as an example is the Send Email
one. This will, just as the previous example, take us to a lot of classes. However, these will be
mostly other classes than in the previous example, part of the Email module. Consulting the
associated sequence diagram along with this explanation is again recommended.

After the user goes to the proper screen for sending emails, this is what happens:

\begin{enumerate}

\item first, the user will type in the message he wants to send.

\item once this is done, the GUI will make a RMI to the User remote object handling this user
and ask it for the address book of this user.

\item the address book remote reference obtained in the previous step is used next to make
a RMI to the AddressBook remote object and ask it for the list of contacts of this user.

\item the address book object will either serve this data from its in-memory structures, if
it is already available there or get it from the database and then return it.

\item the GUI will then allow the user to select the recepient of his message from the list
of contacts provided by the address book object (either a user or a mailing list).

\item next, a new Message object is created locally (since Message, the same as Appointment,
is not used as a remote object) and filled with all the relevant data.

\item the GUI makes a RMI call to the GroupwareManagement remote object and asks it
to deliver the message. Naturally, the message created in the previous step is given as a
parameter.

\item the GroupwareManagement instance will first check if the recepient of the message
is a mailing list and, if so, it will get the names of the users that belong to that mailing list
(the exact way in which this is done can be seen in the sequence diagram).

\item once the recepient(s) are available as real users of the system the GroupwareManagement
instance will iterate over the list of users (which could consist of only one element) and, for
each of them, get (if the user is currently active) or instantiate the User object representing
the user, use it to get the mail account of that user and use the mail account to get the default
mail box of the user (where all incoming messages have to be delivered). The message is then
added to the default mail box.

\item finally, the GUI notifies the user if this process has completed successfully.

\end{enumerate}

In our view, these three use case descriptions, together with the explanations from the previous
sections, should have given you enough of a grasp of the system so that you can, in order to complete
your understanding, on your own go through the rest of the (18!) sequence diagrams and examine
how each of the specific functionality of the system is achieved using our design.

\begin{thebibliography}{99}

\bibitem{desc} Project description, online.\\
\textit{http://catamaran.labs.cs.uu.nl/twiki/st/bin/view/Dos/CourseAssignments}, March 2005.

\end{thebibliography}

\end{document}
