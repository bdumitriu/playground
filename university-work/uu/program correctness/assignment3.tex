\documentclass[fleqn,oneside]{article}

\usepackage{a4}
\usepackage{latexsym}
\usepackage{pncproofmacros}

\title{Solution For Assignment 3}
\author{Bogdan Dumitriu}

\begin{document}

\maketitle

\noindent{}\bf The proof plan:\\
\\
\indent\EXPR{ALL(READ\ b:\ bool[],\ READ\ n:\ int,\ OUT\ r:\ bool)} \\

\EXPR{i,\ j:\ int;}\\

\COND{n\geq 0,\ see\ PROOF\ Init}\\

\EXPR{\{}

\EXPR{\indent{}i:=0;\ j:=n;\ r:=T;}\\

\indent\indent\COND{I,\ see\ PROOFs\ PTC1,\ PTC2,\ PEC,\ PIC.\ Termination\ metric:\ j-i}\\

\EXPR{\indent{}while\ i < j\wedge r\ do}

\EXPR{\indent\{}

\EXPR{\indent\indent{}j:=j-1;}

\EXPR{\indent\indent{}r:=b[i]\wedge b[j];}

\EXPR{\indent\indent{}i:=i+1}

\EXPR{\indent\}}

\EXPR{\}}\\

\COND{r=(\forall k:0\leq k<n:b[k])}\\

\EXPR{ASSUMING}\\

\EXPR{I = (r=(\forall k:(0\leq k<i)\vee(j\leq k<n):b[k]))\wedge(0\leq i\leq n)\wedge(0\leq j\leq n)}\\

\noindent{}\bf The proof:\\
\\
\indent\begin{PROOF}{Init}

\ASSUME{A1}{n\geq 0}

\DEF{D1}{Q=wp\ (i:=0;\ j:=n;\ r:=T)\ I}

\GOAL{G1}{Q}

\begin{BODY}

\item[1] \HINT{Rewrite D1 with definition of I and definition of wp}

\EXPR{Q=(T=(\forall k:(0\leq k<0)\vee(n\leq k<n):b[k]))\wedge(0\leq 0\leq n)\wedge(0\leq 0\leq n)}

\item[2] \HINT{Trivial, from 1, using A1 and 0 $\leq$ 0}

\EXPR{Q=(T=(\forall k:(0\leq k<0)\vee(n\leq k<n):b[k]))\wedge T\wedge T}

\item[3] \HINT{Basic equalities of boolean connectors on 2}

\EXPR{Q=(T=(\forall k:(0\leq k<0)\vee(n\leq k<n):b[k]))}

\item[4] \HINT{Empty domain conversion on 3}

\EXPR{Q=(T=(\forall k:F\vee F:b[k]))}

\item[5] \HINT{Basic equalities of boolean connectors on 4}

\EXPR{Q=(T=(\forall k:F:b[k]))}

\item[6] \HINT{Quantification over empty domain on 5}

\EXPR{Q=(T=T)}

\item[7] \HINT{Trivial, from 6}

\EXPR{Q=T}

\item[8] \HINT{True consequence on 7}

\EXPR{Q}

\end{BODY}

\end{PROOF}

\ \\

\begin{PROOF}{PTC1}

\ASSUME{A1}{I}

\ASSUME{A2}{i<j\wedge r}

\DEF{D1}{Q=wp\ (C:=j-i;\ j:=j-1;\ r:=b[i]\wedge b[j];\ i:=i+1)\ (j-i<C)}

\GOAL{G1}{Q}

\begin{BODY}

\item[1] \HINT{Rewrite D1 with definition of wp}

\EXPR{Q=j-1-i-1<j-i}

\item[2] \HINT{Trivial, from 1}

\EXPR{Q=-2<0}

\item[3] \HINT{Trivial, from 2}

\EXPR{Q=T}

\item[4] \HINT{True consequence on 3}

\EXPR{Q}

\end{BODY}

\end{PROOF}

\ \\

\begin{PROOF}{PTC2}

\ASSUME{A1}{I}

\ASSUME{A2}{i<j\wedge r}

\GOAL{G1}{j-i>0}

\begin{BODY}

\item[1] \HINT{$\wedge$-Elimination on A2}

\EXPR{i<j}

\item[2] \HINT{Trivial, from 1}

\EXPR{j-i>0}

\end{BODY}

\end{PROOF}

\ \\

\begin{PROOF}{PEC}

\ASSUME{A1}{I}

\ASSUME{A2}{j\leq i\vee \neg r}

\GOAL{G1}{r=(\forall k:0\leq k<n:b[k])}

\begin{BODY}

\item[1] \HINT{Rewrite A1 with definition of I}

\EXPR{(r=(\forall k:(0\leq k<i)\vee(j\leq k<n):b[k]))\wedge(0\leq i\leq n)\wedge(0\leq j\leq n)}

\item[2] \HINT{$\wedge$-Elimination on 1}

\EXPR{r=(\forall k:(0\leq k<i)\vee(j\leq k<n):b[k])}

\item[3] \HINT{$\wedge$-Elimination on 1}

\EXPR{0\leq i\leq n}

\item[4] \HINT{$\wedge$-Elimination on 1}

\EXPR{0\leq j\leq n}

\item[5] \HINT{See subproof sp1}

\EXPR{j\leq i\Rightarrow (r=(\forall k:0\leq k<n:b[k]))}

\begin{PROOF}{sp1}

\ASSUME{A1}{j\leq i}

\GOAL{G1}{r=(\forall k:0\leq k<n:b[k])}

\begin{BODY}

\item[1] \HINT{Domain merging on PEC.2, using 0 $\leq$ j (from PEC.4) and A1}

\EXPR{r=(\forall k:(0\leq k<j)\vee(j\leq k<i)\vee(j\leq k<n):b[k])}

\item[2] \HINT{From 1, using the fact that i $\leq$ n (from PEC.3) implies that $\tt [j, i)$ is included in $\tt [j, n)$}

\EXPR{r=(\forall k:(0\leq k<j)\vee(j\leq k<n):b[k])}

\item[3] \HINT{Domain merging on 2, using PEC.4}

\EXPR{r=(\forall k:0\leq k<n:b[k])}

\end{BODY}

\end{PROOF}

\item[6] \HINT{See subproof sp2}

\EXPR{\neg r\Rightarrow (r=(\forall k:0\leq k<n:b[k]))}

\begin{PROOF}{sp2}

\ASSUME{A1}{\neg r}

\GOAL{G1}{r=(\forall k:0\leq k<n:b[k])}

\begin{BODY}

\item[1] \HINT{Rewrite A1 with PEC.2}

\EXPR{\neg(\forall k:(0\leq k<i)\vee(j\leq k<n):b[k])}

\item[2] \HINT{Negate $\forall$ on 1}

\EXPR{(\exists k:(0\leq k<i)\vee(j\leq k<n):\neg b[k])}

\item[3] \HINT{$\exists$-Elimination on 2}

\tt [SOME k]

\EXPR{((0\leq k<i)\vee(j\leq k<n)) \wedge \neg b[k]}

\item[4] \HINT{$\wedge$-Elimination on 3}

\EXPR{(0\leq k<i)\vee(j\leq k<n)}

\item[5] \HINT{$\wedge$-Elimination on 3}

\EXPR{\neg b[k]}

\item[6] \HINT{Trivial, from PEC.3}

\EXPR{0\leq k<i\Rightarrow 0\leq k<n}

\item[7] \HINT{Trivial, from PEC.4}

\EXPR{j\leq k<n\Rightarrow 0\leq k<n}

\item[8] \HINT{Case split on 4, 6 and 7}

\EXPR{0\leq k<n}

\item[9] \HINT{$\exists$-Introduction on 8 and 5}

\EXPR{(\exists k:0\leq k<n:\neg b[k])}

\item[10] \HINT{Negate $\forall$ on 9}

\EXPR{\neg(\forall k:0\leq k<n:b[k])}

\item[11] \HINT{False consequence (see proof at the end of the document) on\nolinebreak\ 10}

\EXPR{(\forall k:0\leq k<n:b[k])=F}

\item[12] \HINT{False consequence (see proof at the end of the document) on\nolinebreak\ A1}

\EXPR{r=F}

\item[13] \HINT{Rewrite 12 with 11}

\EXPR{r=(\forall k:0\leq k<n:b[k])}
 
\end{BODY}

\end{PROOF}

\item[7] {Case split on A2, 5 and 6}

\EXPR{r=(\forall k:0\leq k<n:b[k])}

\end{BODY}

\end{PROOF}

\ \\

\begin{PROOF}{PIC}

\ASSUME{A1}{I}

\ASSUME{A2}{i<j\wedge r}

\DEF{D1}{Q=wp\ (j:=j-1;\ r:=b[i]\wedge b[j];\ i:=i+1)\ I}

\GOAL{G1}{Q}

\begin{BODY}

\item[1] \HINT{Rewrite A1 with definition of I}

\EXPR{(r=(\forall k:(0\leq k<i)\vee(j\leq k<n):b[k]))\wedge(0\leq i\leq n)\wedge(0\leq j\leq n)}

\item[2] \HINT{Rewrite D1 with definition of I and definition of wp}

\EXPR{Q=(b[i]\wedge b[j-1]=(\forall k:(0\leq k<i+1)\vee(j-1\leq k<n):b[k]))\\
\wedge(0\leq i+1\leq n)\wedge(0\leq j-1\leq n)}

\item[3] \HINT{$\wedge$-Elimination on 1}

\EXPR{r=(\forall k:(0\leq k<i)\vee(j\leq k<n):b[k])}

\item[4] \HINT{$\wedge$-Elimination on 1}

\EXPR{0\leq i\leq n}

\item[5] \HINT{$\wedge$-Elimination on 1}

\EXPR{0\leq j\leq n}

\item[6] \HINT{$\wedge$-Elimination on A2}

\EXPR{i<j}

\item[7] \HINT{$\wedge$-Elimination on A2}

\EXPR{r}

\item[8] \HINT{Trivial, from 6 and j $\leq$ n (from 5)}

\EXPR{i < n}

\item[9] \HINT{Trivial, from 8}

\EXPR{i+1\leq n}

\item[10] \HINT{Trivial, from 0 $\leq$ i (from 4)}

\EXPR{0\leq i+1}

\item[11] \HINT{Conjunction on 9 and 10}

\EXPR{0\leq i+1\leq n}

\item[12] \HINT{Trivial, from 6 and 0 $\leq$ i (from 4)}

\EXPR{0 < j}

\item[13] \HINT{Trivial, from 12}

\EXPR{0\leq j-1}

\item[14] \HINT{Trivial, from j $\leq$ n (from 5)}

\EXPR{j-1\leq n}

\item[15] \HINT{Conjunction on 13 and 14}

\EXPR{0\leq j-1\leq n}

\item[16] \HINT{See subproof eq}

\EXPR{b[i]\wedge b[j-1]=(\forall k:(0\leq k<i+1)\vee(j-1\leq k<n):b[k])}

\begin{EQPROOF}{eqq}

\item[] \EXPR{(\forall k:(0\leq k<i+1)\vee(j-1\leq k<n):b[k])}

\item[] $=$ \HINT{By domain merging, justified by 0 $\leq$ i (from PIC.4)}

\item[] \EXPR{(\forall k:(0\leq k<i)\vee(k=i)\vee(j-1\leq k<n):b[k])}

\item[] $=$ \HINT{By domain merging, justified by j-1 $\leq$ j and j $\leq$ n (from PIC.5)}

\item[] \EXPR{(\forall k:(0\leq k<i)\vee(k=i)\vee(j-1\leq k<j)\vee(j\leq k<n):b[k])}

\item[] $=$ \HINT{Trivial: $\tt j-1\leq k<j = (k=j-1)$}

\item[] \EXPR{(\forall k:(0\leq k<i)\vee(k=i)\vee(k=j-1)\vee(j\leq k<n):b[k])}

\item[] $=$ \HINT{By domain split}

\item[] \EXPR{(\forall k:0\leq k<i:b[k])\wedge(\forall k:k=i:b[k])\wedge(\forall k:k=j-1:b[k])\\
\wedge(\forall k:j\leq k<n:b[k])}

\item[] $=$ \HINT{By quantification over singleton domain}

\item[] \EXPR{(\forall k:0\leq k<i:b[k])\wedge b[i]\wedge b[j-1]\wedge(\forall k:j\leq k<n:b[k])}

\item[] $=$ \HINT{By domain split}

\item[] \EXPR{b[i]\wedge b[j-1]\wedge(\forall k:(0\leq k<i)\vee(j\leq k<n):b[k])}

\item[] $=$ \HINT{By rewriting using PIC.3}

\item[] \EXPR{b[i]\wedge b[j-1]\wedge r}

\item[] $=$ \HINT{By rewriting using true consequence on PIC.7}

\item[] \EXPR{b[i]\wedge b[j-1]\wedge T}

\item[] $=$ \HINT{By basic equalities of boolean connectors}

\item[] \EXPR{b[i]\wedge b[j-1]}

\end{EQPROOF}

\item[17] \HINT{Conjunction on 16, 11 and 15}

\EXPR{(b[i]\wedge b[j-1]=(\forall k:(0\leq k<i+1)\vee(j-1\leq k<n):b[k]))\\
\wedge(0\leq i+1\leq n)\wedge(0\leq j-1\leq n)}

\item[18] \HINT{Rewrite 17 with 2}

\EXPR{Q}

\end{BODY}

\end{PROOF}
\ \\
\ \\
\noindent\normalfont{}Finally, the proof of the "false consequence" rule used in subproof sp2 of proof PEC:\\
\\
\indent\begin{PROOF}{FalseConsequence}

\ASSUME{A1}{\neg P}

\GOAL{G1}{P=F}

\begin{BODY}

\item[1] \HINT{See subproof by contradiction spc}

\EXPR{P=F}

\begin{PROOF}{spc}

\ASSUME{A1}{P=T}

\GOAL{G1}{F}

\begin{BODY}

\item[1] \HINT{True consequence on A1}

\EXPR{P}

\item[2] \HINT{Contradiction on FalseConsequence.A1 and 1}

\EXPR{F}

\end{BODY}

\end{PROOF}

\end{BODY}

\end{PROOF}

\end{document}
