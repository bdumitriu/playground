\documentclass[fleqn,oneside]{article}

\usepackage{a4}
\usepackage{latexsym}
\usepackage{pncproofmacros}

\title{Solution For Assignment 2}
\author{Bogdan Dumitriu}

\begin{document}

\maketitle

\bf{Exercise 1} \\

\begin{PROOF}{asgn2\_1}

\DEF{D1}{sumsum1\ ss = SUM\ (map\ SUM\ ss)}

\DEF{D2}{sumsum2\ ss = foldr\ (s\ r \rightarrow SUM\ s + r)\ 0\ ss}

\GOAL{G1}{(\forall ss :: sumsum1\ ss = sumsum2\ ss)}

\begin{BODY}

\item[1] \HINT{See proof Pbase}

\EXPR{sumsum1\ [] = sumsum2\ []}

\begin{EQPROOF}{Pbase}

\item[] \EXPR{sumsum1\ []}

\item[] $=$ \HINT{By definition of sumsum1 (D1)}

\item[] \EXPR{SUM\ (map\ SUM\ [])}

\item[] $=$ \HINT{By definition of map}

\item[] \EXPR{SUM\ []}

\item[] $=$ \HINT{By definition of SUM}

\item[] \EXPR{0}

\item[] $=$ \HINT{By definition of foldr}

\item[] \EXPR{foldr\ (s\ r \rightarrow SUM\ s + r)\ 0\ []}

\item[] $=$ \HINT{By definition of sumsum2 (D2)}

\item[] \EXPR{sumsum2\ []}

\end{EQPROOF}

\item[2] \HINT{See proof Pinduct}

\EXPR{(\forall x,s:: (sumsum1\ s = sumsum2\ s) \Rightarrow (sumsum1\ (x:s) = sumsum2\ (x:s)))}

\begin{PROOF}{Pinduct}

\ASSUME{A1}{[ANY\ x,s]\ sumsum1\ s = sumsum2\ s}

\GOAL{G1}{sumsum1\ (x:s) = sumsum2\ (x:s)}

\begin{BODY}

\item[1] \HINT{See proof eqq}

\EXPR{sumsum1\ (x:s) = sumsum2\ (x:s)}

\begin{EQPROOF}{eqq}

\item[] \EXPR{sumsum2\ (x:s)}

\item[] $=$ \HINT{By definition of sumsum2 (D2)}

\item[] \EXPR{foldr\ (s\ r \rightarrow SUM\ s + r)\ 0\ (x:s)}

\item[] $=$ \HINT{By definition of foldr}

\item[] \EXPR{SUM\ x + (foldr\ (s\ r \rightarrow SUM\ s + r)\ 0\ s)}

\item[] $=$ \HINT{By definition of sumsum2 (D2)}

\item[] \EXPR{SUM\ x + (sumsum2\ s)}

\item[] $=$ \HINT{A1}

\item[] \EXPR{SUM\ x + (sumsum1\ s)}

\item[] $=$ \HINT{By definition of sumsum1 (D1)}

\item[] \EXPR{SUM\ x + (SUM\ (map\ SUM\ s))}

\item[] $=$ \HINT{By definition of SUM}

\item[] \EXPR{SUM\ ((SUM\ x):(map\ SUM\ s))}

\item[] $=$ \HINT{By definition of map}

\item[] \EXPR{SUM\ (map\ SUM\ (x:s))}

\item[] $=$ \HINT{By definition of sumsum1 (D1)}

\item[] \EXPR{sumsum1\ (x:s)}

\end{EQPROOF}

\end{BODY}

\end{PROOF}

\item[3] \HINT{List induction on 1 and 2}

\EXPR{(\forall s:: sumsum1\ s = sumsum2\ s)}

\item[4] \HINT{Rename bound variable of 3}

\EXPR{(\forall ss:: sumsum1\ ss = sumsum2\ ss)}

\end{BODY}

\end{PROOF}

\ \\

\textbf{Exercise 2}\\

\textnormal{\emph{Note 1}: The problem, as described in the text of the assignment, is somewhat
ambiguous in saying whether or not additional blue candies (i.e. blue candies which are not
initially in the box) can be used when taking out a white candy. The program specification,
however, implies that additional candies can (and will, if needed) be used when wanting to take
out a white candy without having at least two \_original\_ blue candies available out of the
box in order to put them back in. This is justified by the fact that $w>0$ is the only condition
for taking out a white candy (i.e. there is no additional condition specifying that at least two
of the \_original\_ blue candies have to be outside of the box in order for a white candy to be
taken out). The following proof is thus based on the program specification itself, which allows
the use of additional blue candies.}

\textnormal{\emph{Note 2}: I have added a new instruction to the program, just before $n:=0$,
in order to store the value of the expression $3*w+b$ computed using the initial values of $w$
and $b$ into the local variable WB. As this is just an assignment to a local variable, it
does not interfere with the program in any way. I have introduced this extra variable because
there was no other way in which I could specify the termination metric and then be still able
to prove everything that needs to be proven.}\\

\ \\

\noindent\bf The proof plan:\\

\EXPR{CANDY(OUT\ b,w,n:int)} \\

\EXPR{int\ WB;\ \tt \{See\ note\ 2\}}\\

\COND{b \geq 1 \wedge w \geq 1,\ see\ PROOF\ Init}\\

\EXPR{WB := 3*w+b;} \tt \{See note 2\}

\EXPR{n := 0;}

\begin{tabbing}
\indent $\tt \{*\ I,\ $\= $\tt see\ PROOFs\ PTC1a,\ PTC1b,\ PTC2,\ PEC,\ PICa,\ PICb.$\\
\> $\tt Termination\ metric:\ WB-n\ *\}$
\end{tabbing}

\EXPR{\texttt{while}}

\EXPR{\ \ \ \ b > 0\ do\ \{\ b := b-1;\ n := n+1\ \}}

\EXPR{\ \ \ \ w > 0\ do\ \{\ w := w-1;\ b := b+2;\ n := n+1\ \}} \\

\COND{(b=0)\wedge(w=0)\wedge n\geq 3} \\

\EXPR{ASSUMING} \\

\EXPR{I = (WB-n=3*w+b) \wedge (b \geq 0) \wedge (w \geq 0) \wedge (WB \geq 4)}\\

\noindent\bf The proof(s):\\

\begin{PROOF}{Init}

\ASSUME{A1}{b\geq 1\wedge w\geq 1}

\DEF{D1}{Q = wp\ (WB:=3*w+b;\ n:=0)\ I}

\GOAL{G1}{Q}

\begin{BODY}

\item[1] \HINT{Rewrite D1 with definition of I and definition of wp}

\EXPR{Q = (3*w+b-0=3*w+b) \wedge (b \geq 0) \wedge (w \geq 0) \wedge (3*w+b \geq 4)}

\item[2] \HINT{Trivial}

\EXPR{3*w+b-0=3*w+b}

\item[3] \HINT{$\wedge$-Elimination on A1}

\EXPR{b\geq 1}

\item[4] \HINT{$\wedge$-Elimination on A1}

\EXPR{w\geq 1}

\item[5] \HINT{Trivial, from 3}

\EXPR{b\geq 0}

\item[6] \HINT{Trivial, from 4}

\EXPR{w\geq 0}

\item[7] \HINT{Trivial, from 3 and 4}

\EXPR{3*w+b\geq 4}

\item[8] \HINT{Conjunction of 2, 5, 6 and 7}

\EXPR{(3*w+b-0=3*w+b) \wedge (b \geq 0) \wedge (w \geq 0) \wedge (3*w+b \geq 4)}

\item[9] \HINT{Rewrite 8 with 1}

\EXPR{Q}

\end{BODY}

\end{PROOF}

\ \\

\begin{PROOF}{PTC1a}

\ASSUME{A1}{I}

\ASSUME{A2}{b>0}

\DEF{D1}{Q=wp\ (C:=WB-n;\ b := b-1;\ n := n+1)\ (WB-n<C)}

\GOAL{G1}{Q}

\begin{BODY}

\item[1] \HINT{Rewrite D1 with definition of wp}

\EXPR{Q = WB-(n+1)<WB-n}

\item[2] \HINT{Trivial, from 1}

\EXPR{Q = -1<0}

\item[3] \HINT{Trivial}

\EXPR{-1<0}

\item[4] \HINT{Rewrite 3 with 2}

\EXPR{Q}

\end{BODY}

\end{PROOF}

\ \\

\begin{PROOF}{PTC1b}

\ASSUME{A1}{I}

\ASSUME{A2}{w>0}

\DEF{D1}{Q = wp\ (C:=WB-n;\ w := w-1;\ b := b+2;\ n := n+1)\ (WB-n<C)}

\GOAL{G1}{Q}

\begin{BODY}

\item[1] \HINT{Rewrite D1 with definition of wp}

\EXPR{Q = WB-(n+1)<WB-n}

\item[2] \HINT{Trivial, from 1}

\EXPR{Q = -1<0}

\item[3] \HINT{Trivial}

\EXPR{-1<0}

\item[4] \HINT{Rewrite 3 with 2}

\EXPR{Q}

\end{BODY}

\end{PROOF}

\ \\

\begin{PROOF}{PTC2}

\ASSUME{A1}{I}

\ASSUME{A2}{b>0 \vee w>0}

\GOAL{G1}{WB-n>0}

\begin{BODY}

\item[1] \HINT{Rewrite A1 with definition of I}

\EXPR{(WB-n=3*w+b) \wedge (b \geq 0) \wedge (w \geq 0) \wedge (WB \geq 4)}

\item[2] \HINT{$\wedge$-Elimination on 1}

\EXPR{WB-n=3*w+b}

\item[3] \HINT{$\wedge$-Elimination on 1}

\EXPR{b \geq 0}

\item[4] \HINT{$\wedge$-Elimination on 1}

\EXPR{w \geq 0}

\item[5] \HINT{Trivial, justified by 4}

\EXPR{b>0 \Rightarrow 3*w+b > 0}

\item[6] \HINT{Trivial, justified by 3}

\EXPR{w>0 \Rightarrow 3*w+b > 0}

\item[7] \HINT{Case split on A2, 5 and 6}

\EXPR{3*w+b>0}

\item[8] \HINT{Rewrite 7 with 2}

\EXPR{WB-n>0}

\end{BODY}

\end{PROOF}

\ \\

\begin{PROOF}{PEC}

\ASSUME{A1}{I}

\ASSUME{A2}{\neg(b>0)}

\ASSUME{A3}{\neg(w>0)}

\GOAL{G1}{(b=0)\wedge(w=0)\wedge n\geq 3}

\begin{BODY}

\item[1] \HINT{Rewrite A1 with definition of I}

\EXPR{(WB-n=3*w+b) \wedge (b \geq 0) \wedge (w \geq 0) \wedge (WB \geq 4)}

\item[2] \HINT{$\wedge$-Elimination on 1}

\EXPR{WB-n=3*w+b}

\item[3] \HINT{$\wedge$-Elimination on 1}

\EXPR{b \geq 0}

\item[4] \HINT{$\wedge$-Elimination on 1}

\EXPR{w \geq 0}

\item[5] \HINT{Trivial, from A2 and 3}

\EXPR{b=0}

\item[6] \HINT{Trivial, from A3 and 4}

\EXPR{w=0}

\item[7] \HINT{Rewrite 2 with 5 and 6}

\EXPR{WB-n=3*0+0}

\item[8] \HINT{Trivial, from 7}

\EXPR{n=WB}

\item[9] \HINT{$\wedge$-Elimination on 1}

\EXPR{WB\geq 4}

\item[10] \HINT{Rewrite 9 with 8}

\EXPR{n\geq 4}

\item[11] \HINT{Trivial, from 10}

\EXPR{n\geq 3}

\item[12] \HINT{Conjunction of 5, 6 and 11}

\EXPR{(b=0)\wedge(w=0)\wedge n\geq 3}

\end{BODY}

\end{PROOF}

\ \\

\begin{PROOF}{PICa}

\ASSUME{A1}{I}

\ASSUME{A2}{b>0}

\DEF{D1}{Q = wp\ (b:=b-1;\ n:=n+1)\ I}

\GOAL{G1}{Q}

\begin{BODY}

\item[1] \HINT{Rewrite D1 with definition of I}

\EXPR{Q = wp\ (b:=b-1;\ n:=n+1)\ ((WB-n=3*w+b) \wedge (b \geq 0) \wedge\nolinebreak(w \geq\nolinebreak 0) \wedge\nolinebreak (WB\nolinebreak\geq\nolinebreak 4))}

\item[2] \HINT{Rewrite 1 with definition of wp}

\EXPR{Q = (WB-n-1=3*w+b-1) \wedge (b-1 \geq 0) \wedge (w \geq 0) \wedge (WB \geq 4)}

\item[3] \HINT{Rewrite A1 with definition of I}

\EXPR{(WB-n=3*w+b) \wedge (b \geq 0) \wedge (w \geq 0) \wedge (WB \geq 4)}

\item[4] \HINT{$\wedge$-Elimination on 3}

\EXPR{WB-n=3*w+b}

\item[5] \HINT{Trivial, from 4}

\EXPR{WB-n-1=3*w+b-1}

\item[6] \HINT{$\wedge$-Elimination on 3}

\EXPR{w\geq 0}

\item[7] \HINT{Trivial, from A2}

\EXPR{b-1\geq 0}

\item[8] \HINT{$\wedge$-Elimination on 3}

\EXPR{WB \geq 4}

\item[9] \HINT{Conjunction on 5, 7, 6 and 8}

\EXPR{(WB-n-1=3*w+b-1) \wedge (b-1 \geq 0) \wedge (w \geq 0) \wedge (WB \geq 4)}

\item[10] \HINT{Rewrite 9 with 2}

\EXPR{Q}

\end{BODY}

\end{PROOF}

\ \\

\begin{PROOF}{PICb}

\ASSUME{A1}{I}

\ASSUME{A2}{w>0}

\DEF{D1}{Q = wp\ (w:=w-1;\ b:=b+2;\ n:=n+1)\ I}

\GOAL{G1}{Q}

\begin{BODY}

\item[1] \HINT{Rewrite D1 with definition of I}

\EXPR{Q = wp\ (w:=w-1;\ b:=b+2;\ n:=n+1)\ (( WB - n = 3 * w + b)\wedge\nolinebreak(b\geq\nolinebreak 0)\wedge\nolinebreak(w\geq\nolinebreak 0)\wedge\nolinebreak(WB\geq\nolinebreak 4)))}

\item[2] \HINT{Rewrite 1 with definition of wp}

\EXPR{Q = (WB-n-1=3*w-3+b+2) \wedge (b+2 \geq 0) \wedge (w-1 \geq 0) \wedge (WB \geq 4)}

\item[3] \HINT{Rewrite A1 with definition of I}

\EXPR{(WB-n=3*w+b) \wedge (b \geq 0) \wedge (w \geq 0) \wedge (WB \geq 4)}

\item[4] \HINT{$\wedge$-Elimination on 3}

\EXPR{WB-n=3*w+b}

\item[5] \HINT{Trivial, from 4}

\EXPR{WB-n-1=3*w-3+b+2}

\item[6] \HINT{$\wedge$-Elimination on 3}

\EXPR{b\geq 0}

\item[7] \HINT{Trivial, from 6}

\EXPR{b+2\geq 0}

\item[8] \HINT{Trivial, from A2}

\EXPR{w-1\geq 0}

\item[9] \HINT{$\wedge$-Elimination on 3}

\EXPR{WB \geq 4}

\item[10] \HINT{Conjunction on 5, 7, 8 and 9}

\EXPR{(WB-n-1=3*w-3+b+2) \wedge (b+2 \geq 0) \wedge (w-1 \geq 0) \wedge (WB \geq 4)}

\item[11] \HINT{Rewrite 10 with 2}

\EXPR{Q}

\end{BODY}

\end{PROOF}

\end{document}