\chapter{Introduction}

\section{Why do research in this area?}

In this day and age it is becoming increasingly clear that all individuals that keep on working
alone are destined to be easily overcome by groups of people working together. In a business environment
that relies strongly on concurrency (as the one nowadays certainly is), assigning serious tasks to the
individual (as opposed to groups) is no longer feasible. Every business that persists in doing so
is destined to fail sooner or later in such an environment.

As one might expect, this exact same principle applies roughly identically in most other areas of human
life as well, it is not limited to business environments alone. It is a well known fact that people work
much more efficiently in groups than alone and that, consequently, there is a strong tendency in our society
that involves the migration towards group work above all else. One can notice this tendency in virtually all
present day organizations, may them be profit or non-profit.

Another ingredient which is essential to the argument "I'm trying to make" refers to the omnisciency of technology
in day to day life. Every single area of work uses and benefits from the advances of technology in some way or
another. Undoubtedly, the most widely occurring form of technology is materialized in computers. They are everywhere
around us, all qualified personnel interacts with them daily and their penetration of the market is ever more deeper.
Under such circumstances, it is only natural that group work itself, along with all other types of human activities,
should eventually migrate towards this emerging digital world. Naturally, in order for such a migration to be
possible, adequate support has to be made available to those that need it. 

This slowly brings us to the point where this research comes into play, namely the point of thinking about the precise
technological tools that members of a group require in order for them to truly benefit from the advantages that
computers have to offer. These tools should ideally form the software foundation on which groups should be able to
base their work. The common name currently used for the study and research directed towards the development of
such tools is Computer-Supported Cooperative Work (CSCW). Essentially, this is a multi-disciplinary
field which brings social and computer scientists together, presenting them with the common task of discovering how
groups can best benefit from technological advances of the moment and developing the right software for this to become
a reality.

The particular subarea of Computer-Supported Cooperative Work that this paper tries to approach is called Collaborative
(or Cooperative) Editing Systems. A rough description of this field can be found in \cite{sun98a} and implies that
such systems are very useful groupware tools that allow physically dispersed people to edit a shared textual document,
draw a shared graph structure, record ideas during a brainstorm meeting or hold a design meeting. Of course, these are
just some basic examples; additional ones can easily be imagined. The motivation for developing such systems can be
found by looking at the means used today by most groups in order to coordinate their work. A typical scenario involves
a very tight group policy to which all members, with no exception, must conform. This policy would usually indicate precisely
what parts of the common objective (may it be a document, a piece of graphics, an architectural design, etc.) each member
of the group can modify, rendering all its other parts virtually inaccessible (or at least unchangeable) to this particular
member. For two (or more) persons to actually work on the same part of the ``document'' extremely careful coordination must
be employed, with geographical distance making this a difficult, if not impossible, task. Therefore, most ``real'' group
work today is drastically limited by the need of physical proximity of the group members. Under these circumstances, the
necessity for the kind of tools advertised by the field of Collaborative Editing Systems becomes obvious.

Open source software development is, for many, the most relevant example of collaborative work which involves members scattered
all over the world. These people may be freelance programmers, developers working for various companies, researchers in universities
or simply individuals who want to contribute with their own share to the advance of the various software tools. It is clear that
each of these people do their part of the work from their own physical location (home, university, office or wherever) and
somehow must be aided by technology in order for their joint efforts to actually materialize into something useful.
This particular example has been brought into discussion with the purpose of extending the argument exposed so far with
the concept of ``asynchrony'' that appears in the title of this paper alongside that of ``collaborative editing systems''.
In this line of thought, the remark that needs to be made is that open source software development, besides
being a relevant example of collaborative work, is also a typical case in which the members of the group don't
need to keep so tight a synchronization among each other as in the case of various other group activities. The reason for
this is that often during the process of software development several changes (some of which could take hours or even days
to be completed) have to be joined before the master ``document'' can be updated in order for the others to become aware
of the modifications. By contrast, there are many cases when a more frequent (up to the point of constant) synchronization
must be kept among the participants. Such cases do not constitute the subject of this paper, even though reference
to them will occasionally be made, when aspects thereof are relevant to the subject at hand.

This research work, as explained before, places itself within the area of Collaborative Editing Systems and comes as an
addition to the already existing achievements in the field, narrowing its focus, however, into two major ways. The first
confine is given by the fact that only asynchronous editing will be employed throughout this paper, while the second
is related to the issue of dealing with text editing only. This implies that the results achieved by the current work
will be useful to those interested in editing various types of text documents (such as source code, technical reports,
\LaTeX{} documents, books and virtually anything that can be expressed in characters alone) in an asynchronous manner
(i.e. working together on the same project(s), but only interested in reaching common versions only every once in a while).
Despite the fact that serious research is being conducted in this field, there is still a lot of room for
improvement in various areas, such as efficiency (both in running time and in storage capacity usage), closer
adherence to the user's intentions, reliability or mere usability.

In light of all that has been discussed up to this point, hopefully both the subject and the motivation
of this research have become sufficiently clear to the reader, thus enabling us to move on to a short
description of the work that has been done.

\section{Project outline}

The idea behind this project was to create a completely functional system which would allow users to
concurrently edit in isolation the same text document(s) and synchronize their work in order to obtain
a common view of the edited documents. The synchronization process is to be done in a reliable and
consistent way, such that none of the modifications made by any of the users are lost and that all
users end up with exactly the same version of the document after the synchronization has taken place.
Additionally, all the editing and network communication support has to be provided by the application.

The most challenging aspect of the project consisted in developing the robust algorithm to be
used during the synchronization phase. For this purpose, the first part of the work involved
a thorough survey of available algorithms both for real-time and asynchronous synchronization,
including those requiring the use of a dedicated central server as well as those without such
a requirement. This was done with the intention of identifying the most appropriate one to
be adapted for the project's needs. Developing this algorithm was a thorough process, during
which several very serious issues had to be overcome in order to obtain a correctly working
version. The most notable of these issues arose from our idea of going even further than other
researchers have gone and design this algorithm so that it could be recursively applied on a tree
representation of the document. Choosing a general structured model of the document allows
modelling of a larger class of documents such as XML documents compared to the linear model
approach. Moreover, the hierarchical structure offers a set of enhanced features such as
increased efficiency and improvements in the semantic.

An entire ``infrastructure'' had then to be built around this algorithm in order to present
the user with a usable and fully functional product. This included the development of an
appropriate repository where all versions of the document(s) could be stored and offered,
on request, to any interested client. Aside from this, the repository also had to provide
its clients with the possibility of committing their changes to the document in order for
others to have access to them. Another part of the ``infrastructure'' consisted in the
implementation of a graphical user interface to allow the clients to retrieve, modify
and send their modifications back to the repository (only to mention the basic functions).

Direct user synchronization is yet another feature that we wanted to provide our users
with. This had implications both at the algorithmic level (since we needed to further
ensure consistency and user intention preservation in the presence of this new type of
synchronization) and at the ``infrastructure'' level (since all clients needed to be
able to behave not only as editors of documents, but also as providers of documents).
Efforts were made in this direction because we realized that such an addition to the
functions of the application would be a great benefit to the users that prefer working
in smaller groups within the large groups.

The entire project was implemented using exclusively Java technologies, thus making it
possible for users with various platforms and operating systems to easily work together
with the exact same software.

\section{Overview}

This paper begins by looking at the efforts of other researchers which have worked in this field and
showing their achievements as well as their weak points, and suggesting the new ideas we have pursued in
order to fill some of the gaps others have left empty. We also try to explain why we think our
work is worth while by presenting some real life possible uses of asynchronous text editing.

The next chapter goes into more detail regarding various aspects, concepts and ideas peculiar
to asynchronous editing in general and asynchronous text editing in particular, concepts which
are necessary for the understanding of the rest of this paper.

Chapter four presents some (mostly theoretical) issues concerning operation-based editing. The most
important part of the chapter tries to explain what the inclusion and exclusion transformations
are as these are fundamental aspects which are important for the actual implementation. Other
concepts such as consistency models and operation context are introduced as well.

The fifth chapter represents the most important part of this paper since it is the chapter in
which our actual work is presented. We begin by describing some of the data strctures we use, the
employed operation model and the basic functions that work with the model. Then we go on to
describing the most important algorithms we have developed as well as some of the extra features
we have introduced. The chapter ends with an informal efficiency analysis of the most relevant
algorithms.

Our paper ends by suggesting a few ways in which we believe our work can be continued by others
and drawing some conclusions based on our results.