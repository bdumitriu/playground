\documentclass{beamer}

\mode<presentation>
{
  \usetheme{Warsaw}

  \setbeamercovered{transparent}
}

\usepackage[english]{babel}
\usepackage[latin1]{inputenc}
\usepackage{hyperref}
\usepackage{listings}
%\usepackage{pgf,pgfarrows}

\lstset{% general command to set parameter(s)
        %frame=single, % frame code fragments
        breaklines=true, % break long lines
        basicstyle=\tt\small % print whole listing small
        %keywordstyle=\color{black}\bfseries\underbar, % underlined bold black keywords
        %identifierstyle=, % nothing happens
        %commentstyle=\color{white}, % white comments
        %stringstyle=\ttfamily % typewriter type for strings
        %showstringspaces=false % no special string spaces
       }

\title{The Interceptor Pattern}

\author{Bogdan Dumitriu}

\institute
{
  Department of Computer Science\\
  University of Utrecht
}

\date{\today}

% If you wish to uncover everything in a step-wise fashion, uncomment
% the following command: 
%\beamerdefaultoverlayspecification{<+->}

\begin{document}

\begin{frame}
  \titlepage
\end{frame}

\begin{frame}
  \frametitle{Outline}
  \tableofcontents
\end{frame}

\section{Preliminaries}

\subsection{Problem Description}

\begin{frame}
  \frametitle{The Problem in a Nutshell}

  Given a system, how do we
  \begin{itemize}
  \item allow others to monitor what goes on inside it
  \pause
  \item and (optionally) change/extend some of its behaviour
  \pause
  \item without making them understand our code
  \pause
  \item and without making them change our code
  \pause
  \item and without affecting the system?
  \end{itemize}
  \pause
  Naturally, by using the \alert{Interceptor Pattern}!

\end{frame}

\subsection{Running Example}

\begin{frame}
  \frametitle{Instant Messenger}

  Design an IM system with:
  \begin{itemize}
  \item basic IM functionality
  \item logging of activity (history)
  \item encryption of messages
  \item auto response
  \item text replacement
  \item \ldots what not?
  \end{itemize}

  \begin{flushright}
  \pgfdeclareimage[interpolate=true,height=3.5cm]{im}{img/im}
  \pgfuseimage{im}
  \end{flushright}

\end{frame}

\begin{frame}
  \frametitle{Solution (BAD)}

  We could do it like this:

  \begin{center}
  \pgfdeclareimage[interpolate=true,height=5cm]{im-hell}{img/im-hell}
  \pgfuseimage{im-hell}
  \end{center}

\end{frame}

\begin{frame}
  \frametitle{Solution (GOOD)}

  Or like this:

  \begin{center}
  \pgfdeclareimage[interpolate=true,height=5cm]{im-heaven}{img/im-heaven}
  \pgfuseimage{im-heaven}
  \end{center}

\end{frame}

\begin{frame}
  \frametitle{Solution (in words)}

  \begin{itemize}
  \item build core with basic IM functionality
  \item allow services to register with the core
  \item make sure core triggers these services
  \item let others do the hard work
  \end{itemize}

\end{frame}

\subsection{Interceptor Pattern}

\begin{frame}
  \frametitle{Participants}

  \begin{itemize}
  \item Framework (or core system)
  \item Dispatcher(s)
  \item Context Object(s)
  \item Interceptor Interface(s)
  \item Concrete Interceptor(s)
  \item Application (using the Concrete Interceptor(s))
  \end{itemize}

\end{frame}

\begin{frame}
  \frametitle{Solution (class diagram)}

  \begin{center}
  \pgfdeclareimage[interpolate=true,height=6cm]{im-class-diag}{img/im-class-diag}
  \pgfuseimage{im-class-diag}
  \end{center}

\end{frame}

\begin{frame}
  \frametitle{Solution (class diagram)}

  \begin{center}
  \pgfdeclareimage[interpolate=true,height=6cm]{im-seq-diag}{img/im-seq-diag}
  \pgfuseimage{im-seq-diag}
  \end{center}

\end{frame}

\section{The Interceptor Pattern at Use}

\subsection{IM Implementation}

\begin{frame}
  \frametitle{Back to our example}

  How do we go about the Instant Messenger implementation?

  \begin{itemize}
  \item draw state diagram of system
  \item identify and group about interception points
  \item define dispatcher-interceptor-context object triplets
  \item make sure the core calls the dispatchers
  \item (leave to others) write interceptors to provide services
  \end{itemize}

\end{frame}

\begin{frame}
  \frametitle{State diagram}

  \begin{center}
  \pgfdeclareimage[interpolate=true,height=6cm]{im-state-diag}{img/im-state-diag}
  \pgfuseimage{im-state-diag}
  \end{center}

\end{frame}

\begin{frame}
  \frametitle{Identify interception points}

  \begin{center}
  \pgfdeclareimage[interpolate=true,height=6cm]{im-state-diag-mod}{img/im-state-diag-mod}
  \pgfuseimage{im-state-diag-mod}
  \end{center}

\end{frame}

\begin{frame}
  \frametitle{Group interception points}

  Loging in/Loging out Group:
  \begin{itemize}
  \item pre-login and post-login events
  \item pre-logout and post-logout events
  \end{itemize}
  All these are read-only.

\end{frame}

\begin{frame}
  \frametitle{Group interception points}

  Friend-Related Group:
  \begin{itemize}
  \item friend added event
  \item friend removed event
  \item friend logged in event
  \item friend logged out event
  \item friends list received event
  \end{itemize}
  All these are read-only.

\end{frame}

\begin{frame}
  \frametitle{Group interception points}

  Message Exchange-Related Group:
  \begin{itemize}
  \item pre message send event
  \item post message received event
  \end{itemize}
  These are both read/write.

\end{frame}

\begin{frame}
  \frametitle{Define triplets}

  For each group of interception points we define:
  \begin{itemize}
  \item a Dispatcher
  \item an Interceptor Interface
  \item a Context Object
  \end{itemize}

\end{frame}

\frame[all:1] {
\begin{block}{Friend-Related Group Interceptor Interface}
\begin{lstlisting}
public interface FriendInterceptor
{
  public void friendAdded(FriendContext ctx);

  public void friendRemoved(FriendContext ctx);

  public void wentOnline(FriendContext ctx);

  public void wentOffline(FriendContext ctx);

  public void receivedFriendsList(FriendListContext ctx);
}
\end{lstlisting}
\end{block}
}

\frame[all:1] {
\begin{block}{Friend-Related Group Context Object}
\begin{lstlisting}
public enum Outcome {NONE, SUCCESSFUL, FAILED}
public class FriendContext
{
  public Outcome getOutcome()
  { return outcome; }

  public String getUserName()
  { return userName; }

  public String getFriendName()
  { return friendName; }

  ...
}
\end{lstlisting}
\end{block}
}

\frame[all:1] {
\begin{block}{Friend-Related Group Dispatcher}
\begin{lstlisting}
public class FriendDispatcher implements FriendInterceptor
{
  public void friendAdded(FriendContext ctx)
  { invokeAll("friendAdded", ctx); }

  public void friendRemoved(FriendContext ctx)
  { invokeAll("friendRemoved", ctx); }

  public void wentOnline(FriendContext ctx)
  { invokeAll("wentOnline", ctx); }

  public void wentOffline(FriendContext ctx)
  { invokeAll("wentOffline", ctx); }

  ...
\end{lstlisting}
\end{block}
}

\frame[all:1] {
\begin{block}{Friend-Related Group Dispatcher}
\begin{lstlisting}
  private void invokeAll(String methodName, FriendContext ctx)
  {
    Method method = FriendInterceptor.class.
      getMethod(methodName,
        FriendContext.class);
    for (FriendInterceptor interceptor : interceptors.values())
    {
      method.invoke(interceptor, ctx);
    }
  }

  private Map<Integer, FriendInterceptor> interceptors;
}
\end{lstlisting}
\end{block}
}

\frame[all:1] {
\begin{block}{Message Exchange-Related Group Dispatcher}
\begin{lstlisting}
public class MessageDispatcher implements MessageInterceptor
{
  public void addMessageInterceptor(
    MessageInterceptor interceptor,
    int priority)
  { interceptors.put(priority, interceptor); }

  public void preMessageSend(MessageContext ctx)
  {
    ArrayList<MessageInterceptor> values =
      new ArrayList<MessageInterceptor>(
      interceptors.values());
    for (int i = values.size() - 1; i >= 0; i--)
    { values.get(i).preMessageSend(ctx); }
  }
\end{lstlisting}
\end{block}
}

\frame[all:1] {
\begin{block}{Message Exchange-Related Group Dispatcher}
\begin{lstlisting}
  public void postMessageReceive(
    MessageContext ctx)
  {
    for (MessageInterceptor interceptor : interceptors.values())
    { interceptor.postMessageReceived(ctx) }
  }

  private Map<Integer, MessageInterceptor> interceptors;
}
\end{lstlisting}
\end{block}
}

\subsection{Discussion}

\begin{frame}
  \frametitle{Design Decisions}

  A few variability points:
  \begin{itemize}
  \item Context objects: single vs. multiple interfaces
  \pause
  \item Context objects: per-registration vs. per-event
  \pause
  \item Dispatcher callback: sequential vs. multi-threaded
  \pause
  \item Dispatcher callback policy
  \end{itemize}
\end{frame}

\begin{frame}
  \frametitle{Observations}

  My observations after implementing the IM:
  \begin{itemize}
  \item priority mechanism can be tricky to implement
  \item don't forget to call the dispatchers
  \item context objects can be difficult to fill in
  \item once the ``hooks'' are there, adding functionality is easy
  \end{itemize}

\end{frame}

\begin{frame}
  \frametitle{IM Interceptors}

  Interceptors implemented for the IM:
  \begin{itemize}
  \item the GUI of the application itself
  \item a logger
  \item a very stupid translator (knows 4 words)
  \end{itemize}

\end{frame}

\begin{frame}
  \frametitle{IM Interceptors}

  Interceptors possible to be implemented for the IM:
  \begin{itemize}
  \item a message modifier (e.g. bad words remover)
  \item an encrypter
  \item a spell checker
  \item an auto replier
  \item ... many more
  \end{itemize}

\end{frame}

\begin{frame}
  \frametitle{Related Design Patters}

  Related to Interceptor Pattern are:

  \begin{block}{The Chain of Responsability Pattern}
  A variant of the Interceptor Pattern where processing can stop along the way.
  \end{block}

  \begin{block}{The Template Method Pattern}
  A lightweight variant of the Interceptor Pattern, with a single "interceptor".
  \end{block}

  \begin{block}{The Observer Pattern}
  Interceptor is a much improved Observer.
  \end{block}

\end{frame}

\begin{frame}
  \frametitle{Used Design Patters}

  Patterns that can be used together with the Interceptor Pattern are:

  \begin{block}{The Observer Pattern}
  Dispatcher callback is based on the Observer pattern.
  \end{block}

  \begin{block}{The Strategy Pattern}
  Can be used by Dispatchers for supporting more callback policies.
  \end{block}

  \begin{block}{The Singleton Pattern}
  Usually used to implement Dispatchers.
  \end{block}

  \begin{block}{The Proxy Pattern}
  Can be used to implement the Interceptor Proxy variant.
  \end{block}

\end{frame}

\subsection{Examples}

\begin{frame}
  \frametitle{Interceptors with EJBs}

  \begin{block}{Definition}
  An \alert{interceptor} is a method that intercepts a business method invocation.
  \end{block}

  \pause

  Possible locations:
  \begin{itemize}
  \item Enterprise Bean class
  \item separate Interceptor class
  \end{itemize}

\end{frame}

\begin{frame}
  \frametitle{Specifying Interceptors for EJBs}

  An Interceptor can be specified by:

  \begin{block}{Annotation (in Bean class)}
  \tt\alert{@AroundInvoke} \\
  public Object myInterceptor(InvocationContext ctx) \\
  \ \ throws Exception \{...\}
  \end{block}

  \begin{block}{Annotation (in separate class)}
  \tt\alert{@Interceptor(ProfilingInterceptor.class)} \\
  public class ProfilingInterceptor \{ \\
  \ \ \alert{@AroundInvoke} \\
  \ \ public Object profile(InvocationContext ctx) \\
  \ \ \ \ throws Exception \{...\} \\
  \}
  \end{block}

\end{frame}

\frame[all:1] {
\begin{block}{Deployment Descriptor}
\begin{lstlisting}
<session>
  <ejb-name>HelloEJB</ejb-name>
  <remote>mypackage.Hello</remote>
  <ejb-class>mypackage.HelloBean</ejb-class>
  ...
  <interceptor>
    <class>ProfilingInterceptor</class>
    <method-name>profile</method-name>
  </interceptor>
  <interceptor>
    <method-name>myInterceptor</method-name>
  </interceptor>
</session>
\end{lstlisting}
\end{block}
}

\begin{frame}
  \frametitle{Possible uses}

  Interceptors can be used to:
  \begin{itemize}
  \item modify parameters before they're passed to the bean
  \item modify the value returned from the bean
  \item catch and handle method exceptions
  \item interrupt the call completely
  \item provide method profiling
  \item many more
  \end{itemize}
\end{frame}

\frame[all:1] {
\begin{block}{InvocationContext Interface}
\begin{lstlisting}
public interface InvocationContext
{
  public Object getBean();
  public Method getMethod();
  public Object[] getParameters();
  public void setParameters(Object[]);
  public EJBContext getEJBContext();
  public java.util.Map getContextData();
  public Object proceed() throws Exception;
}
\end{lstlisting}
\end{block}
}

\begin{frame}
  \frametitle{Interceptors in JBoss}

  Interceptors are used in JBoss to provide various services such as:
  \begin{itemize}
  \item security
  \item transaction management
  \item logging
  \item persistence
  \item thread safety
  \item more on next slide
  \end{itemize}

\end{frame}

\frame[all:1] {
\begin{block}{Interceptors in JBoss}
\begin{lstlisting}
<container-interceptors>
  <interceptor>org.jboss.ejb.plugins.ProxyFactoryFinderInterceptor</interceptor>
  <interceptor>org.jboss.ejb.plugins.LogInterceptor</interceptor>
  <interceptor>org.jboss.ejb.plugins.SecurityInterceptor</interceptor>
  <interceptor>org.jboss.ejb.plugins.TxInterceptorCMT</interceptor>
  <interceptor>org.jboss.ejb.plugins.CallValidationInterceptor</interceptor>
  <interceptor>org.jboss.ejb.plugins.EntityCreationInterceptor</interceptor>
  <interceptor>org.jboss.ejb.plugins.EntityLockInterceptor</interceptor>
  ...
</container-interceptors>
\end{lstlisting}
\end{block}
}

\begin{frame}
  \frametitle{Other Examples}

  And the list goes on...
  \begin{itemize}
  \item Component-Based Application Servers (the Interceptor Proxy variant)
  \item CORBA implementations
  \item Web browsers
  \item Web servers
  \item Generally, all systems that support plugins
  \end{itemize}

\end{frame}

\section{Wrapping Up}

\subsection{Pattern Evaluation}

\begin{frame}
  \frametitle{Pro's}

  Advantages of Interceptor Pattern:
  \begin{itemize}
  \item Extensibility and flexibility
  \item Separation of concerns
  \item Allows monitoring and control of systems
  \item Promotes reusability
  \end{itemize}

\end{frame}

\begin{frame}
  \frametitle{Con's}

  Disadvantages of Interceptor Pattern:
  \begin{itemize}
  \item Difficult design decision
  \item Security and stability issues
  \item Possiblity for infinite invocation
  \end{itemize}

\end{frame}

\begin{frame}
  \frametitle{Conclusions}

  \begin{block}{The Interceptor Pattern is helpful if:}
  \begin{itemize}
  \item you need to design extensible frameworks
  \item you want people to keep out of the core code
  \end{itemize}
  \end{block}

  \pause

  \begin{block}{Using the pattern will imply:}
  \begin{itemize}
  \item increased complexity of your application
  \item opening the door to possible problems
  \end{itemize}
  \end{block}

\end{frame}

\begin{frame}
  \frametitle{Questions}

  \begin{center}
  ???
  \end{center}

\end{frame}

\subsection{Demo}

\begin{frame}
  \frametitle{Demo}

  Instant Messenger Demo...
\end{frame}

\subsection{Your Evaluation}

\begin{frame}
  \frametitle{15' Exercise}

  Assume you're designing the kernel of an operating system. \\
  These would be some of its components:
  \begin{itemize}
  \item Process/Thread Manager/Scheduler
  \item (Virtual) File System
  \item Virtual Memory Manager
  \item Device Drivers
  \end{itemize}

\end{frame}

\begin{frame}
  \frametitle{15' Exercise}

  Choose any one of these (or some other kernel component not on the
  list) and discuss the following issues, related to your chosen component:
  \begin{itemize}
  \item if the component implemented the Interceptor Pattern, how would
            it be useful to the system (name one or two possible uses)?
  \item what would you gain by making the component provide the Interceptor
            Pattern?
  \item what are the problems that this would create?
  \item do you think, then, that using the Interceptor Pattern for this would
            be a good idea? Why or why not?
  \end{itemize}

\end{frame}

\end{document}
