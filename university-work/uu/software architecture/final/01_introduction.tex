\chapter{Background information}

\section{Scope of this document}
In June 2004 two famous Dutch insurance companies (Zeker \& Vast
and ForSure) merged into SureThing. As a consequence of this merging,
the two companies needed to integrate both their hardware and
software into a single, unified, system which could serve for running the
business of the new SureThing. This document comes to describe the
architecture which we propose as a support for this hardware/software
integration.

The Zeker \& Vast Company had quite a different company strategy
than the ForSure Company. Zeker \& Vast has cooperated with some
hundreds of intermediaries for offering proposals and taking care
of claim handling for various amounts of customers. The intermediaries
communicated with the company through its Call-Center. The target
of this company was the middle class society. The ForSure Company did
not have a Call-Center, but had five account managers providing a single
point of contact to each of their wealthy customers, with each account
manager managing about 100 customers. The company's target was the
upper class society.

While Zeker \& Vast already comes with quite an advanced IT system,
ForSure has a rather rudimentary one, based mostly on the use of various
Microsoft Office tools by each account manager separately, with no standards
and no integration. Both of these systems are described to a certain
extent in the Request for Architecture which we have been presented
with and which can be made available separately. Some of the current
situation (especially the one at Zeker \& Vast) will also be discussed
in this document as a starting point for our changes.

After the merging, the SureThing Company operates at five different
locations in The Netherlands: three former locations of Zeker \&
Vast, one location of ForSure and the new headquarters. The
management desires that the new system provides a fully shared IT
application landscape that can be used by every employee
regardless of the location at which the employee resides. Most of
the hardware will be running at the old Zeker \& Vast headquarters in
Amsterdam, which is connected to each of the other locations by  2 Mbps
lines.

The management of the newly created SureThing Company has a few goals
which guided the developing the architecture presented in this document:

\begin{itemize}
\item they would like to ensure an increase in profit and future growth
of the market share of SureThing, as well as make the customer handling
process more efficient and less costly, by serving Zeker \& Vast's customers
directly, besides dealing with the intermediaries (e.g. by providing web access
to insurance proposal requests).
\item they would like to provide real-time proposal calculation, as opposed
to the current situation where requests for proposal calculations are honoured
only one day after they are made.
\item they would like to have an information system which allows them
more flexibility in defining the company's products.
\item they would like the new system to use a so-called Generic Output Component
which Zeker \& Vast currently has in development for all paper communication.
\item they would also desire a preferred supplier strategy for hardware (and
software).
\item since they are concerned about the social and cultural consequences that the
changes in the IT situation will trigger, they decided to adopt an incremental strategy
for developing and moving onto the new system.
\end{itemize}

\section{Assumptions}

The architecture presented in this document is based on a number of assumptions
which we have made (note: in real life these would have been clear facts which
we would've found out during the process of requirements elicitation). They are stated
below.

\begin{enumerate}

\item Currently, Zeker \& Vast only works with intermediaries. We make this assumption
in order to be able to limit the number of stakeholders which we consider.

\item We assume that management's "incremental strategy" translates to a desire
for an architecture which involves several transformation steps to reach the final goal.
This is a vital assumption for our architecture, as based on it we introduced several
phases for changing the current system, striving to ensure that in each of these
phases the system is fully operational and behaving correctly.

\item We assume that CHIPS \& BuRP do not use the Oracle database directly in order
to get / store their information, but rather use UCIS as an intermediary. We assume
this since one of the points from the Desired Situation section in our assignment description
read: ``CHIPS and BuRP will need to be upgraded to be able to interface with the new
UCIS''. Consequently, we believe that if CHIPS/BuRP had been using the database directly,
they would no longer have had to interface with UCIS (either the old or the new one).

\item On the other hand, we assume that STIFF accesses the Oracle database directly
(i.e., not through the UCIS system). Since the phrase ``Information from UCIS's Oracle
database is read directly [by STIFF]'' can be interpreted in either way, we simply chose to
give this interpretation.

\item Also regarding STIFF, we assume that it is started indirectly by UCIS, by placing
certain information in the Oracle database. This assumption was made because the
way in which communication takes place between UCIS, STIFF and the Oracle database
was not clearly mentioned in the description. In connection with this assumption, we
also assume there is some kind of cron daemon (see \cite{cron}) which starts STIFF
every night (since we are being told that ``STIFF is a batch-oriented system that runs
nightly'').

\item We assume that the information about clients and proposals of the ForSure
Company is still available even if the account managers have left the company. We
make this assumption since it is unlikely that company policy would've allowed
the account managers to take this information with them, supposedly leaving the
company without some of the records regarding its business.

\item As a worst case scenario, we assume that STIFF is written in Cobol, running on
quite old hardware, which would make it difficult to change in order to accept direct
requests for proposal calculation, leaving us with the only option of rewriting it
completely. We need STIFF to accept direct requests because that's the only way we
can make it compute proposals in real time (see also next assumptions).

\item We assume that STIFF in its current (Cobol) form can be run on new hardware
as well. As nothing was mentioned about the internals of the STIFF component in the
assignment description, we had to assume this in order to be able to provide solutions
for making STIFF run in real time.

\item We assume that the old STIFF software is not multithreaded. This assumption
is used in the physical view in order to justify the type of hardware that needs to be
bought.

\item We assume that computation of a single proposal by STIFF can be done in terms
of a few seconds on new hardware (and thus allow for a real time response to the customer).
We need this assumption because otherwise we cannot ensure real-time proposal calculation.

\item We assume that the SureThing company wants different access rights for different
groups of employees.

\item We assume that the current hardware configuration is as follows: most of the IT system
(including UCIS, STIFF and the Oracle database) will continue to run in the old headquarters from
Amsterdam (where it was running before the merging) and that 2 Mbps lines connect this IT
center to the other four locations of the SureThing Company. We had to assume this as there
is no information about how the system will be run after the merging and this was the most
reasonable assumption, since it involved the smallest amount of reorganization.

\item We assume SureThing does not own a spare rack mount cabinet at the moment. Clearly,
no mention of this appears in the description, and we need the assumption for cost computation.

\end{enumerate}

\section{Estimations}

\subsubsection{Number of requests for proposal}

One of the main aspects the system has to be designed for, is being able
to sustain a certain number of requests for proposals.

Measurements over the past year, as mentioned in the Request for
Architecture document, have indicated that about 140,000 proposals
were issued as a result of a request made to Zeker \& Vast. The document
also indicates that this number represents about 30\% of the total
number of calls made by customers to the Call-Center. This leads to the
conclusion that about 470,000 customers have contacted the company
during a year. Since we can expect quite an increase in the number of
requests for proposals once the company provides the possibility of requests
being issued also directly, through the Internet, we estimate that probably
the percentage of issued proposals will increase from 30\% to about 60\%.
We assume this because it is a known fact that customers, once no longer being
faced with a human being at the other end of the line, will tend to request
proposal calculations even without any intent whatsoever of purchasing
an insurance. The number of requests is thus expected to increase to 280,000.
If we add to this the number of customers the company will gain simply
by offering another means of contact - the Internet - in addition to the Call-Center,
we expect somewhere around 350,000 customers which will request proposals
to be computed. Since the number of customers ForSure currently has is about
500 (5 account managers x about 100 customers each), we can safely go
with the 350,000 figure as the final estimation (since the 500 extra will make
no difference).

This number of customers implies an average customer load of 1,000 customers
a day, which (as we assume they are more or less all in the same time zone) will
all be accessing the system probably during daytime, i.e., 10 to 12 hours. This
yields an average load of about 1 or 2 customers / second, and a maximum load
of probably somewhere around 10 to 20 customers / second.

\subsubsection{Number of developers needed}

We do have quite a large system to deal with, on one hand, but most of the
components are already available, on the other, which means that the development
work involved will not be extremely high. We have therefore created this architecture
on the estimation of having about 8 to 10 developers available for the duration
of the implementation. We believe that this number is sufficient to ensure the
development, deployment and initial testing (with possible corrections) of the
system in the 11 months which are available until the end of 2005.