\documentclass{beamer}

\mode<presentation>
{
  \usetheme{Warsaw}

  \setbeamercovered{transparent}
}

\usepackage[english]{babel}
\usepackage[latin1]{inputenc}
\usepackage{hyperref}
\usepackage{pgf,pgfarrows}

\title{The Internals of the Monet Database}

\author{
  Bogdan Dumitriu \and
  Lee Provoost
}

\institute
{
  Department of Computer Science\\
  University of Utrecht
}

\date{\today}

% If you wish to uncover everything in a step-wise fashion, uncomment
% the following command: 
%\beamerdefaultoverlayspecification{<+->}

\begin{document}

\begin{frame}
  \titlepage
\end{frame}

\begin{frame}
  \frametitle{Outline}
  \tableofcontents
\end{frame}

\section{Databases}

\subsection{Classical databases}

\begin{frame}
  \frametitle{Databases}
  \framesubtitle{Classical databases}

Most existing databases:
  \begin{itemize}
  \item OLTP oriented
  \pause
  \item high performance on large \# of small updates
  \pause
  \item table data clustered by row on disk
  \pause
  \item \alert{unsuited for query-intensive due to}
  \begin{itemize}
  \item \alert{a lot of unnecessary I/O}
  \end{itemize}
  \end{itemize}

\end{frame}

\subsection{Monet database}

\begin{frame}
  \frametitle{Databases}
  \framesubtitle{Vertical fragmentation}

  Keyword: \alert{vertical fragmentation}

\begin{center}
  \begin{tabular}{|c|c|c|c|}
    \hline
    Id & Name & Postal Code & Date of Birth \\
    \hline
    1 & John & 2345 BP & 17-09-1976 \\
    2 & Jane & 6146 TY & 21-04-1959 \\
    3 & Bob & 8127 PR & 04-04-1990 \\
    \hline
  \end{tabular}
\end{center}

  \begin{columns}

  \column{2cm}

  \begin{tabular}{|c|c|}
    \hline
    Id & Name \\
    \hline
    1 & John \\
    2 & Jane \\
    3 & Bob \\
    \hline
  \end{tabular}

  \column{2cm}

  \begin{tabular}{|c|c|}
    \hline
    Id & Postal Code \\
    \hline
    1 & 2345 BP \\
    2 & 6146 TY \\
    3 & 8127 PR \\
    \hline
  \end{tabular}

  \column{2cm}

  \begin{tabular}{|c|c|}
    \hline
    Id & Date of Birth \\
    \hline
    1 & 17-09-1976 \\
    2 & 21-04-1959 \\
    3 & 04-04-1990 \\
    \hline
  \end{tabular}

  \end{columns}

\end{frame}

\begin{frame}
  \frametitle{Databases}
  \framesubtitle{Monet goals}

The goals of the Monet database:
\begin{enumerate}
\item \alert{primary: achieve high performance on query-intensive applications}
\pause
\item support multiple logical data models
\item providing parallelism
\item extensibility to specific application domains
\end{enumerate}

\end{frame}

\begin{frame}
  \frametitle{Databases}
  \framesubtitle{Monet model}

The model employed by Monet:
\begin{itemize}
\item Decomposed Storage Model
\item kernel of primitives on binary tables
\pause
\item vertical fragmentation is explicit
\pause
\item a simple, elegant and very flexible model
\pause
\item downside: a lot of joining (partially solved)
\end{itemize}

\end{frame}

\begin{frame}
  \frametitle{Databases}
  \framesubtitle{Monet and main memory}

Monet is above all a main memory DBMS:
\begin{itemize}
\item shifts cost of processing from I/O to CPU cycles
\item both its algorithms \& its data structures are optimized for main memory access
\item not a main memory-only DBMS, though:
\begin{itemize}
\item uses OS-controlled virtual memory during operation
\item uses disk for long-term storage (naturally)
\end{itemize}
\end{itemize}

\end{frame}

\section{The MIL Language}

\subsection{The language}

\begin{frame}
  \frametitle{The MIL Language}
  \framesubtitle{Monet architecture}

  \begin{center}
  \pgfdeclareimage[interpolate=true,height=7cm]{architecture}{img/architecture}
  \pgfuseimage{architecture}
  \end{center}
\end{frame}

\begin{frame}
  \frametitle{The MIL Language}
  \framesubtitle{The language in a nutshell}

  \begin{center}
  \begin{pgfpicture}{0cm}{0cm}{7cm}{7cm}
    \pgfdeclareimage[interpolate=true,height=7cm]{mil}{img/mil}
    \begin{pgfrotateby}{\pgfdegree{-1}}
    \pgfuseimage{mil}
    \end{pgfrotateby}
  \end{pgfpicture}
  \end{center}
\end{frame}

\begin{frame}
  \frametitle{The MIL Language}
  \framesubtitle{MIL types}

  \begin{block}{1. $t \in \mathcal{A}_f  \cup \mathcal{A}_v \Rightarrow t \in \mathcal{T}$}
    \begin{itemize}
    \item atomic data types
    \item fixed: $\mathcal{A}_f  = \lbrace{\tt bit,chr,sht,int,lng,flt,dbl,oid}\rbrace$
    \item variable $\mathcal{A}_v  = \lbrace{\tt str}\rbrace$
    \end{itemize}
  \end{block}

  \pause

  \begin{block}{2. $T_1,T_2 \in \mathcal{T} \Rightarrow {\tt bat}[T_1,T_2] \in \mathcal{T}$}
    \begin{itemize}
    \item the BAT (Binary Table) type
    \item each tuple in a BAT called a Binary Unit (BUN)
    \item left column - \textit{head}
    \item right column - \textit{tail}
    \item can be nested
    \end{itemize}
  \end{block}

\end{frame}

\begin{frame}
  \frametitle{The MIL Language}
  \framesubtitle{Ingmar's question}

  \begin{block}{Ingmar:}
  On page 6 it is mentioned that MIL supports nested BATs. That
  sounds really interesting, but what are they (BATs within BATs,
  because that doesn't sound like a BAT anymore)?
  \end{block}

\end{frame}

\begin{frame}
  \frametitle{The MIL Language}
  \framesubtitle{MIL features}

Main features of the language:
  \begin{itemize}
  \item basic unit of execution: the \alert{operator}
  \pause
  \item operators can be \alert{overloaded}, most are also \alert{polymorphic}
  \pause
  \item MIL is a \alert{dynamically typed} language
  \pause
  \item a \alert{procedural} block-structured language (if-then-else, while-do, @)
  \pause
  \item allows \alert{extension modules}
  \pause
  \item provides the usual operators ($=$, $\neq$, $<$, $>$, $\leq$, $\geq$, etc.)
  \end{itemize}

\end{frame}

\subsection{The BAT algebra}

\begin{frame}
  \frametitle{The MIL Language}
  \framesubtitle{The BAT algebra}

  Core functionality of MIL offered by a \alert{BAT algebra} of operators which:
\begin{itemize}
\item have an algebraic definition
\item are free of side-effects
\item take BATs and return BATs $\Rightarrow$ closed algebra on BATs
\end{itemize}

\end{frame}

\begin{frame}
  \frametitle{The MIL Language}
  \framesubtitle{The mirror operator}

  \begin{block}{Ingmar:}
  What's the use of the mirror operator? Why would you want a table with identical columns?
  \end{block}

  Possible uses:

  \begin{itemize}
  \item perform an set operation on a BAT (or on 2 BATs)
  \item perform a join operation\ldots
  \item certainly many others
  \end{itemize}

\end{frame}

\begin{frame}
  \frametitle{The MIL Language}
  \framesubtitle{The mirror operator}

  \begin{columns}

  \column{1cm}

  \begin{tabular}{|c|c|}
    \hline
    Id & Name \\
    \hline
    4 & Jack \\
    2 & Bill \\
    1 & Bob \\
    6 & Clare \\
    \hline
  \end{tabular}

  \column{1cm}

  \begin{tabular}{|c|c|}
    \hline
    Id & Name \\
    \hline
    1 & Daniels \\
    2 & Gates \\
    3 & Hope \\
    4 & Jones \\
    5 & Heart \\
    6 & James \\
    \hline
  \end{tabular}

  \end{columns}

  \pause

  \begin{center}$\downarrow{\tt mirror}$\end{center}

\end{frame}

\begin{frame}
  \frametitle{The MIL Language}
  \framesubtitle{The mirror operator}

  \begin{columns}

  \column{1cm}

  \begin{tabular}{|c|c|}
    \hline
    Id & Id \\
    \hline
    4 & 4 \\
    2 & 2 \\
    1 & 1 \\
    6 & 6 \\
    \hline
  \end{tabular}

  \column{1cm}

  \begin{tabular}{|c|c|}
    \hline
    Id & Name \\
    \hline
    1 & Daniels \\
    2 & Gates \\
    3 & Hope \\
    4 & Jones \\
    5 & Heart \\
    6 & James \\
    \hline
  \end{tabular}

  \column{1cm}

  $\longrightarrow{join}$

  \column{1cm}

  \begin{tabular}{|c|c|}
    \hline
    Id & Id \\
    \hline
    4 & Jones \\
    2 & Gates \\
    1 & Daniels \\
    6 & James \\
    \hline
  \end{tabular}

  \end{columns}

\end{frame}

\begin{frame}
  \frametitle{The MIL Language}
  \framesubtitle{Grouping operators}

  \begin{columns}

  \column{2cm}

  \onslide<1-2>{
  \begin{tabular}{|c|c|}
    \hline
    Id & Name \\
    \hline
    1 & Bob \\
    2 & Amy \\
    3 & Bob \\
    4 & Clare \\
    5 & Clare \\
    6 & Bob \\
    \hline
  \end{tabular}
  }

  \column{1cm}

  \onslide<2>{$\longrightarrow{\tt unary\ group}$}

  \column{2cm}

  \onslide<2->{
  \begin{tabular}{|c|c|}
    \hline
    Id & GID \\
    \hline
    1 & 1 \\
    2 & 2 \\
    3 & 1 \\
    4 & 4 \\
    5 & 4 \\
    6 & 1 \\
    \hline
  \end{tabular}
  }

  \end{columns}

  \begin{columns}

  \column{2cm}

  \onslide<1->{
  \begin{tabular}{|c|c|}
    \hline
    Id & Year-of-birth \\
    \hline
    1 & 1949 \\
    2 & 1948 \\
    3 & 1950 \\
    4 & 1948 \\
    5 & 1948 \\
    6 & 1950 \\
    \hline
  \end{tabular}
  }

  \column{1cm}

  \onslide<3>{$\longrightarrow{\tt binary\ group}$}

  \column{2cm}

  \onslide<3>{
  \begin{tabular}{|c|c|}
    \hline
    Id & GID \\
    \hline
    1 & 1 \\
    2 & 2 \\
    3 & 3 \\
    4 & 4 \\
    5 & 4 \\
    6 & 3 \\
    \hline
  \end{tabular}
  }

  \end{columns}

\end{frame}

\begin{frame}
  \frametitle{The MIL Language}
  \framesubtitle{Split operator}

  \begin{columns}

  \column{2cm}

  \begin{tabular}{|c|c|}
    \hline
    Id & Name \\
    \hline
    1 & Bob \\
    2 & Amy \\
    3 & Joe \\
    4 & Clare \\
    5 & Susan \\
    6 & Jeff \\
    \hline
  \end{tabular}

  \column{1cm}

  $\longrightarrow{\tt split\ (n=2)}$

  \column{2cm}

  \begin{tabular}{|c|c|}
    \hline
    Id & Id \\
    \hline
    1 & 3 \\
    4 & 6 \\
    \hline
  \end{tabular}

  \end{columns}

\end{frame}

\begin{frame}
  \frametitle{The MIL Language}
  \framesubtitle{Fragment operator}

  \begin{columns}

  \column{2cm}

  \begin{tabular}{|c|c|}
    \hline
    Id & Name \\
    \hline
    1 & Bob \\
    2 & Amy \\
    3 & Joe \\
    4 & Clare \\
    5 & Susan \\
    6 & Jeff \\
    \hline
  \end{tabular}

  \column{2cm}

  \begin{tabular}{|c|c|}
    \hline
    Id & Id \\
    \hline
    1 & 3 \\
    4 & 6 \\
    \hline
  \end{tabular}

  \column{1cm}

  $\longrightarrow{\tt fragment}$

  \column{2cm}

  \begin{tabular}{|c|c|}
    \hline
    Id & BAT \\
    \hline
    3 & \begin{tabular}{|c|c|}\hline 1 & nil \\ 2 & nil \\ 3 & nil \\ \hline \end{tabular} \\
    \hline
    6 & \begin{tabular}{|c|c|}\hline 4 & nil \\ 5 & nil \\ 6 & nil \\ \hline \end{tabular} \\
    \hline
  \end{tabular}

  \end{columns}

\end{frame}

\begin{frame}
  \frametitle{The MIL Language}
  \framesubtitle{Multi-join map}

\begin{center}
Let \alert{$vol(x,y,z)=x*y*z$} be the function for computing the volume of a parallelepiped.
\end{center}

  \begin{columns}

  \column{1cm}

  \begin{tabular}{|c|c|}
    \hline
    Id & W \\
    \hline
    1 & 5 \\
    2 & 7 \\
    3 & 9 \\
    4 & 1 \\
    \hline
  \end{tabular}

  \column{1cm}

  \begin{tabular}{|c|c|}
    \hline
    Id & H \\
    \hline
    1 & 2 \\
    2 & 6 \\
    3 & 1 \\
    4 & 7 \\
    \hline
  \end{tabular}

  \column{1cm}

  \begin{tabular}{|c|c|}
    \hline
    Id & L \\
    \hline
    1 & 4 \\
    2 & 2 \\
    3 & 2 \\
    4 & 8 \\
    \hline
  \end{tabular}

%\end{columns}

%\begin{columns}

  \column{1cm}

  $\longrightarrow{\tt [vol]}$

  \column{1cm}

  \begin{tabular}{|c|c|}
    \hline
    Id & Vol \\
    \hline
    1 & 40 \\
    2 & 98 \\
    3 & 18 \\
    4 & 56 \\
    \hline
  \end{tabular}

  \end{columns}

\end{frame}

\begin{frame}
  \frametitle{The MIL Language}
  \framesubtitle{Pump}

  \begin{columns}

  \column{1cm}

  \begin{tabular}{|c|c|}
    \hline
    GID & Name \\
    \hline
    2 & July \\
    1 & Ethan \\
    2 & Jill \\
    1 & Clare \\
    3 & Bob \\
    4 & Tim \\
    \hline
  \end{tabular}

  \column{1cm}

  $\longrightarrow{\tt count}$

  \column{1cm}

  6

  \end{columns}

  \pause

  \begin{columns}

  \column{1cm}

  \begin{tabular}{|c|c|}
    \hline
    GID & Name \\
    \hline
    2 & July \\
    1 & Ethan \\
    2 & Jill \\
    1 & Clare \\
    3 & Bob \\
    4 & Tim \\
    \hline
  \end{tabular}

  \column{1cm}

  \begin{tabular}{|c|c|}
    \hline
    GID & \\
    \hline
    1 & \\
    2 & \\
    4 & \\
    5 & \\
    \hline
  \end{tabular}

  \column{1cm}

  $\longrightarrow{\tt \lbrace{}count\rbrace}$

  \column{1cm}

  \begin{tabular}{|c|c|}
    \hline
    GID & Count \\
    \hline
    1 & 2 \\
    2 & 2 \\
    4 & 1 \\
    5 & 0 \\
    \hline
  \end{tabular}

  \end{columns}

\end{frame}

\begin{frame}
  \frametitle{The MIL Language}
  \framesubtitle{Peter's question}

  \ldots which I hope answers Peter's request:

  \begin{block}{Peter:}
  Can you show an example of how the pump operator is used, and the
  results it creates?
  \end{block}

  \pause

  \begin{block}{Peter:}
  What is the \textit{str} $s$ in the {\tt save}, {\tt load} and {\tt remove}
  operators?
  \end{block}

\end{frame}

\begin{frame}
  \frametitle{The MIL Language}
  \framesubtitle{Peter's question}

  \begin{itemize}
  \item A BAT Buffer Pool manages all known BATs
  \item It administers logical \& physical names
  \item {\tt bbpname (BAT[any,any], str s) : bit} can be used to name a BAT
  \item This name is global
  \item The \textit{str} $s$ refers to this logical name
  \end{itemize}


\end{frame}

\subsection{OQL to MIL translation}

\begin{frame}
  \frametitle{The MIL Language}
  \framesubtitle{Laurence's question}

  \begin{block}{Laurence:}
  Can you show, in a step-by-step fashion how the OQL query on page 109
  is translated to MIL?
  \end{block}

\end{frame}

\begin{frame}
  \frametitle{The MIL Language}
  \framesubtitle{OQL to MIL translation}

  \begin{block}{Order class}
{\tt
class Order $\lbrace$ \\
\ \ attribute date day; \\
\ \ attribute float discount; \\
\ \ relation Set$<$Item$>$ items; \\
$\rbrace$
}
  \end{block}

  \pause

  \begin{block}{Item class}
{\tt
class Item $\lbrace$ \\
\ \ attribute float price; \\
\ \ attribute float tax; \\
\ \ relation Order order inverse Order.items; \\
$\rbrace$
}
  \end{block}

\end{frame}

\begin{frame}
  \frametitle{The MIL Language}
  \framesubtitle{OQL to MIL translation}

  \begin{block}{The OQL query}
{\tt
SELECT year, sum(total) \\
FROM ( SELECT price * tax AS total, \\
\ \ \ \ \ \ \ \ \ \ \ \ \ \ year(item.order.day) AS year \\
\ \ \ \ \ \ \ FROM\ \ \ item \\
\ \ \ \ \ \ \ WHERE\ \ order.discount BETWEEN 0.00 AND 0.06) \\
GROUP BY year \\
ORDER BY year
}
  \end{block}

\end{frame}

\begin{frame}
  \frametitle{The MIL Language}
  \framesubtitle{OQL to MIL translation}

  The Order class:

  \begin{columns}

  \column{1cm}

  order\_day
  \begin{tabular}{|c|c|}
    \hline
    oid & date \\
    \hline
    100 & 4/4/98 \\
    101 & 9/4/98 \\
    102 & 1/2/98 \\
    103 & 9/4/98 \\
    104 & 7/2/98 \\
    105 & 1/2/98 \\
    \hline
  \end{tabular}

  \column{1cm}

  order\_discount
  \begin{tabular}{|c|c|}
    \hline
    oid & float \\
    \hline
    100 & 0.175 \\
    101 & 0.065 \\
    102 & 0.175 \\
    103 & 0.000 \\
    104 & 0.000 \\
    105 & 0.065 \\
    \hline
  \end{tabular}

  \column{1cm}

  order\_item
  \begin{tabular}{|c|c|}
    \hline
    oid & oid \\
    \hline
    100 & 1000 \\
    100 & 1001 \\
    101 & 1002 \\
    101 & 1003 \\
    101 & 1004 \\
    102 & 1005 \\
    103 & 1006 \\
    103 & 1007 \\
    103 & 1008 \\
    104 & 1009 \\
    104 & 1010 \\
    \hline
  \end{tabular}

  \end{columns}

\end{frame}

\begin{frame}
  \frametitle{The MIL Language}
  \framesubtitle{OQL to MIL translation}

  The Item class:

  \begin{columns}

  \column{1cm}

  item\_price
  \begin{tabular}{|c|c|}
    \hline
    oid & float \\
    \hline
    1000 & 04.75 \\
    1001 & 11.50 \\
    1002 & 10.20 \\
    1003 & 75.00 \\
    1004 & 02.50 \\
    1005 & 92.80 \\
    1006 & 37.50 \\
    1007 & 14.25 \\
    1008 & 17.99 \\
    1009 & 22.33 \\
    1010 & 42.67 \\
    \hline
  \end{tabular}

  \column{1cm}

  item\_tax
  \begin{tabular}{|c|c|}
    \hline
    oid & float \\
    \hline
    1000 & 0.10 \\
    1001 & 0.00 \\
    1002 & 0.00 \\
    1003 & 0.00 \\
    1004 & 0.00 \\
    1005 & 0.10 \\
    1006 & 0.10 \\
    1007 & 0.00 \\
    1008 & 0.00 \\
    1009 & 0.00 \\
    1010 & 0.10 \\
    \hline
  \end{tabular}

  \column{1cm}

  item\_order
  \begin{tabular}{|c|c|}
    \hline
    oid & oid \\
    \hline
    1000 & 100 \\
    1001 & 100 \\
    1002 & 101 \\
    1003 & 101 \\
    1004 & 101 \\
    1005 & 102 \\
    1006 & 103 \\
    1007 & 103 \\
    1008 & 103 \\
    1009 & 104 \\
    1010 & 104 \\
    \hline
  \end{tabular}

  \end{columns}

\end{frame}

\begin{frame}
  \frametitle{The MIL Language}
  \framesubtitle{OQL to MIL translation}

  \begin{block}{}
  {\tt ORD\_NIL $:=$ select(order\_discount, "between", 0.0, 0.6)}
  \end{block}

  \begin{columns}

  \column{1cm}

  order\_discount
  \begin{tabular}{|c|c|}
    \hline
    oid & float \\
    \hline
    100 & 0.175 \\
    101 & 0.065 \\
    102 & 0.175 \\
    103 & 0.000 \\
    104 & 0.000 \\
    105 & 0.065 \\
    \hline
  \end{tabular}

  \column{1cm}

  $\longrightarrow$

  \column{1cm}

  ORD\_NIL
  \begin{tabular}{|c|c|}
    \hline
    oid & oid \\
    \hline
    100 & nil \\
    102 & nil \\
    103 & nil \\
    104 & nil \\
    \hline
  \end{tabular}

  \end{columns}

\end{frame}

\begin{frame}
  \frametitle{The MIL Language}
  \framesubtitle{OQL to MIL translation}

  \begin{block}{}
  {\tt ORD\_SEL $:=$ ORD\_NIL.mark(oid(0))}
  \end{block}

  \begin{columns}

  \column{1cm}

  ORD\_NIL
  \begin{tabular}{|c|c|}
    \hline
    oid & oid \\
    \hline
    100 & nil \\
    102 & nil \\
    103 & nil \\
    104 & nil \\
    \hline
  \end{tabular}

  \column{1cm}

  $\longrightarrow$

  \column{1cm}

  ORD\_SEL
  \begin{tabular}{|c|c|}
    \hline
    oid & oid \\
    \hline
    100 & 0 \\
    102 & 1 \\
    103 & 2 \\
    104 & 3 \\
    \hline
  \end{tabular}

  \end{columns}

\end{frame}

\begin{frame}
  \frametitle{The MIL Language}
  \framesubtitle{OQL to MIL translation}

  \begin{block}{}
  {\tt SEL\_DAY $:=$ join(ORD\_SEL.reverse, order\_day, "$=$")}
  \end{block}

  \begin{columns}

  \column{2cm}

  ORD\_SEL.reverse
  \begin{tabular}{|c|c|}
    \hline
    oid & oid \\
    \hline
    0 & 100 \\
    1 & 102 \\
    2 & 103 \\
    3 & 104 \\
    \hline
  \end{tabular}

  \column{1.5cm}

  order\_day
  \begin{tabular}{|c|c|}
    \hline
    oid & date \\
    \hline
    100 & 4/4/98 \\
    101 & 9/4/98 \\
    102 & 1/2/98 \\
    103 & 9/4/98 \\
    104 & 7/2/98 \\
    105 & 1/2/98 \\
    \hline
  \end{tabular}

  \column{0.5cm}

  $\longrightarrow$

  \column{1cm}

  SEL\_DAY
  \begin{tabular}{|c|c|}
    \hline
    oid & date \\
    \hline
    0 & 4/4/98 \\
    1 & 1/2/98 \\
    2 & 9/4/98 \\
    3 & 7/2/98 \\
    \hline
  \end{tabular}

  \end{columns}

\end{frame}

\begin{frame}
  \frametitle{The MIL Language}
  \framesubtitle{OQL to MIL translation}

  \begin{block}{}
  {\tt SEL\_YEA $:=$ [year](SEL\_DAY)}
  \end{block}

  \begin{columns}

  \column{1cm}

  SEL\_DAY
  \begin{tabular}{|c|c|}
    \hline
    oid & date \\
    \hline
    0 & 4/4/98 \\
    1 & 1/2/98 \\
    2 & 9/4/98 \\
    3 & 7/2/98 \\
    \hline
  \end{tabular}

  \column{1cm}

  $\longrightarrow$

  \column{1cm}

  SEL\_YEA
  \begin{tabular}{|c|c|}
    \hline
    oid & int \\
    \hline
    0 & 98 \\
    1 & 98 \\
    2 & 98 \\
    3 & 98 \\
    \hline
  \end{tabular}

  \end{columns}

\end{frame}

\begin{frame}
  \frametitle{The MIL Language}
  \framesubtitle{OQL to MIL translation}

  \begin{block}{}
  {\tt GRP\_SEL $:=$ group(SEL\_YEA).reverse}
  \end{block}

  \begin{columns}

  \column{1cm}

  SEL\_YEA
  \begin{tabular}{|c|c|}
    \hline
    oid & int \\
    \hline
    0 & 98 \\
    1 & 98 \\
    2 & 98 \\
    3 & 98 \\
    \hline
  \end{tabular}

  \column{1cm}

  $\longrightarrow$

  \column{1cm}

  GRP\_SEL
  \begin{tabular}{|c|c|}
    \hline
    oid & oid \\
    \hline
    0 & 0 \\
    0 & 1 \\
    0 & 2 \\
    0 & 3 \\
    \hline
  \end{tabular}

  \end{columns}

\end{frame}

\begin{frame}
  \frametitle{The MIL Language}
  \framesubtitle{OQL to MIL translation}

  \begin{block}{}
  {\tt GRP\_GRP $:=$ unique(GRP\_SEL.mirror)}
  \end{block}

  \begin{columns}

  \column{1cm}

  GRP\_SEL.mirror
  \begin{tabular}{|c|c|}
    \hline
    oid & int \\
    \hline
    0 & 0 \\
    0 & 0 \\
    0 & 0 \\
    0 & 0 \\
    \hline
  \end{tabular}

  \column{1cm}

  $\longrightarrow$

  \column{1cm}

  GRP\_GRP
  \begin{tabular}{|c|c|}
    \hline
    oid & oid \\
    \hline
    0 & 0 \\
    \hline
  \end{tabular}

  \end{columns}

\end{frame}

\begin{frame}
  \frametitle{The MIL Language}
  \framesubtitle{OQL to MIL translation}

  \begin{block}{}
  {\tt GRP\_YEA $:=$ join(GRP\_GRP, SEL\_YEA, "$=$")}
  \end{block}

  \begin{columns}

  \column{2cm}

  GRP\_GRP
  \begin{tabular}{|c|c|}
    \hline
    oid & oid \\
    \hline
    0 & 0 \\
    \hline
  \end{tabular}

  \column{1.5cm}

  SEL\_YEA
  \begin{tabular}{|c|c|}
    \hline
    oid & int \\
    \hline
    0 & 98 \\
    1 & 98 \\
    2 & 98 \\
    3 & 98 \\
    \hline
  \end{tabular}

  \column{0.5cm}

  $\longrightarrow$

  \column{1cm}

  GRP\_YEA
  \begin{tabular}{|c|c|}
    \hline
    oid & int \\
    \hline
    0 & 98 \\
    \hline
  \end{tabular}

  \end{columns}

\end{frame}

\begin{frame}
  \frametitle{The MIL Language}
  \framesubtitle{OQL to MIL translation}

  \begin{block}{}
  {\tt ITM\_SEL $:=$ join(item\_order, ORD\_SEL, "$=$")}
  \end{block}

  \begin{columns}

  \column{2cm}

  item\_order
  \begin{tabular}{|c|c|}
    \hline
    oid & oid \\
    \hline
    1000 & 100 \\
    1001 & 100 \\
    1002 & 101 \\
    1003 & 101 \\
    1004 & 101 \\
    1005 & 102 \\
    1006 & 103 \\
    1007 & 103 \\
    1008 & 103 \\
    1009 & 104 \\
    1010 & 104 \\
    \hline
  \end{tabular}

  \column{1.5cm}

  ORD\_SEL
  \begin{tabular}{|c|c|}
    \hline
    oid & oid \\
    \hline
    100 & 0 \\
    102 & 1 \\
    103 & 2 \\
    104 & 3 \\
    \hline
  \end{tabular}

  \column{0.5cm}

  $\longrightarrow$

  \column{1cm}

  ITM\_SEL
  \begin{tabular}{|c|c|}
    \hline
    oid & oid \\
    \hline
    1000 & 0 \\
    1001 & 0 \\
    1005 & 1 \\
    1006 & 2 \\
    1007 & 2 \\
    1008 & 2 \\
    1009 & 3 \\
    1010 & 3 \\
    \hline
  \end{tabular}

  \end{columns}

\end{frame}

\begin{frame}
  \frametitle{The MIL Language}
  \framesubtitle{OQL to MIL translation}

  \begin{block}{}
  {\tt UNQ\_ITM $:=$ ITM\_SEL.mark(oid(0)).reverse}
  \end{block}

  \begin{columns}

  \column{1cm}

  ITM\_SEL
  \begin{tabular}{|c|c|}
    \hline
    oid & oid \\
    \hline
    1000 & 0 \\
    1001 & 0 \\
    1005 & 1 \\
    1006 & 2 \\
    1007 & 2 \\
    1008 & 2 \\
    1009 & 3 \\
    1010 & 3 \\
    \hline
  \end{tabular}

  \column{1cm}

  $\longrightarrow$

  \column{1cm}

  UNQ\_ITM
  \begin{tabular}{|c|c|}
    \hline
    oid & oid \\
    \hline
    0 & 1000 \\
    1 & 1001 \\
    2 & 1005 \\
    3 & 1006 \\
    4 & 1007 \\
    5 & 1008 \\
    6 & 1009 \\
    7 & 1010 \\
    \hline
  \end{tabular}

  \end{columns}

\end{frame}

\begin{frame}
  \frametitle{The MIL Language}
  \framesubtitle{OQL to MIL translation}

  \begin{block}{}
  {\tt SEL\_UNQ $:=$ ITM\_SEL.reverse.mark(oid(0))}
  \end{block}

  \begin{columns}

  \column{1cm}

  ITM\_SEL
  \begin{tabular}{|c|c|}
    \hline
    oid & oid \\
    \hline
    1000 & 0 \\
    1001 & 0 \\
    1005 & 1 \\
    1006 & 2 \\
    1007 & 2 \\
    1008 & 2 \\
    1009 & 3 \\
    1010 & 3 \\
    \hline
  \end{tabular}

  \column{1cm}

  $\longrightarrow$

  \column{1cm}

  SEL\_UNQ
  \begin{tabular}{|c|c|}
    \hline
    oid & oid \\
    \hline
    0 & 0 \\
    0 & 1 \\
    1 & 2 \\
    2 & 3 \\
    2 & 4 \\
    2 & 5 \\
    3 & 6 \\
    3 & 7 \\
    \hline
  \end{tabular}

  \end{columns}

\end{frame}

\begin{frame}
  \frametitle{The MIL Language}
  \framesubtitle{OQL to MIL translation}

  \begin{block}{}
  {\tt UNQ\_PRI $:=$ join(UNQ\_ITM, item\_price, "$=$")}
  \end{block}

  \begin{columns}

  \column{2cm}

  UNQ\_ITM
  \begin{tabular}{|c|c|}
    \hline
    oid & oid \\
    \hline
    0 & 1000 \\
    1 & 1001 \\
    2 & 1005 \\
    3 & 1006 \\
    4 & 1007 \\
    5 & 1008 \\
    6 & 1009 \\
    7 & 1010 \\
    \hline
  \end{tabular}

  \column{1.5cm}

  item\_price
  \begin{tabular}{|c|c|}
    \hline
    oid & float \\
    \hline
    1000 & 04.75 \\
    1001 & 11.50 \\
    1002 & 10.20 \\
    1003 & 75.00 \\
    1004 & 02.50 \\
    1005 & 92.80 \\
    1006 & 37.50 \\
    1007 & 14.25 \\
    1008 & 17.99 \\
    1009 & 22.33 \\
    1010 & 42.67 \\
    \hline
  \end{tabular}

  \column{0.5cm}

  $\longrightarrow$

  \column{1cm}

  UNQ\_PRI
  \begin{tabular}{|c|c|}
    \hline
    oid & float \\
    \hline
    0 & 04.75 \\
    1 & 11.50 \\
    2 & 92.80 \\
    3 & 37.50 \\
    4 & 14.25 \\
    5 & 17.99 \\
    6 & 22.33 \\
    7 & 42.67 \\
    \hline
  \end{tabular}

  \end{columns}

\end{frame}

\begin{frame}
  \frametitle{The MIL Language}
  \framesubtitle{OQL to MIL translation}

  \begin{block}{}
  {\tt UNQ\_TAX $:=$ join(UNQ\_ITM, item\_tax, "$=$")}
  \end{block}

  \begin{columns}

  \column{2cm}

  UNQ\_ITM
  \begin{tabular}{|c|c|}
    \hline
    oid & oid \\
    \hline
    0 & 1000 \\
    1 & 1001 \\
    2 & 1005 \\
    3 & 1006 \\
    4 & 1007 \\
    5 & 1008 \\
    6 & 1009 \\
    7 & 1010 \\
    \hline
  \end{tabular}

  \column{1.5cm}

  item\_tax
  \begin{tabular}{|c|c|}
    \hline
    oid & float \\
    \hline
    1000 & 0.10 \\
    1001 & 0.00 \\
    1002 & 0.00 \\
    1003 & 0.00 \\
    1004 & 0.00 \\
    1005 & 0.10 \\
    1006 & 0.10 \\
    1007 & 0.00 \\
    1008 & 0.00 \\
    1009 & 0.00 \\
    1010 & 0.10 \\
    \hline
  \end{tabular}

  \column{0.5cm}

  $\longrightarrow$

  \column{1cm}

  UNQ\_TAX
  \begin{tabular}{|c|c|}
    \hline
    oid & float \\
    \hline
    0 & 0.10 \\
    1 & 0.00 \\
    2 & 0.10 \\
    3 & 0.10 \\
    4 & 0.00 \\
    5 & 0.00 \\
    6 & 0.00 \\
    7 & 0.10 \\
    \hline
  \end{tabular}

  \end{columns}

\end{frame}

\begin{frame}
  \frametitle{The MIL Language}
  \framesubtitle{OQL to MIL translation}

  \begin{block}{}
  {\tt UNQ\_TOT $:=$ [*](UNQ\_PRI, UNQ\_TAX)}
  \end{block}

  \begin{columns}

  \column{2cm}

  UNQ\_PRI
  \begin{tabular}{|c|c|}
    \hline
    oid & float \\
    \hline
    0 & 04.75 \\
    1 & 11.50 \\
    2 & 92.80 \\
    3 & 37.50 \\
    4 & 14.25 \\
    5 & 17.99 \\
    6 & 22.33 \\
    7 & 42.67 \\
    \hline
  \end{tabular}

  \column{1.5cm}

  UNQ\_TAX
  \begin{tabular}{|c|c|}
    \hline
    oid & float \\
    \hline
    0 & 0.10 \\
    1 & 0.00 \\
    2 & 0.10 \\
    3 & 0.10 \\
    4 & 0.00 \\
    5 & 0.00 \\
    6 & 0.00 \\
    7 & 0.10 \\
    \hline
  \end{tabular}

  \column{0.5cm}

  $\longrightarrow$

  \column{1cm}

  UNQ\_TOT
  \begin{tabular}{|c|c|}
    \hline
    oid & float \\
    \hline
    0 & 005.225 \\
    1 & 011.500 \\
    2 & 102.080 \\
    3 & 041.250 \\
    4 & 014.250 \\
    5 & 017.990 \\
    6 & 022.330 \\
    7 & 046.937 \\
    \hline
  \end{tabular}

  \end{columns}

\end{frame}

\begin{frame}
  \frametitle{The MIL Language}
  \framesubtitle{OQL to MIL translation}

  \begin{block}{}
  {\tt GRP\_UNQ $:=$ join(GRP\_SEL, SEL\_UNQ, "$=$")}
  \end{block}

  \begin{columns}

  \column{2cm}

  GRP\_SEL
  \begin{tabular}{|c|c|}
    \hline
    oid & oid \\
    \hline
    0 & 0 \\
    0 & 1 \\
    0 & 2 \\
    0 & 3 \\
    \hline
  \end{tabular}

  \column{1.5cm}

  SEL\_UNQ
  \begin{tabular}{|c|c|}
    \hline
    oid & oid \\
    \hline
    0 & 0 \\
    0 & 1 \\
    1 & 2 \\
    2 & 3 \\
    2 & 4 \\
    2 & 5 \\
    3 & 6 \\
    3 & 7 \\
    \hline
  \end{tabular}

  \column{0.5cm}

  $\longrightarrow$

  \column{1cm}

  GRP\_UNQ
  \begin{tabular}{|c|c|}
    \hline
    oid & oid \\
    \hline
    0 & 0 \\
    0 & 1 \\
    0 & 2 \\
    0 & 3 \\
    0 & 4 \\
    0 & 5 \\
    0 & 6 \\
    0 & 7 \\
    \hline
  \end{tabular}

  \end{columns}

\end{frame}

\begin{frame}
  \frametitle{The MIL Language}
  \framesubtitle{OQL to MIL translation}

  \begin{block}{}
  {\tt GRP\_TOT $:=$ join(GRP\_UNQ, UNQ\_TOT, "$=$")}
  \end{block}

  \begin{columns}

  \column{2cm}

  GRP\_UNQ
  \begin{tabular}{|c|c|}
    \hline
    oid & oid \\
    \hline
    0 & 0 \\
    0 & 1 \\
    0 & 2 \\
    0 & 3 \\
    0 & 4 \\
    0 & 5 \\
    0 & 6 \\
    0 & 7 \\
    \hline
  \end{tabular}

  \column{1.5cm}

  UNQ\_TOT
  \begin{tabular}{|c|c|}
    \hline
    oid & float \\
    \hline
    0 & 005.225 \\
    1 & 011.500 \\
    2 & 102.080 \\
    3 & 041.250 \\
    4 & 014.250 \\
    5 & 017.990 \\
    6 & 022.330 \\
    7 & 046.937 \\
    \hline
  \end{tabular}

  \column{0.5cm}

  $\longrightarrow$

  \column{1cm}

  GRP\_TOT
  \begin{tabular}{|c|c|}
    \hline
    oid & float \\
    \hline
    0 & 005.225 \\
    0 & 011.500 \\
    0 & 102.080 \\
    0 & 041.250 \\
    0 & 014.250 \\
    0 & 017.990 \\
    0 & 022.330 \\
    0 & 046.937 \\
    \hline
  \end{tabular}

  \end{columns}

\end{frame}

\begin{frame}
  \frametitle{The MIL Language}
  \framesubtitle{OQL to MIL translation}

  \begin{block}{}
  {\tt GRP\_SUM $:=$ $\lbrace$sum$\rbrace$(GRP\_TOT, GRP\_GRP)}
  \end{block}

  \begin{columns}

  \column{2cm}

  GRP\_TOT
  \begin{tabular}{|c|c|}
    \hline
    oid & float \\
    \hline
    0 & 005.225 \\
    0 & 011.500 \\
    0 & 102.080 \\
    0 & 041.250 \\
    0 & 014.250 \\
    0 & 017.990 \\
    0 & 022.330 \\
    0 & 046.937 \\
    \hline
  \end{tabular}

  \column{1.5cm}

  GRP\_GRP
  \begin{tabular}{|c|c|}
    \hline
    oid & oid \\
    \hline
    0 & 0 \\
    \hline
  \end{tabular}

  \column{0.5cm}

  $\longrightarrow$

  \column{1cm}

  GRP\_SUM
  \begin{tabular}{|c|c|}
    \hline
    oid & float \\
    \hline
    0 & 261.562 \\
    \hline
  \end{tabular}

  \end{columns}

\end{frame}

\begin{frame}
  \frametitle{The MIL Language}
  \framesubtitle{OQL to MIL translation}

  \begin{block}{}
  {\tt table("1", GRP\_YEA, GRP\_SUM)}
  \end{block}

  \begin{columns}

  \column{2cm}

  GRP\_YEA
  \begin{tabular}{|c|c|}
    \hline
    oid & int \\
    \hline
    0 & 98 \\
    \hline
  \end{tabular}
  \column{1.5cm}

  GRP\_SUM
  \begin{tabular}{|c|c|}
    \hline
    oid & float \\
    \hline
    0 & 261.562 \\
    \hline
  \end{tabular}

  \column{0.5cm}

  $\longrightarrow$

  \column{1cm}

  Result:
  \begin{tabular}{|c|c|}
    \hline
    int & float \\
    \hline
    98 & 261.562 \\
    \hline
  \end{tabular}

  \end{columns}

\end{frame}

\section{Running Monet}

\begin{frame}
  \frametitle{Running Monet}
  \framesubtitle{Running the server}

  \begin{block}{Start the Monet server}
  \alert{\tt Mserver} \\
  {\tt \# Monet Database Server V4.6.2} \\
  {\tt \# Copyright (c) 1993-2005, CWI. All rights reserved.} \\
  {\tt \# Compiled for $<$arch$>$; dynamically linked.} \\
  {\tt \# Visit http://monetdb.cwi.nl/ for further info.} \\
  {\tt MonetDB>}
  \end{block}

  \pause

  \begin{block}{Allowing MIL clients}
  {\tt MonetDB>module(mapi);} \\
  {\tt MonetDB>listen(50000).fork();}
  \end{block}

  \pause

  \begin{block}{Shutting down}
  {\tt MonetDB>quit();}
  \end{block}

\end{frame}

\begin{frame}
  \frametitle{Running Monet}
  \framesubtitle{Running the clients}

  \begin{block}{Start the Mapi client}
  \alert{\tt MapiClient} \\
  {\tt \# Monet Database Server V4.6.2} \\
  {\tt \# Copyright (c) 1993-2005, CWI. All rights reserved.} \\
  {\tt \# Compiled for $<$arch$>$; dynamically linked.} \\
  {\tt \# Visit http://monetdb.cwi.nl/ for further info.} \\
  {\tt mil>}
  \end{block}

  \pause

  \begin{block}{Start Mknife}
  \alert{\tt java -jar Mknife-1.6.2-1.jar} \\
  {\tt (choose MIL demo)}
  \end{block}

\end{frame}

\begin{frame}
  \frametitle{Running Monet}
  \framesubtitle{Using the SQL front-end}

  \begin{block}{Start the SQL front end}
  {\tt MonetDB>module(sql\_server);} \\
  {\tt MonetDB>sql\_server\_start();}
  \end{block}

  \pause

  \begin{block}{Use Mapi client}
  \alert{\tt MapiClient -l sql} \\
  {\tt sql>}
  \end{block}

  \pause

  \begin{block}{Use JDBC client}
  \alert{\tt java -jar share/MonetDB/lib/MonetDB\_JDBC.jar -umonetdb} \\
  {\tt (use 'monetdb' as password)}
  \end{block}

\end{frame}

\section{The Implementation of MIL}

\end{document}
