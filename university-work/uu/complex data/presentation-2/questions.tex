\documentclass{article}

\title{Session 9: The MIL Language}
\author{Bogdan Dumitriu}
\date{\today}

\begin{document}

\begin{enumerate}

\item {\bf (Laurence)} In Fig. 6 OQL query and MIL translation, on page 109 it is shown how the 
OQL query is translated to the MIL translation. It is a bit unclear to 
me how it is done exactly. Can u maybe show in a step by step fashion 
how the query is translated.

\item {\bf (Peter)} Can you show an example of how the pump operator is used, and the 
results it creates? (there's some info about it 
implementation/optimization as well, but an example explained would be 
nice)

\item {\bf (Peter)} With the save, load, and remove functions, there is the operand "str 
s". They do not mention it in the description. Is this the global name 
of the bat in the persistant storage?

\item {\bf (Ingmar)} What's the use of the mirror operator, it seems such a strange
operator to me. Why would you want a table with identical columns?

\item {\bf (Ingmar)} On page 6 it is mentioned that Mil supports nested bats. That
sounds really interesting, but what are they (bats within bats,
because that doesn't sound like a bat anymore)?

\end{enumerate}

\newpage

\begin{enumerate}

\item {\bf (Laurence)} In Fig. 6 OQL query and MIL translation, on page 109 it is shown how the 
OQL query is translated to the MIL translation. It is a bit unclear to 
me how it is done exactly. Can u maybe show in a step by step fashion 
how the query is translated.

\item {\bf (Peter)} Can you show an example of how the pump operator is used, and the 
results it creates? (there's some info about it 
implementation/optimization as well, but an example explained would be 
nice)

\item {\bf (Peter)} With the save, load, and remove functions, there is the operand "str 
s". They do not mention it in the description. Is this the global name 
of the bat in the persistant storage?

\item {\bf (Ingmar)} What's the use of the mirror operator, it seems such a strange
operator to me. Why would you want a table with identical columns?

\item {\bf (Ingmar)} On page 6 it is mentioned that Mil supports nested bats. That
sounds really interesting, but what are they (bats within bats,
because that doesn't sound like a bat anymore)?

\end{enumerate}

\newpage

\begin{enumerate}

\item {\bf (Laurence)} In Fig. 6 OQL query and MIL translation, on page 109 it is shown how the 
OQL query is translated to the MIL translation. It is a bit unclear to 
me how it is done exactly. Can u maybe show in a step by step fashion 
how the query is translated.

\item {\bf (Peter)} Can you show an example of how the pump operator is used, and the 
results it creates? (there's some info about it 
implementation/optimization as well, but an example explained would be 
nice)

\item {\bf (Peter)} With the save, load, and remove functions, there is the operand "str 
s". They do not mention it in the description. Is this the global name 
of the bat in the persistant storage?

\item {\bf (Ingmar)} What's the use of the mirror operator, it seems such a strange
operator to me. Why would you want a table with identical columns?

\item {\bf (Ingmar)} On page 6 it is mentioned that Mil supports nested bats. That
sounds really interesting, but what are they (bats within bats,
because that doesn't sound like a bat anymore)?

\end{enumerate}

\newpage

\begin{enumerate}

\item {\bf (Laurence)} In Fig. 6 OQL query and MIL translation, on page 109 it is shown how the 
OQL query is translated to the MIL translation. It is a bit unclear to 
me how it is done exactly. Can u maybe show in a step by step fashion 
how the query is translated.

\item {\bf (Peter)} Can you show an example of how the pump operator is used, and the 
results it creates? (there's some info about it 
implementation/optimization as well, but an example explained would be 
nice)

\item {\bf (Peter)} With the save, load, and remove functions, there is the operand "str 
s". They do not mention it in the description. Is this the global name 
of the bat in the persistant storage?

\item {\bf (Ingmar)} What's the use of the mirror operator, it seems such a strange
operator to me. Why would you want a table with identical columns?

\item {\bf (Ingmar)} On page 6 it is mentioned that Mil supports nested bats. That
sounds really interesting, but what are they (bats within bats,
because that doesn't sound like a bat anymore)?

\end{enumerate}

\newpage

\begin{enumerate}

\item {\bf (Laurence)} In Fig. 6 OQL query and MIL translation, on page 109 it is shown how the 
OQL query is translated to the MIL translation. It is a bit unclear to 
me how it is done exactly. Can u maybe show in a step by step fashion 
how the query is translated.

\item {\bf (Peter)} Can you show an example of how the pump operator is used, and the 
results it creates? (there's some info about it 
implementation/optimization as well, but an example explained would be 
nice)

\item {\bf (Peter)} With the save, load, and remove functions, there is the operand "str 
s". They do not mention it in the description. Is this the global name 
of the bat in the persistant storage?

\item {\bf (Ingmar)} What's the use of the mirror operator, it seems such a strange
operator to me. Why would you want a table with identical columns?

\item {\bf (Ingmar)} On page 6 it is mentioned that Mil supports nested bats. That
sounds really interesting, but what are they (bats within bats,
because that doesn't sound like a bat anymore)?

\end{enumerate}

\newpage

\begin{enumerate}

\item {\bf (Laurence)} In Fig. 6 OQL query and MIL translation, on page 109 it is shown how the 
OQL query is translated to the MIL translation. It is a bit unclear to 
me how it is done exactly. Can u maybe show in a step by step fashion 
how the query is translated.

\item {\bf (Peter)} Can you show an example of how the pump operator is used, and the 
results it creates? (there's some info about it 
implementation/optimization as well, but an example explained would be 
nice)

\item {\bf (Peter)} With the save, load, and remove functions, there is the operand "str 
s". They do not mention it in the description. Is this the global name 
of the bat in the persistant storage?

\item {\bf (Ingmar)} What's the use of the mirror operator, it seems such a strange
operator to me. Why would you want a table with identical columns?

\item {\bf (Ingmar)} On page 6 it is mentioned that Mil supports nested bats. That
sounds really interesting, but what are they (bats within bats,
because that doesn't sound like a bat anymore)?

\end{enumerate}

\newpage

\begin{enumerate}

\item {\bf (Laurence)} In Fig. 6 OQL query and MIL translation, on page 109 it is shown how the 
OQL query is translated to the MIL translation. It is a bit unclear to 
me how it is done exactly. Can u maybe show in a step by step fashion 
how the query is translated.

\item {\bf (Peter)} Can you show an example of how the pump operator is used, and the 
results it creates? (there's some info about it 
implementation/optimization as well, but an example explained would be 
nice)

\item {\bf (Peter)} With the save, load, and remove functions, there is the operand "str 
s". They do not mention it in the description. Is this the global name 
of the bat in the persistant storage?

\item {\bf (Ingmar)} What's the use of the mirror operator, it seems such a strange
operator to me. Why would you want a table with identical columns?

\item {\bf (Ingmar)} On page 6 it is mentioned that Mil supports nested bats. That
sounds really interesting, but what are they (bats within bats,
because that doesn't sound like a bat anymore)?

\end{enumerate}

\newpage

\begin{enumerate}

\item {\bf (Laurence)} In Fig. 6 OQL query and MIL translation, on page 109 it is shown how the 
OQL query is translated to the MIL translation. It is a bit unclear to 
me how it is done exactly. Can u maybe show in a step by step fashion 
how the query is translated.

\item {\bf (Peter)} Can you show an example of how the pump operator is used, and the 
results it creates? (there's some info about it 
implementation/optimization as well, but an example explained would be 
nice)

\item {\bf (Peter)} With the save, load, and remove functions, there is the operand "str 
s". They do not mention it in the description. Is this the global name 
of the bat in the persistant storage?

\item {\bf (Ingmar)} What's the use of the mirror operator, it seems such a strange
operator to me. Why would you want a table with identical columns?

\item {\bf (Ingmar)} On page 6 it is mentioned that Mil supports nested bats. That
sounds really interesting, but what are they (bats within bats,
because that doesn't sound like a bat anymore)?

\end{enumerate}

\newpage

\begin{enumerate}

\item {\bf (Laurence)} In Fig. 6 OQL query and MIL translation, on page 109 it is shown how the 
OQL query is translated to the MIL translation. It is a bit unclear to 
me how it is done exactly. Can u maybe show in a step by step fashion 
how the query is translated.

\item {\bf (Peter)} Can you show an example of how the pump operator is used, and the 
results it creates? (there's some info about it 
implementation/optimization as well, but an example explained would be 
nice)

\item {\bf (Peter)} With the save, load, and remove functions, there is the operand "str 
s". They do not mention it in the description. Is this the global name 
of the bat in the persistant storage?

\item {\bf (Ingmar)} What's the use of the mirror operator, it seems such a strange
operator to me. Why would you want a table with identical columns?

\item {\bf (Ingmar)} On page 6 it is mentioned that Mil supports nested bats. That
sounds really interesting, but what are they (bats within bats,
because that doesn't sound like a bat anymore)?

\end{enumerate}

\newpage

\begin{enumerate}

\item {\bf (Laurence)} In Fig. 6 OQL query and MIL translation, on page 109 it is shown how the 
OQL query is translated to the MIL translation. It is a bit unclear to 
me how it is done exactly. Can u maybe show in a step by step fashion 
how the query is translated.

\item {\bf (Peter)} Can you show an example of how the pump operator is used, and the 
results it creates? (there's some info about it 
implementation/optimization as well, but an example explained would be 
nice)

\item {\bf (Peter)} With the save, load, and remove functions, there is the operand "str 
s". They do not mention it in the description. Is this the global name 
of the bat in the persistant storage?

\item {\bf (Ingmar)} What's the use of the mirror operator, it seems such a strange
operator to me. Why would you want a table with identical columns?

\item {\bf (Ingmar)} On page 6 it is mentioned that Mil supports nested bats. That
sounds really interesting, but what are they (bats within bats,
because that doesn't sound like a bat anymore)?

\end{enumerate}

\newpage

\begin{enumerate}

\item {\bf (Laurence)} In Fig. 6 OQL query and MIL translation, on page 109 it is shown how the 
OQL query is translated to the MIL translation. It is a bit unclear to 
me how it is done exactly. Can u maybe show in a step by step fashion 
how the query is translated.

\item {\bf (Peter)} Can you show an example of how the pump operator is used, and the 
results it creates? (there's some info about it 
implementation/optimization as well, but an example explained would be 
nice)

\item {\bf (Peter)} With the save, load, and remove functions, there is the operand "str 
s". They do not mention it in the description. Is this the global name 
of the bat in the persistant storage?

\item {\bf (Ingmar)} What's the use of the mirror operator, it seems such a strange
operator to me. Why would you want a table with identical columns?

\item {\bf (Ingmar)} On page 6 it is mentioned that Mil supports nested bats. That
sounds really interesting, but what are they (bats within bats,
because that doesn't sound like a bat anymore)?

\end{enumerate}

\newpage

\begin{enumerate}

\item {\bf (Laurence)} In Fig. 6 OQL query and MIL translation, on page 109 it is shown how the 
OQL query is translated to the MIL translation. It is a bit unclear to 
me how it is done exactly. Can u maybe show in a step by step fashion 
how the query is translated.

\item {\bf (Peter)} Can you show an example of how the pump operator is used, and the 
results it creates? (there's some info about it 
implementation/optimization as well, but an example explained would be 
nice)

\item {\bf (Peter)} With the save, load, and remove functions, there is the operand "str 
s". They do not mention it in the description. Is this the global name 
of the bat in the persistant storage?

\item {\bf (Ingmar)} What's the use of the mirror operator, it seems such a strange
operator to me. Why would you want a table with identical columns?

\item {\bf (Ingmar)} On page 6 it is mentioned that Mil supports nested bats. That
sounds really interesting, but what are they (bats within bats,
because that doesn't sound like a bat anymore)?

\end{enumerate}

\newpage

\begin{enumerate}

\item {\bf (Laurence)} In Fig. 6 OQL query and MIL translation, on page 109 it is shown how the 
OQL query is translated to the MIL translation. It is a bit unclear to 
me how it is done exactly. Can u maybe show in a step by step fashion 
how the query is translated.

\item {\bf (Peter)} Can you show an example of how the pump operator is used, and the 
results it creates? (there's some info about it 
implementation/optimization as well, but an example explained would be 
nice)

\item {\bf (Peter)} With the save, load, and remove functions, there is the operand "str 
s". They do not mention it in the description. Is this the global name 
of the bat in the persistant storage?

\item {\bf (Ingmar)} What's the use of the mirror operator, it seems such a strange
operator to me. Why would you want a table with identical columns?

\item {\bf (Ingmar)} On page 6 it is mentioned that Mil supports nested bats. That
sounds really interesting, but what are they (bats within bats,
because that doesn't sound like a bat anymore)?

\end{enumerate}

\newpage

\begin{enumerate}

\item {\bf (Laurence)} In Fig. 6 OQL query and MIL translation, on page 109 it is shown how the 
OQL query is translated to the MIL translation. It is a bit unclear to 
me how it is done exactly. Can u maybe show in a step by step fashion 
how the query is translated.

\item {\bf (Peter)} Can you show an example of how the pump operator is used, and the 
results it creates? (there's some info about it 
implementation/optimization as well, but an example explained would be 
nice)

\item {\bf (Peter)} With the save, load, and remove functions, there is the operand "str 
s". They do not mention it in the description. Is this the global name 
of the bat in the persistant storage?

\item {\bf (Ingmar)} What's the use of the mirror operator, it seems such a strange
operator to me. Why would you want a table with identical columns?

\item {\bf (Ingmar)} On page 6 it is mentioned that Mil supports nested bats. That
sounds really interesting, but what are they (bats within bats,
because that doesn't sound like a bat anymore)?

\end{enumerate}

\newpage

\begin{enumerate}

\item {\bf (Laurence)} In Fig. 6 OQL query and MIL translation, on page 109 it is shown how the 
OQL query is translated to the MIL translation. It is a bit unclear to 
me how it is done exactly. Can u maybe show in a step by step fashion 
how the query is translated.

\item {\bf (Peter)} Can you show an example of how the pump operator is used, and the 
results it creates? (there's some info about it 
implementation/optimization as well, but an example explained would be 
nice)

\item {\bf (Peter)} With the save, load, and remove functions, there is the operand "str 
s". They do not mention it in the description. Is this the global name 
of the bat in the persistant storage?

\item {\bf (Ingmar)} What's the use of the mirror operator, it seems such a strange
operator to me. Why would you want a table with identical columns?

\item {\bf (Ingmar)} On page 6 it is mentioned that Mil supports nested bats. That
sounds really interesting, but what are they (bats within bats,
because that doesn't sound like a bat anymore)?

\end{enumerate}

\newpage

\begin{enumerate}

\item {\bf (Laurence)} In Fig. 6 OQL query and MIL translation, on page 109 it is shown how the 
OQL query is translated to the MIL translation. It is a bit unclear to 
me how it is done exactly. Can u maybe show in a step by step fashion 
how the query is translated.

\item {\bf (Peter)} Can you show an example of how the pump operator is used, and the 
results it creates? (there's some info about it 
implementation/optimization as well, but an example explained would be 
nice)

\item {\bf (Peter)} With the save, load, and remove functions, there is the operand "str 
s". They do not mention it in the description. Is this the global name 
of the bat in the persistant storage?

\item {\bf (Ingmar)} What's the use of the mirror operator, it seems such a strange
operator to me. Why would you want a table with identical columns?

\item {\bf (Ingmar)} On page 6 it is mentioned that Mil supports nested bats. That
sounds really interesting, but what are they (bats within bats,
because that doesn't sound like a bat anymore)?

\end{enumerate}

\end{document}