\documentclass{beamer}

\usepackage{listings}

\lstset{% general command to set parameter(s)
        frame=single, % frame code fragments
        breaklines=true, % break long lines
        basicstyle=\tt\tiny % print whole listing small
        %keywordstyle=\color{black}\bfseries\underbar, % underlined bold black keywords
        %identifierstyle=, % nothing happens
        %commentstyle=\color{white}, % white comments
        %stringstyle=\ttfamily % typewriter type for strings
        %showstringspaces=false % no special string spaces
       }

\mode<presentation>
{
  \usetheme{Warsaw}

  %\setbeamercovered{transparent}
}

\usepackage[english]{babel}

\usepackage[latin1]{inputenc}

\usepackage{times}
\usepackage[T1]{fontenc}
% Or whatever. Note that the encoding and the font should match. If T1
% does not look nice, try deleting the line with the fontenc.

\title{The Haskore Computer Music System}
\subtitle{An example of an Embedded Domain Specific Language in Haskell}

\author{Bogdan Dumitriu \and Jos\'e Pedro Magalh\~aes}

\institute
{
  Department of Computer Science\\
  Utrecht University
}

\date{\today}

% If you wish to uncover everything in a step-wise fashion, uncomment
% the following command:
%\beamerdefaultoverlayspecification{<+->}

\begin{document}

\begin{frame}
  \titlepage
\end{frame}

\begin{frame}
  \frametitle{Outline}
  \tableofcontents%[pausesections]
\end{frame}

\section{Introduction}

\subsection{Definition}

\begin{frame}
  \frametitle{The basics of Haskore}
  \framesubtitle{What is Haskore?}
  \begin{itemize}
    \item Collection of Haskell modules designed for expressing musical structures in Haskell
    \vspace{0.6cm}
    \item Musical objects consist of:
    \begin{itemize}
        \item Primitive notions (notes and rests)
        \item Operations to transform musical objects (transposition and tempo)
        \item Operations to combine musical objects (sequential/parallel compositions)
    \end{itemize}
  \end{itemize}
\end{frame}

\begin{frame}
  \frametitle{The basics of Haskore}
  \framesubtitle{Overall System Diagram}
  \begin{center}
      \pgfdeclareimage[interpolate=true,height=6cm]{diagram}{diagram}
      \pgfuseimage{diagram}
  \end{center}
\end{frame}

\subsection{Usage of Haskell features}
\frame[all:1] {
  \frametitle{The basics of Haskore}
  \framesubtitle{Usage of Haskell features}

  The {\tt Music} data type:
  \begin{lstlisting}
data Music = Note Pitch Dur [NoteAttribute] -- a note   \ atomic
           | Rest Dur                       -- a rest   / objects
           | Music :+: Music                -- sequential composition
           | Music :=: Music                -- parallel composition
           | Tempo (Ratio Int) Music        -- scale the tempo
           | Trans Int Music                -- transposition
           | Instr IName Music              -- instrument label
           | Player PName Music             -- player label
           | Phrase [PhraseAttribute] Music -- phrase attributes
             deriving (Show, Eq)
  \end{lstlisting}
}

  \frame[all:1] {
  \frametitle{The basics of Haskore}
  \framesubtitle{Usage of Haskell features}

  The {\tt Music} data type:
  \begin{lstlisting}
data Music = Note Pitch Dur [NoteAttribute] -- a note   \ atomic
           | Rest Dur                       -- a rest   / objects
           | Music :+: Music                -- sequential composition
           | Music :=: Music                -- parallel composition
           | Tempo (Ratio Int) Music        -- scale the tempo
           | Trans Int Music                -- transposition
           | Instr IName Music              -- instrument label
           | Player PName Music             -- player label
           | Phrase [PhraseAttribute] Music -- phrase attributes
             deriving (Show, Eq)
  \end{lstlisting}
  \begin{itemize}

  \item Higher-order functions:
  \begin{lstlisting}
line, chord :: [Music] -> Music
line = foldr1 (:+:)
chord = foldr1 (:=:)
  \end{lstlisting}

  \end{itemize}}

  \frame[all:1] {
  \frametitle{The basics of Haskore}
  \framesubtitle{Usage of Haskell features}

  The {\tt Music} data type:
  \begin{lstlisting}
data Music = Note Pitch Dur [NoteAttribute] -- a note   \ atomic
           | Rest Dur                       -- a rest   / objects
           | Music :+: Music                -- sequential composition
           | Music :=: Music                -- parallel composition
           | Tempo (Ratio Int) Music        -- scale the tempo
           | Trans Int Music                -- transposition
           | Instr IName Music              -- instrument label
           | Player PName Music             -- player label
           | Phrase [PhraseAttribute] Music -- phrase attributes
             deriving (Show, Eq)
  \end{lstlisting}
  \begin{itemize}

  \item Higher-order functions:
  \begin{lstlisting}
line, chord :: [Music] -> Music
line = foldr1 (:+:)
chord = foldr1 (:=:)
  \end{lstlisting}

  \item Infinite objects:
  \begin{lstlisting}
repeatM :: Music -> Music
repeatM m = m :+: repeatM m
  \end{lstlisting}

  \end{itemize}}

\subsection{Players and Performance}

\frame[all:1]
{
  \frametitle{Players and Performance}
  \framesubtitle{Players - abstraction}
  \begin{itemize}
    \item {\tt Music} abstraction independent from musical interpretation
    \item Haskore models real life!
  \end{itemize}

\begin{lstlisting}
data Player = MkPlayer { pName :: PName
                       , playNote :: NoteFun
                       , interpPhrase :: PhraseFun
                       , notatePlayer :: NotateFun
                       } deriving Show

type PMap = PName -> Player
type NoteFun = Context -> Pitch -> Dur -> [NoteAttribute] -> Performance
type PhraseFun =
  PMap -> Context -> [PhraseAttribute] -> Music -> (Performance,DurT)
type NotateFun = () -- not implemented
\end{lstlisting}
}

\frame[all:1] {
  \frametitle{Players and Performance}
  \framesubtitle{Performance}
  \begin{itemize}
    \item {\tt Performance} $=$ Haskore's internal representation of a music's interpretation
    \item {\tt Music} $\rightarrow$ {\tt Performance} allows for different interpretations
for the same music
  \end{itemize}

\begin{lstlisting}
type Performance = [Event]
data Event = Event {eTime   :: Time
                   ,eInst   :: IName
                   ,ePitch  :: AbsPitch
                   ,eDur    :: DurT
                   ,eVol    :: Volume
                   ,pFields :: [Float]
                   } deriving (Eq,Ord,Show)

perform :: PMap -> Context -> Music -> Performance
\end{lstlisting}
}

\section{Input / Output}
\subsection{MIDI and CSound files}
\begin{frame}
  \frametitle{Input / Output}
  \framesubtitle{MIDI and CSound files}

    \begin{block}{MIDI}
    \small Haskore has also built-in support for writing and reading MIDI
files. MIDI ("Musical Instrument Digital Interface") is a standard
protocol adopted by most, if not all, manufacturers of electronic
instruments. In this way Haskore ensures it will be able to
communicate with virtually any other decent music system.
    \end{block}
    \pause
    \begin{block}{CSound}
    \small Furthermore, Haskore can also output to CSound files. CSound
is a software synthesizer that allows its user to create a virtually
unlimited number of sounds and instruments. By supporting CSound
output, Haskore gives its users access to all the powerful features
of a software sound synthesizer.
    \end{block}
\end{frame}

\section{Mathematic reasoning}
\subsection{Music theory expressed in Haskell}

\frame[all:1] {
  \frametitle{Mathematic reasoning for music}
  %\framesubtitle{}
  \begin{itemize}
    \item Haskore also allows for mathematical reasoning, due to
    \begin{itemize}
        \item Notion of literal interpretation
        \item Haskell being a pure language
    \end{itemize}
    \item Example:
  \end{itemize}

  \begin{block}{Axiom 1}
$\forall r_1, r_2, r_3, r_4 \textnormal{ and } m$:

$\quad\textnormal{Tempo }r_1\ r_2\ (\textnormal{Tempo }r_3\ r_4\ m) = \textnormal{Tempo }(r_1*r_3)\ (r_2*r_4)\ m$
  \end{block}

  \begin{block}{Proof:}
\begin{lstlisting}
  perform dt (Tempo r1 r2 (Tempo r3 r4 m))
= perform (r2*dt/r1) (Tempo r3 r4 m)   -- unfolding perform
= perform (r4*(r2*dt/r1)/r3) m         -- unfolding perform
= perform ((r2*r4)*dt/(r1*r3)) m       -- simple arithmetic
= perform dt (Tempo (r1*r3) (r2*r4) m) -- folding perform
\end{lstlisting}
    \end{block}
}

\section{Examples}
%\subsection{Beethoven's F�r Elise - Original theme}

\begin{frame}
  \frametitle{Haskore examples}
  \framesubtitle{Beethoven's F�r Elise - Original theme}
  \begin{center}
      \pgfdeclareimage[interpolate=true,width=11cm]{furEliseOrig}{furEliseOrig}
      \pgfuseimage{furEliseOrig}
  \end{center}
      \begin{block}{}
  \begin{center}
  {\tt furElise}
    \end{center}
  \end{block}
\end{frame}

\frame[all:1] {
  \frametitle{Haskore examples}
  \framesubtitle{Beethoven's F�r Elise - {\tt Music}}
\begin{lstlisting}
furElise :: Music
furElise =
  let up1 = (Note (E,5) sn []) :+: (Note (Ef,5) sn []) :+: (Note (E,5) sn [])
         :+: (Note (Ef,5) sn []) :+: (Note (E,5) sn []) :+: (Note (B,4) sn [])
         :+: (Note (D,5) sn []) :+: (Note (C,5) sn []) :+: (Note (A,4) en [])
         :+: (Rest sn) :+: (Note (C,4) sn []) :+: (Note (E,4) sn [])
         :+: (Note (A,4) sn []) :+: (Note (B,4) en []) :+: (Rest sn)
      up2 = (Note (E,4) sn []) :+: (Note (Gs,4) sn []) :+: (Note (B,4) sn [])
         :+: (Note (C,5) en []) :+: (Rest sn) :+: (Note (E,4) sn [])
      up3 = (Note (D,4) sn []) :+: (Note (C,5) sn []) :+: (Note (B,4) sn [])
         :+: (Note (A,4) qn [])
      up  = up1 :+: up2 :+: up1 :+: up3

      dw1 = (Rest hn) :+: (Note (A,2) sn []) :+: (Note (E,3) sn [])
         :+: (Note (A,3) sn []) :+: (Rest sn) :+: (Rest en)
         :+: (Note (E,2) sn []) :+: (Note (E,3) sn []) :+: (Note (Gs,3) sn [])
         :+: (Rest sn) :+: (Rest en) :+: (Note (A,2) sn [])
         :+: (Note (E,3) sn []) :+: (Note (A,3) sn []) :+: (Rest sn)
      dw  = dw1 :+: dw1

  in (Instr "Acoustic Grand Piano" (dw :=: up))
\end{lstlisting}
}

%\subsection{Beethoven's F�r Elise - Different performances}
\begin{frame}
  \frametitle{Haskore examples}
  \framesubtitle{Beethoven's F�r Elise - Different performances}

  \begin{block}{Basic performance}
    {\tt\tiny myPerform = perform (const defPlayer) c}\\
    {\tt\tiny \ \ where c = Context 0 defPlayer "" (metro 300 en) (absPitch (C, 1)) 50}
  \end{block}

  \pause

  \begin{block}{Variation 1 - 2 octaves higher, double time}
    {\tt\tiny myPerform = perform (const defPlayer) c}\\
    {\tt\tiny \ \ where c = Context 0 defPlayer "" (metro 150 en) (absPitch (C, 3)) 50}
  \end{block}

  \pause

  \begin{block}{Variation 2 - 1 octave lower, 2 semitones lower, double volume}
    {\tt\tiny myPerform = perform (const defPlayer) c}\\
    {\tt\tiny \ \ where c = Context 0 defPlayer "" (metro 150 en) (absPitch (A, 0)) 100}
  \end{block}

\end{frame}

%\subsection{Beethoven's F�r Elise - Parallel transposition}
\begin{frame}
  \frametitle{Haskore examples}
  \framesubtitle{Beethoven's F�r Elise - Parallel transposition}

  \begin{block}{Transpose piece, delay it \& play in parallel with original}
    {\tt (delay 1 (Trans 5 m)) :=: m}
  \end{block}

\end{frame}

%\subsection{Beethoven's F�r Elise - Inverse}
\begin{frame}
  \frametitle{Haskore examples}
  \framesubtitle{Beethoven's F�r Elise - Inverse}
  \begin{center}
      \pgfdeclareimage[interpolate=true,width=11cm]{furEliseInv}{furEliseInv}
      \pgfuseimage{furEliseInv}
  \end{center}
  \begin{block}{}
  \begin{center}
    {\tt invert furElise}
  \end{center}
  \end{block}
\end{frame}

%\subsection{Beethoven's F�r Elise - Retrograde}
\begin{frame}
  \frametitle{Haskore examples}
  \framesubtitle{Beethoven's F�r Elise - Retrograde}
  \begin{center}
      \pgfdeclareimage[interpolate=true,width=11cm]{furEliseRetr}{furEliseRetr}
      \pgfuseimage{furEliseRetr}
  \end{center}
    \begin{block}{}
  \begin{center}
  {\tt retrograde furElise}
    \end{center}
  \end{block}
\end{frame}

%\subsection{Beethoven's F�r Elise - Inverse retrograde}
\begin{frame}
  \frametitle{Haskore examples}
  \framesubtitle{Beethoven's F�r Elise - Inverse retrograde}
  \begin{center}
      \pgfdeclareimage[interpolate=true,width=11cm]{furEliseInvRetr}{furEliseInvRetr}
      \pgfuseimage{furEliseInvRetr}
  \end{center}
    \begin{block}{}
  \begin{center}
  {\tt (invert . retrograde) furElise}
    \end{center}
  \end{block}
\end{frame}

\end{document}
